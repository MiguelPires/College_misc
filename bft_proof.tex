\section{Correctness Proofs} \label{bft_proof}

This section argues for the correctness of the \acrlong{bgp} protocol in terms of the specified consensus properties.\par

\begin{table}[h!]
	\renewcommand{\arraystretch}{1.5}
	\centering
	\begin{tabularx}{\linewidth}{ |c|X|}
		%\hline
		%\multicolumn{2}{|c|}{Notation}\\
		\hline
		Invariant/Symbol & Definition \\
		\hline
		$\thicksim$ & Equivalence relation between sequences \\
		\hline
		$X \overset{e}{\implies} Y$ & $X$ implies that $Y$ is eventually true \\
		\hline
		$X \sqsubseteq Y$ & The sequence $X$ is a prefix of sequence $Y$ \\
		\hline
		$\mathcal{L}$ & Set of learner processes \\
		\hline
		$\mathcal{P}$ & Set of proposals (commands or sequences of commands) \\
		\hline
		$\mathcal{B}$ & Set of ballots \\
		\hline
		$\bot$ & Empty command \\
		\hline		
		$learned_{l_i}$ & Learner $l_i$'s $learned$ sequence of commands \\
		\hline
		$learned(l_i,s)$ & $learned_{l_i}$ contains the sequence $s$ \\
		\hline
		$maj\_accepted(s,b)$ & $N-f$ acceptors sent phase 2b messages to the learners for sequence $s$ in ballot $b$ \\
		\hline
		$min\_accepted(s,b)$ & $f+1$ acceptors sent phase 2b messages to the learners for sequence $s$ in ballot $b$\\
		\hline
		$proposed(s)$ & A correct proposer proposed $s$ \\
		\hline
		
	\end{tabularx} 
	\vspace{\smallskipamount}
	\caption{BGP proof notation} 
	\label{table:bft_proof}
\end{table}

\subsubsection{Consistency}
\begin{theorem}
	At any time and for any two correct learners $l_i$ and $l_j$, $learned_{l_i}$ and $learned_{l_j}$ can subsequently be extended to equivalent sequences \par
\end{theorem} 
\textbf{Proof:} \par
\parbox{\linewidth}{\strut1. At any given instant, $\forall s,s' \in \mathcal{P}, \forall l_i,l_j \in \mathcal{L}, learned(l_j,s) \land learned(l_i,s') \implies \exists \sigma_1,\sigma_2 \in \mathcal{P} \cup \{\bot\}, s \bullet \sigma_1 \thicksim s' \bullet \sigma_2$}  \par
\indent\indent\parbox{\linewidth}{\strut\textbf{Proof:} }\par
\indent\indent\indent\parbox{\linewidth-\algorithmicindent*3}{\strut1.1. At any given instant, $\forall s,s' \in \mathcal{P}, \forall l_i,l_j \in \mathcal{L}, learned(l_i,s) \land learned(l_j,s') \implies (maj\_accepted(s,b) \lor (min\_accepted(s,b) \land s \bullet \sigma_1 \thicksim x \bullet \sigma_2)) \land (maj\_accepted(s',b') \lor (min\_accepted(s',b') \land s' \bullet \sigma_1 \thicksim x \bullet \sigma_2)), \exists \sigma_1, \sigma_2 \in \mathcal{P} \cup \{\bot\}, \forall x \in \mathcal{P},\forall b,b' \in \mathcal{B}$} \par
\indent\indent\indent\indent\parbox{\linewidth-\algorithmicindent*4}{\strut\textbf{Proof:} A sequence can only be learned in some ballot $b$ if the learner gathers $N-f$ votes (i.e., $maj\_accepted(s,b)$), each containing $N-f$ valid proofs, or if it is universally commutative (i.e., $s \bullet \sigma_1 \thicksim x \bullet \sigma_2,\ \exists \sigma_1, \sigma_2 \in \mathcal{P} \cup \{\bot\}, \forall x \in \mathcal{P}$) and the learner gathers $f+1$ votes (i.e., $min\_accepted(s,b)$).The first case requires gathering $N-f$ votes from each acceptor and validating that each proof corresponds to the correct ballot and value (Algorithm \ref{BFT-Learn} lines \{1-12\}). The second case requires that the sequence must be commutative with any other (Algorithm \ref{BFT-Learn} \{14-18\}). This is encoded in the logical expression $s \bullet \sigma_1 \thicksim x \bullet \sigma_2$ which is true if the learned sequence can be extended with $\sigma_1$ to the same that any other sequence $x$ can be extended to with a possibly different sequence $\sigma_2$, therefore making it impossible to result in a conflict.}

\indent\indent\indent\parbox{\linewidth-\algorithmicindent*3}{\strut1.2. At any given instant, $\forall s,s' \in \mathcal{P},\forall b,b' \in \mathcal{B}, maj\_accepted(s,b) \land maj\_accepted(s',b') \implies \exists \sigma_1,\sigma_2 \in \mathcal{P} \cup \{\bot\}, s \bullet \sigma_1 \thicksim s' \bullet \sigma_2$}\par
\indent\indent\indent\indent\parbox{\linewidth-\algorithmicindent*4}{\strut\textbf{Proof:} We divide the following proof in two main cases: (1.2.1.) sequences $s$ and $s'$ are accepted in the same ballot $b$ and (1.2.2.) sequences $s$ and $s'$ are accepted in different ballots $b$ and $b'$.}\par
\indent\indent\indent\indent\indent\parbox{\linewidth-\algorithmicindent*5}{\strut1.2.1.~At any given instant, $\forall s,s' \in \mathcal{P},\forall b \in \mathcal{B}, maj\_accepted(s,b) \land maj\_accepted(s',b) \implies \exists \sigma_1,\sigma_2 \in \mathcal{P} \cup \{\bot\}, s \bullet \sigma_1 \thicksim s' \bullet \sigma_2$} \par
\indent\indent\indent\indent\indent\indent\parbox{\linewidth}{\strut\textbf{Proof:} Proved by contradiction.}\par
\indent\indent\indent\indent\indent\indent\indent\parbox{\linewidth-\algorithmicindent*5}{\strut1.2.1.1.~At any given instant, $\forall s,s' \in \mathcal{P}, \forall \sigma_1,\sigma_2 \in \mathcal{P} \cup \{\bot \},\forall b \in \mathcal{B}, maj\_accepted(s,b) \land maj\_accepted(s',b) \wedge s \bullet \sigma_1 \not\thicksim s' \bullet \sigma_2$} \par
\indent\indent\indent\indent\indent\indent\indent\indent\parbox{\linewidth}{\strut\textbf{Proof:} Contradiction assumption.}\par
\indent\indent\indent\indent\indent\indent\indent\parbox{\linewidth-\algorithmicindent*5}{\strut1.2.1.2. Take a pair proposals $s$ and $s'$ that meet the conditions of 1.2.1 (and are certain to exist by the previous point), then $s$ and $s'$ contain non-commutative commands}\par
\indent\indent\indent\indent\indent\indent\indent\indent\parbox{\linewidth-\algorithmicindent*6}{\strut\textbf{Proof:} The statement $\forall s,s' \in \mathcal{P}, \forall \sigma_1,\sigma_2 \in \mathcal{P} \cup \{\bot \}, s \bullet \sigma_1 \not\thicksim s' \bullet \sigma_2$ is trivially false because it implies that for any combination of sequences and suffixes, the extended sequences would never be equivalent. Since there must be some $s,s',\sigma_1$ and $\sigma_2$ (for instance $s=s'$ and $\sigma_1=\sigma_2$) for which the extensions are equivalent, then the statement is false.}\par
\indent\indent\indent\indent\indent\indent\indent\parbox{\linewidth}{\strut1.2.1.3. A contradiction is found, Q.E.D. }\par
\indent\indent\indent\indent\indent\parbox{\linewidth-\algorithmicindent*5}{\strut1.2.2.~At any given instant, $\forall s,s' \in \mathcal{P},\forall b,b' \in \mathcal{B}, maj\_accepted(s,b) \land maj\_accepted(s',b') \land b \neq b' \implies \exists \sigma_1,\sigma_2 \in \mathcal{P} \cup \{\bot\}, s \bullet \sigma_1 \thicksim s' \bullet \sigma_2$} 
\indent\indent\indent\indent\indent\indent\parbox{\linewidth-\algorithmicindent*6}{\strut\textbf{Proof:}To prove that values accepted in different ballots are extensible to equivalent sequences, it suffices to prove that for any sequences $s$ and $s'$ accepted at ballots $b$ and $b'$, respectively, such that $b < b'$ then $s \sqsubseteq s'$. By Algorithm \ref{BFT-Acc} lines \{10-15\}, any correct acceptor only votes for its variable $val_a$ when it receives proofs for the same value from other $2f$ acceptors. Therefore, we prove that a $val_a$ that receives $2f+1$ verification messages, is always an extension of a previous $val_a$ that received $2f+1$ verification messages. By Algorithm \ref{BFT-Acc} lines \{31,44\}, $val_a$ only changes when a leader sends a proposal in a classic ballot or when an acceptor sends a sequence in a fast ballot.\par}
\indent\indent\indent\indent\indent\indent\parbox{\linewidth-\algorithmicindent*6}{In the first case, $val_a$ is substituted by the leader's proposal which means we must prove it's an extension of any $val_a$ that previously obtained $2f+1$ verification votes. By Algorithm \ref{BFT-Lead} lines \{24-47\}, the leader's proposal is prefixed by the largest of the proven sequences (i.e., $val_a$ sequences that receives $2f+1$ votes) relayed by a quorum of acceptors in phase $1b$ messages. Note that, the verification in Algorithm \ref{BFT-Acc} line \{26\} prevents a Byzantine leader from sending a sequence that isn't an extension of previous proved sequences. If all the proven sequences are extensible to the same sequence then it's clear that the largest proven sequence is an extension of all the proven sequences relayed in phase $1b$ messages. However, since that is the same result we are trying to prove, we must use induction to do so. As our induction step, we assume that for some ballot $b$ and any two proven sequences $s$ and $s'$, the following is true $s \sqsubseteq s' \lor s' \sqsubseteq s$ (i.e., all sequences are extensible to the same sequence). In the next ballot $b'=b+1$, by picking the largest sequence as in Algorithm \ref{BFT-Lead} lines \{41-47\}, the leader will create a proposal that is an extension of any previously $val_a$ value that gathered $2f+1$ verification messages. Therefore, if all the proven sequences in some ballot $b$ are extensible to the same sequence, then every proven $val_a$ in subsequent ballots will be an extension of those values and, therefore, also an extensible to the same sequence.\par
In the second case, a proposer's proposal $c$ is appended to $val_a$ which means that any acceptors new value $val_a'$ will be extensible to the same sequence as the previous $val_a$ since, by definition of the append operation, $c \bullet val_a \sqsubseteq val_a'$.\par}
\indent\indent\indent\parbox{\linewidth-\algorithmicindent*3}{\strut1.3. For any pair of proposals $s$ and $s'$, at any given instant, $\forall x \in \mathcal{P}, \exists \sigma_1,\sigma_2,\sigma_3,\sigma_4 \in \mathcal{P} \cup \{\bot\}, \forall b,b' \in \mathcal{B}, (maj\_accepted(s,b) \lor (min\_accepted(s,b) \land s \bullet \sigma_1 \thicksim x \bullet \sigma_2)) \land (maj\_accepted(s',b') \lor (min\_accepted(s',b') \land s \bullet \sigma_1 \thicksim x \bullet \sigma_2)) \implies s \bullet \sigma_3 \thicksim s' \bullet \sigma_4$}\par
\indent\indent\indent\indent\parbox{\linewidth}{\strut\textbf{Proof:} By 1.2 and by definition of $s \bullet \sigma_1 \thicksim x \bullet \sigma_2$.}\par
\indent\indent\indent\parbox{\linewidth-\algorithmicindent*3}{\strut1.4. At any given instant, $\forall s,s' \in \mathcal{P}, \forall l_i,l_j \in \mathcal{L}, learned(l_i,s)\ \land\ learned(l_j,s') \implies \exists \sigma_1,\sigma_2 \in \mathcal{P} \cup \{\bot\}, s \bullet \sigma_1 \thicksim s' \bullet \sigma_2$ }\par
\indent\indent\indent\indent\parbox{\linewidth}{\strut\textbf{Proof:} By 1.1 and 1.3.}\par
\indent\indent\indent\parbox{\linewidth}{\strut1.5. Q.E.D. }\par
\parbox{\linewidth-\algorithmicindent*3}{\strut2. At any given instant, $\forall l_i,l_j \in \mathcal{L}, learned(l_j,learned_j) \land learned(l_i,learned_i) \implies \exists \sigma_1,\sigma_2 \in \mathcal{P} \cup \{\bot\}, learned_i \bullet \sigma_1 \thicksim learned_j \bullet \sigma_2$}\par
\indent\indent\parbox{\linewidth}{\strut\textbf{Proof:} By 1.}\par
\parbox{\linewidth}{\strut3. Q.E.D.} \par

\subsubsection{Non-Triviality}
\begin{theorem}
If all proposers are correct, $learned_l$ can only contain proposed commands. \label{N-T1} \par
\end{theorem} 
\textbf{Proof:} \par
\parbox{\linewidth}{\strut1. At any given instant, $\forall l_i \in \mathcal{L}, \forall s \in \mathcal{P}, learned(l_i,s) \implies \forall x \in \mathcal{P}, \exists \sigma \in \mathcal{P}, \forall b \in \mathcal{B}, \ maj\_accepted(s,b) \lor (min\_accepted(s,b) \land  (s \thicksim x \bullet \sigma \lor x \thicksim s \bullet \sigma))$ }\par
\indent\indent\parbox{\linewidth}{\strut\textbf{Proof:} By Algorithm \ref{BFT-Acc} lines \{15,29,42\} and Algorithm \ref{BFT-Learn} lines \{1-18\}, if a correct learner learned a sequence $s$ at any given instant then either $N-f$ or $f+1$ (if $s$ is universally commutative) acceptors must have executed phase $2b$ for $s$.}\par
\parbox{\linewidth}{\strut2. At any given instant, $\forall s \in \mathcal{P}, \forall b \in \mathcal{B}, maj\_accepted(s,b) \lor min\_accepted(s,b) \implies proposed(s)$ }\par
\indent\indent\parbox{\linewidth}{\strut\textbf{Proof:} By Algorithm \ref{BFT-Acc} lines \{17-22\}, for either $N-f$ or $f+1$ acceptors to accept a proposal it must have been proposed by a proposer.}\par
\parbox{\linewidth}{\strut3. At any given instant, $\forall s \in \mathcal{P}, learned(l_i,s) \implies proposed(s),\forall l_i \in \mathcal{L}$}\par
\indent\indent\parbox{\linewidth}{\strut\textbf{Proof:} By 1 and 2.}\par
\parbox{\linewidth}{\strut4. Q.E.D.}\par

\subsubsection{Stability}
\begin{theorem}
If $learned_l = s$ then, at all later times, $s \sqsubseteq learned_l$, for any sequence $s$ and correct learner $l$\looseness=-1 \par
\end{theorem} 
\textbf{Proof:} By Algorithm \ref{BFT-Learn} lines \{12,18,20-26\}, a correct learner can only append new commands to its $learned$ command sequence.

\subsubsection{Liveness}
\begin{theorem}
For any proposal $s$ from a correct proposer, and correct learner $l$, eventually $learned_l$ contains $s$\par
\end{theorem} 
\parbox{\linewidth}{\textbf{Proof:}} \par
\parbox{\linewidth}{\strut1. $\forall\ l_i \in \mathcal{L},\forall s,x \in \mathcal{P}, \exists \sigma \in \mathcal{P}, \forall b \in \mathcal{B}, maj\_accepted(s,b) \lor (min\_accepted(s,b) \land  (s \thicksim x \bullet \sigma \lor x \thicksim s \bullet \sigma))\overset{e}{\implies} learned(l_i,s)$}\par
\indent\indent\parbox{\linewidth}{\strut\textbf{Proof:} By Algorithm \ref{BFT-Acc} lines \{10-15,28-29,41-42\} and Algorithm \ref{BFT-Learn} lines \{1-18\}, when either $N-f$ or $f+1$ (if $s$ is universally commutative) acceptors accept a sequence $s$ (or some equivalent sequence), eventually $s$ will be learned by any correct learner.}\par
\parbox{\linewidth}{\strut2. $\forall s \in \mathcal{P}, proposed(s) \overset{e}{\implies} \forall x \in \mathcal{P}, \exists \sigma \in \mathcal{P}, \forall b \in \mathcal{B}, maj\_accepted(s,b) \lor (min\_accepted(s,b) \land  (s \thicksim x \bullet \sigma \lor x \thicksim s \bullet \sigma))$} \par
\indent\indent\parbox{\linewidth}{\strut\textbf{Proof:} A proposed sequence is either conflict-free when its incorporated into every acceptor's current sequence or it creates conflicting sequences at different acceptors. In the first case, it's accepted by a quorum (Algorithm \ref{BFT-Acc} lines \{10-15,28-29,41-42\}) and, in the second case, it's sent in phase $1b$ messages to the in leader in the next ballot (Algorithm \ref{BFT-Acc} lines \{1-4\}) and incorporated in the next proposal (Algorithm \ref{BFT-Lead} lines \{24-47\}).} \par
\parbox{\linewidth}{\strut3. $\forall l_i \in \mathcal{L}, \forall s \in \mathcal{P}, proposed(s) \overset{e}{\implies} learned(l_i,s)$} \par
\indent\indent\parbox{\linewidth}{\strut\textbf{Proof:} By 1 and 2.} \par
\parbox{\linewidth}{\strut4. Q.E.D.}