\section{Background and related work}
\label{sec:related} 
\subsection{Paxos and its variants} \label{Paxos} 

The Paxos protocol family solves consensus by finding an equilibrium in face of the well-known FLP impossibility result~\cite{FLP85}. It does this by always guaranteeing safety despite asynchrony, but
%at the same time making the observation that most of the time systems have periods during which they can be considered synchronous, since long delays are often sporadic and temporary. Therefore, Paxos only
foregoing progress during the temporary periods of asynchrony, or if more than $f$ faults occur for a system of $N > 2f$ replicas~\cite{L01}. The classic form of Paxos uses a set of proposers, acceptors and learners, runs in a sequence of ballots, and employs two phases (numbered 1 and 2), with a similar message pattern: proposer to acceptors (phase 1a or 2a), acceptors to proposer (phase 1b or 2b), and, in phase 2b, also acceptors to learners. To ensure progress during synchronous periods, proposals are serialized by a distinguished proposer, which is called the leader.\par
Paxos is most commonly deployed as Multi (Decree)-Paxos, which provides an optimization of the basic message pattern by omitting the first phase of messages from all but the first ballot for each leader~\cite{Renesse2011}. This means that a leader only needs to send a \textit{phase 1a} message once and subsequent proposals may be sent directly in \textit{phase 2a} messages. This reduces the message pattern in the common case from five message delays to just three (from proposing to learning). \par
Fast Paxos observes that it is possible to improve on the previous latency (in the common case) by allowing proposers to propose values directly to acceptors~\cite{L06}. To this end, the protocol distinguishes between fast and classic ballots, where fast ballots bypass the leader by sending proposals directly to acceptors and classic ballots work as in the original Paxos protocol. The reduced latency of fast ballots comes at the added cost of using a quorum size of $N-e$ instead of a classic majority quorum, where $e$ is the number of faults that can be tolerated while using fast ballots. In addition, instead of the usual requirement that $N> 2f$, to ensure that fast and classic quorums intersect, a new requirement must be met: $N > 2e+f$. This means that if we wish to tolerate the same number of faults for classic and fast ballots (i.e., $e=f$), then the minimum number of replicas is $3f+1$ instead of the usual $2f+1$. Since fast ballots only take two message steps (\textit{phase 2a} messages between a proposer and the acceptors, and \textit{phase 2b} messages between acceptors and learners), there is the possibility of two proposers concurrently proposing values and generating a conflict, which must be resolved by falling back to a recovery protocol.\looseness=-1 \par
Generalized Paxos improves the performance of Fast Paxos by addressing the issue of collisions. In particular, it allows acceptors to accept different sequences of commands as long as non-commutative operations are totally ordered \cite{Lamport2005}. In the original description, non-commutativity between operations is generically represented as an interference relation. In this context, Generalized Paxos abstracts the traditional consensus problem of agreeing on a single value to the problem of agreeing on an increasing set of values. \textit{C-structs} provide this increasing sequence abstraction and allow the definition of different consensus problems. If we define the sequence of learned commands of a learner $l_i$ as a \textit{c-struct} $learned_{l_i}$, then the consistency requirement for generalized consensus can be defined as: $learned_{l_1}$ and $learned_{l_2}$ must have a \textit{common upper bound}, for all learners $l_1$ and $l_2$. This means that, for any $learned_{l_1}$ and $learned_{l_2}$, there must exist some \textit{c-struct} of which they are both prefixes. This prohibits interfering commands from being concurrently accepted because no subsequent \textit{c-struct} would extend them both. \par
%Defining \textit{c-structs} as command histories enables acceptors to agree on different sequences of commands and still preserve consistency as long as dependence relationships are not violated. This means that commutative commands can be ordered differently regarding each other but interfering commands must preserve the same order across each sequence at any learner. This guarantees that solving the consensus problem for histories is enough to implement a state-machine replicated system. \par
More recently, other Paxos variants have been proposed to address specific issues. For example, Mencius~\cite{Mao2008} avoids the latency penalty in wide-area deployments of having a single leader, through which every proposal must go through. In Mencius, the leader of each round rotates between every process: the leader of round $i$ is process $p_k$, such that $k = n\ mod\ i$.  
% Leaders with nothing to propose can skip their turn by proposing a \textit{no-op}. If a leader is slow or faulty, the other replicas can execute \textit{phase 1} to revoke the leader's right to propose a value, but they can only propose a \textit{no-op} instead \cite{Mao2008}. Considering that non-leader replicas can only propose \textit{no-ops}, a \textit{no-op} command from the leader can be accepted in a single message delay since there is no chance of another value being accepted. If some non-leader server revokes the leader's right to propose and suggests a \textit{no-op}, then the leader can still suggest a value $v \neq$ \textit{no-op}, which will eventually be accepted as long as $l$ is not permanently suspected. Mencius also takes advantage of commutativity by allowing out-of-order commits, where values $x$ and $y$ can be learned in different orders by different learners if there does not exist a dependence relationship between them.
Another variant is Egalitarian Paxos (EPaxos), which achieves a better throughput than Paxos by removing the bottleneck caused by having a leader \cite{Moraru2013}. To avoid choosing a leader, the proposal of commands for a command slot is done in a decentralized manner, taking advantage of the commutativity observations made by Generalized Paxos \cite{Lamport2005}. Conflicts between commands are handled by having replicas reply with a command dependency, which then leads to falling back to using another protocol phase with $f+\lfloor\frac{f+1}{2}\rfloor$ replicas.
%If two replicas unknowingly propose commands concurrently, one will commit its proposal in one round trip after getting replies from a quorum of replicas. However, some replica will see that another command was concurrently proposed and may interfere with the already committed command. If the commands are non-commutative then the replica must reply with a dependency between the commands, committing its command in two rounds trips. This commit latency is achieved by using a \textit{fast-path quorum} of $f+\lfloor\frac{f+1}{2}\rfloor$ replicas. Similarly to Mencius, EPaxos achieves a substantially higher throughput than Multi-Paxos.

\subsection{Byzantine fault tolerant replication} \label{Non-Crash}
%Non-crash fault models emerged to cope with the effect of malicious attacks and software errors. These models (e.g., the arbitrary fault model) assume a stronger adversary than previous crash fault models.
Consensus in the Byzantine model was originally defined by Lamport et al.~\cite{LSP82}. Almost two decades later, a surge of research in the area started with the PBFT protocol, which solves consensus for state machine replication with $3f+1$ replicas while tolerating up to $f$ Byzantine faults \cite{CL99}. In PBFT, the system moves through configurations called \textit{views}, in which one replica is the primary and the remaining replicas are the backups. The protocol proceeds in a sequence of steps, where messages are sent from the client to the primary, from the primary to the backups, followed by two all-to-all steps between the replicas, with the last step proceeding in parallel with sending a reply to the clients. \par
%The safety property of the algorithm requires that operations be totally ordered. The protocol starts when a client sends a request for an operation to the primary, which in turn assigns a sequence number to the request and multicasts a \textit{pre-prepare} message to the backups. If a backup replica accepts the pre-prepare message, it multicasts a \textit{prepare} message and adds both messages to its log. Both of these phases are needed to ensure that the requested operation is totally ordered at every correct replica, therefore satisfying the protocol's safety property. After receiving $2f$ prepare messages, a replica multicasts a \textit{commit} message and commits the message to its log when it has received $2f$ commit messages from other replicas. The liveness property requires that clients must eventually receive replies to their requests, provided that there are at most $\lfloor\frac{n-1}{3}\rfloor$ faults and the transmission time does not increase continuously. Backups can trigger new views after increasingly long timeouts if they suspect the leader to be Byzantine. \par
Zeno is a Byzantine fault tolerance state machine replication protocol that trades availability for consistency~\cite{Singh2009}. In particular, it offers eventual consistency by allowing state machine commands to execute in a \textit{weak quorum} of  $f+1$ replicas. This ensures that at least one correct replica will execute the request and commit it to the linear history, but does not guarantee the intersection property that is required for linearizability. \par
The closest related work is Fast Byzantine Paxos (FaB), which solves consensus in the Byzantine setting within two message communication steps in the common case, while requiring $5f+1$ acceptors to ensure safety and liveness \cite{Martin2006}. A variant that is proposed in the same paper is the Parameterized FaB Paxos protocol, which generalizes FaB by decoupling replication for fault tolerance from replication for performance. As such, the Parameterized FaB Paxos requires $3f+2t+1$ replicas to solve consensus, preserving safety while tolerating up to $f$ faults and completing in two steps despite up to $t$ faults. Therefore, FaB Paxos is a special case of Parameterized FaB Paxos where $t=f$. It has also been shown that $N>5f$ is a lower bound on the number of acceptors required to guarantee 2-step execution in the Byzantine model. In this sense, the FaB protocol is tight since it requires $5f+1$ acceptors to provide the same guarantees.\par
In comparison to FaB Paxos, our protocol, Byzantine Generalized Paxos (BGP), requires a lower number of acceptors than what is stipulated by FaB's lower bound. However, this does not constitute a violation of the result since BGP does not guarantee a 2-step execution in the Byzantine scenario. Instead, BGP only provides a two communication step latency when proposed sequences are commutative with any other concurrently proposed sequence. In other words, BGP leverages a weaker performance guarantee to decrease the requirements regarding the minimum number of processes. In particular, a proposed sequence may not gather enough votes to be learned in the ballot in which it is proposed, either due to Byzantine behaviour or contention between non-commutative commands. However, any sequence is guaranteed to eventually be learned in a way such that non-commutative commands are totally ordered at any correct learner.\looseness=-1
\raggedbottom