\section{Correctness Proofs} \label{vft_proofs}

This section argues for the correctness of the Visigoth Generalized Paxos protocol in terms of the specified consensus properties.\par


\begin{table}[h!]
	\renewcommand{\arraystretch}{1.5}
	\centering
	\begin{tabularx}{\linewidth}{ |c|X|}
		%\hline
		%\multicolumn{2}{|c|}{Notation}\\
		\hline
		Invariant/Symbol & Definition \\
		\hline
		$\thicksim$ & Equivalence relation between sequences \\
		\hline
		$X \overset{e}{\implies} Y$ & $X$ implies that $Y$ is eventually true \\
		\hline
		$X \sqsubseteq Y$ & The sequence $X$ is a prefix of sequence $Y$ \\
		\hline
		$\mathcal{L}$ & Set of learner processes \\
		\hline
		$\mathcal{P}$ & Set of proposals (commands or sequences of commands) \\
		\hline
		$\mathcal{B}$ & Set of ballots \\
		\hline
		$\bot$ & Empty command \\
		\hline
		$learned_{l_i}$ & Learner $l_i$'s $learned$ sequence of commands \\
		\hline
		$learned(l_i,s)$ & $learned_{l_i}$ contains the sequence $s$ \\
		\hline
		$min\_accepted(s,b)$ & $u+1$ acceptors sent \textit{phase 2b} messages to the learners for sequence $s$ in ballot $b$\\
		\hline
		$proposed(s)$ & A correct proposer sent a message with proposal $s$ \\
		\hline
		$acc\_send(x,s,b)$ & At least $x$ acceptors have sent verification messages for the sequence $s$ in ballot $b$\\
		\hline
		$wait\_learn(t,l,s,b)$ & After $t$ time units have passed since the learner $l$ received the first \textit{phase 2b} message with a sequence $s$ in ballot $b$, $N-s$ \textit{phase 2b} messages haven't been gathered\\
		\hline
  	\end{tabularx} 
	\caption{VGP proof notation} 
	\label{table:vft_proof}
\end{table}

\subsubsection{Consistency}
\begin{theorem}At any time and for any two correct learners $l_i$ and $l_j$, $learned_{l_i}$ and $learned_{l_j}$ can subsequently be extended to equivalent sequences \par
\end{theorem} 
\textbf{Proof:} \par
\parbox{\linewidth-\algorithmicindent}{\strut1. At any given instant, $\forall seq,seq' \in \mathcal{P}, \forall l_i,l_j \in \mathcal{L}, learned(l_j,seq) \land learned(l_i,seq') \implies \exists \sigma_1,\sigma_2 \in \mathcal{P} \cup \{\bot\}, seq \bullet \sigma_1 \thicksim seq' \bullet \sigma_2$}  \par
\indent\indent\parbox{\linewidth-\algorithmicindent*2}{\strut\textbf{Proof:} }\par
\indent\indent\indent\parbox{\linewidth-\algorithmicindent*3}{\strut1.1. At any given instant, $\forall seq,seq' \in \mathcal{P}, \forall l_i,l_j \in \mathcal{L}, \forall b,b' \in \mathcal{B}, learned(l_i,seq) \land learned(l_j,seq') \implies \exists \sigma_1, \sigma_2 \in \mathcal{P}, \forall x \in \mathcal{P}, (acc\_send(N-s, seq,b) \lor (acc\_send(N-u, seq,b) \land wait\_learn(3T,l_i,seq,b)) \lor (min\_accepted(seq,b) \land seq \bullet \sigma_1 \thicksim x \bullet \sigma_2)) \land (acc\_send(N-s, seq',b') \lor (acc\_send(N-u, seq',b') \land wait\_learn(3T,l_j,seq',b')) \lor (min\_accepted(seq',b') \land seq' \bullet \sigma_1 \thicksim x \bullet \sigma_2 ))$} \par
\indent\indent\indent\indent\parbox{\linewidth-\algorithmicindent*4}{\strut\textbf{Proof:} A sequence can be learned if: (1) the learner gathers either $N-s$ votes (Algorithm \ref{VFT-Learn}, lines \{1-17\}), (2) the learner gathers $N-u$ votes after a timeout of $3T$ occurs and a verification quorum is gathered (Algorithm \ref{VFT-Learn}, lines \{25-28,30-45\}) or (3) if the sequence is universally commutative  and the learner gathers $f+1$ votes (Algorithm \ref{VFT-Learn}, lines \{19-23\}). The latter is encoded in the logical expression $s \bullet \sigma_1 \thicksim x \bullet \sigma_2$ which is true if the learned sequence can be extended with $\sigma_1$ to the same that any other sequence $x$ can be extended to with a possibly different sequence $\sigma_2$, therefore making it impossible to result in a conflict since the definition of conflicting sequences are sequences which cannot be extended to equivalent sequences.}
\indent\indent\indent\parbox{\linewidth-\algorithmicindent*3}{\strut1.2.~At any given instant, $\forall seq,seq',x \in \mathcal{P}, \exists \sigma_1,\sigma_2,\sigma_3,\sigma_4 \in \mathcal{P} \cup \{\bot\}, \forall b,b' \in \mathcal{B}, (acc\_send(N-s, seq,b) \lor (acc\_send(N-u, seq,b) \land wait\_learn(3T,l_i,seq,b))) \land (acc\_send(N-s, seq',b') \lor (acc\_send(N-u, seq',b') \land wait\_learn(3T,l_j,seq',b'))) \implies \exists \sigma_5,\sigma_6 \in \mathcal{P} \cup \{\bot\}, seq \bullet \sigma_5 \thicksim seq' \bullet \sigma_6$}\par
\indent\indent\indent\indent\parbox{\linewidth-\algorithmicindent*4}{\strut\textbf{Proof:} We divide this proof in two main cases: (1.2.1.) sequences $seq$ and $seq'$ are accepted in the same ballot $b$ and (1.2.2.) sequences $seq$ and $seq'$ are accepted in different ballots $b$ and $b'$.}\par
\indent\indent\indent\indent\indent\parbox{\linewidth-\algorithmicindent*5}{\strut1.2.1.~At any given instant, $\forall seq,seq' \in \mathcal{P}, \forall b \in \mathcal{B}, (acc\_send(N-s, seq,b) \lor (acc\_send(N-u, seq,b) \land wait\_learn(3T,l_i,seq,b))) \land (acc\_send(N-s, seq',b) \lor (acc\_send(N-u, seq',b) \land wait\_learn(3T,l_j,seq',b))) \implies \exists \sigma_1,\sigma_2  \in \mathcal{P} \cup \{\bot\}, seq \bullet \sigma_1 \thicksim seq' \bullet \sigma_2$} \par
\indent\indent\indent\indent\indent\indent\parbox{\linewidth-\algorithmicindent*6}{\strut\textbf{Proof:} To prove that if any two sequences are accepted then they must be extensible to equivalent sequences, we must prove that two non-commutative sequences can't both gather a quorum of \textit{phase 2b} messages to be learned. In order to gather for correct acceptors to send a \textit{phase 2b} message for a sequence, they must gather $N-f$ verification messages (Algorithm \ref{VFT-Acc}, lines \{18-23\}). A correct acceptor will not send phase verification messages for two non-commutative sequences because it will only receive the commands in them once and therefore, only assemble them in one possible serialization (Algorithm \ref{VFT-Acc}, lines \{10,13\}). Since we know that a correct acceptor will only send one phase verification message for one possibly non-commutative sequence, we must prove that, in any two gathered quorums, the intersection between them contains at least one correct process.\strut}
\indent\indent\indent\indent\indent\indent\parbox{\linewidth-\algorithmicindent*6}{\strut
Gathered quorums can be of size $Q_1=N-s$ or $N-u \leq Q_2 < N-s$, which means there are three possible combinations: (1) two quorums of size $Q_1$, (2) two quorums of size $Q_2$ and (3) one quorum of size $Q_1$ and one quorum of size $Q_2$.\par
To guarantee that two quorums of size $Q_1$ intersect in at least one correct process, it must hold that $N-s+N-s-N> o$. Assuming the worst case scenario where $N=u+o+s+1$:}
\indent\indent\indent\indent\indent\indent\parbox{\linewidth-\algorithmicindent*6}{
\begin{align*}
	N-s+N-s-N> o \\
	N-2s > o \\
	u+o+s+1-2s > o \\
	u+1-s>0 \\
	u+1>s 
\end{align*}}
\indent\indent\indent\indent\indent\indent\parbox{\linewidth-\algorithmicindent*6}{
Since we already assume that $u>s$, the condition holds.\par
Given two quorums with a size that is somewhere in the  interval $N-u \leq Q_2 < N-s$, when the first quorum is gathered the sizes are $q_1= N-u+a$ and $q_2=N-u+b$. The quantities $a$ and $b$ are such that $q_1$ and $q_2$ may be in some arbitrary position within the aforementioned interval. This means that there are $u-a-s$ crashed processes since, out of the maximum number of processes $u$ that can fail to participate in the quorum, $a$ did participate and $s$ did not but may be slow instead of faulty. Therefore, when these two quorums are assembled, we know that the system size is $N'=N-u+a+s$. To guarantee that the intersection between the quorums contains at least one correct process, the following must hold $q_1+q_2-N'>o$. }
\indent\indent\indent\indent\indent\indent\parbox{\linewidth-\algorithmicindent*6}{
\begin{align*}
	q_1+q_2-N'>o \\
	N-u+a+N-u+b-N+u-a-s>o \\
	N-u+b-s>o \\
	u+o+s+1-u+b-s>o\\
	b+1>0
\end{align*}}
\indent\indent\indent\indent\indent\indent\parbox{\linewidth-\algorithmicindent*6}{
Given a quorum of size $Q_1=N-s$ and a quorum of size $Q_2=N-u+a$, at least $u-a-s$ processes are faulty since, out of the maximum number of processes $u$ that can fail to participate in the quorum, $a$ did participate and $s$ did not but may be slow instead of faulty. This means that, when the quorums are gathered, the system's size is $N'=N-u+a+s$. To guarantee that the intersection between quorums contains at least one correct process, the following must hold $Q_1+Q_2-N'>o$: }
\indent\indent\indent\indent\indent\indent\parbox{\linewidth-\algorithmicindent*6}{
\begin{align*}
	Q_1+Q_2-N'>o \\
	N-s+N-u+a-N+u-a-s>o\\
	N-2s>o\\
	u+o+s+1-2s>o\\
	u-s+1>0
\end{align*}
Since we already assume that $u>s$, the condition holds.}
\indent\indent\indent\indent\indent\parbox{\linewidth-\algorithmicindent*5}{\strut1.2.2.~At any given instant, $\forall s,s' \in \mathcal{P},\forall b,b' \in \mathcal{B}, (acc\_send(N-s, seq,b) \lor (acc\_send(N-u, seq,b) \land wait\_learn(3T,l_i,seq,b))) \land (acc\_send(N-s, seq',b') \lor (acc\_send(N-u, seq',b') \land wait\_learn(3T,l_j,seq',b'))) \land b \neq b' \implies \exists \sigma_1,\sigma_2 \in \mathcal{P} \cup \{\bot\}, seq \bullet \sigma_1 \thicksim seq' \bullet \sigma_2$} 
\indent\indent\indent\indent\indent\indent\parbox{\linewidth-\algorithmicindent*6}{\strut\textbf{Proof:} To prove that values accepted in different ballots are extensible to equivalent sequences, it suffices to prove that for any sequences $seq$ and $seq'$ accepted at ballots $b$ and $b'$, respectively, such that $b < b'$ then $seq \sqsubseteq seq'$. By Algorithm \ref{VFT-Acc} lines \{18-23\}, any correct acceptor only votes for a value in variable $val_a$ when it receives $2f+1$ proofs for a matching value. Therefore, we prove that a value $val_a$ that receives $2f+1$ verification messages is always an extension of a previous $val_a$ that received $2f+1$ verification messages. By Algorithm \ref{VFT-Acc} lines \{36,47\}, $val_a$ only changes when a leader sends a proposal in a classic ballot or when a proposer sends a sequence in a fast ballot.\strut}
\indent\indent\indent\indent\indent\indent\parbox{\linewidth-\algorithmicindent*6}{\strut In the first case, $val_a$ is substituted by the leader's proposal which means we must prove that this proposal is an extension of any $val_a$ that previously obtained $2f+1$ verification votes. By Algorithm \ref{VFT-Lead} lines \{24-39,41-48\}, the leader's proposal is prefixed by the largest of the proven sequences (i.e., $val_a$ sequences that receives $2f+1$ votes) relayed by a quorum of acceptors in \textit{phase 1b} messages. Note that, the verification in Algorithm \ref{VFT-Acc} line \{31\} prevents a Byzantine leader from sending a sequence that isn't an extension of previous proved sequences. Since the verification phase prevents non-commutative sequences from being accepted by a quorum, every proven sequence in a ballot is extensible to equivalent sequences which means that the largest proven sequence is simply the most up-to-date sequence of the previous ballot. To prove that the leader can only propose extensions to previous values by picking the largest proven sequence as its proposal's prefix, we need to assert that a proven sequence is an extension any previous sequence. However, since that is the same result that we are trying to prove, we must use induction to do so:\strut}
\indent\indent\indent\indent\indent\indent\indent\parbox{\linewidth-\algorithmicindent*7}{\strut\textbf{1. Base Case}: In the first ballot, any proven sequence will be an extension of the empty command $\bot$ and, therefore, an extension of the previous sequence.\strut}
\indent\indent\indent\indent\indent\indent\indent\parbox{\linewidth-\algorithmicindent*7}{\strut\textbf{2. Induction Hypothesis}: Assume that, for some ballot $b$, any sequence that gathers $2f+1$ verification votes from acceptors is an extension of previous proven sequences.\strut}
\indent\indent\indent\indent\indent\indent\indent\parbox{\linewidth-\algorithmicindent*7}{\strut\textbf{3. Inductive Step}: By the quorum intersection property, in a classic ballot $b+1$, the \textit{phase 1b} quorum will contain ballot $b$'s proven sequences. Given the largest proven sequence $seq$ in the \textit{phase 1b} quorum (which, by our hypothesis, is an extension of any previous proven sequences), by picking $seq$ as the prefix of its \textit{phase 2a} proposal (Algorithm \ref{VFT-Lead}, lines \{41-48\}), the leader will assemble a proposal that is an extension of any previous proven sequence.\strut}
\indent\indent\indent\indent\indent\indent\parbox{\linewidth-\algorithmicindent*6}{\strut In the second case, a proposer's proposal $c$ is appended to an acceptor's $val_a$ variable. By definition of the append operation, $val_a \sqsubseteq val_a \bullet c$ which means that the acceptor's new value $val_a \bullet c$ is an extension of previous ones.\par}
\indent\indent\indent\parbox{\linewidth-\algorithmicindent*3}{\strut1.3. For any pair of proposals $seq$ and $seq'$, at any given instant, $\forall x \in \mathcal{P}, \exists \sigma_1,\sigma_2,\sigma_3,\sigma_4 \in \mathcal{P} \cup \{\bot\}, \forall b,b' \in \mathcal{B},
 (acc\_send(N-s, seq,b) \lor (acc\_send(N-u, seq,b) \land wait\_learn(3T,l_i,seq,b)) \lor (min\_accepted(seq,b) \land seq \bullet \sigma_1 \thicksim x \bullet \sigma_2)) \land (acc\_send(N-s, seq',b') \lor (acc\_send(N-u, seq',b') \land wait\_learn(3T,l_j,seq',b') \lor (min\_accepted(seq',b') \land seq \bullet \sigma_1 \thicksim x \bullet \sigma_2)))	\implies seq \bullet \sigma_3 \thicksim seq' \bullet \sigma_4$}\par
\indent\indent\indent\indent\parbox{\linewidth-\algorithmicindent*4}{\strut\textbf{Proof:} By 1.2 and by definition of $seq \bullet \sigma_i \thicksim x \bullet \sigma_j$}\par
\indent\indent\indent\parbox{\linewidth-\algorithmicindent*3}{\strut1.4. At any given instant, $\forall seq,seq' \in \mathcal{P}, \forall l_i,l_j \in \mathcal{L}, learned(l_i,seq)\ \land\ learned(l_j,seq') \implies \exists \sigma_1,\sigma_2 \in \mathcal{P} \cup \{\bot\}, seq \bullet \sigma_1 \thicksim seq' \bullet \sigma_2$ }\par
\indent\indent\indent\indent\parbox{\linewidth}{\strut\textbf{Proof:} By 1.1 and 1.3.}\par
\indent\indent\indent\parbox{\linewidth}{\strut1.5. Q.E.D. }\par


\indent\parbox{\linewidth-\algorithmicindent}{\strut2. At any given instant, $\forall l_i,l_j \in \mathcal{L}, learned(l_j,learned_j) \land learned(l_i,learned_i) \implies \exists \sigma_1,\sigma_2 \in \mathcal{P} \cup \{\bot\}, learned_i \bullet sigma_1 \thicksim learned_j \bullet \sigma_2$}\par
\indent\indent\parbox{\linewidth}{\strut\textbf{Proof:} By 1.}\par
\indent\parbox{\linewidth-\algorithmicindent}{\strut3. Q.E.D.} \par

\subsubsection{Nontriviality}
\begin{theorem}
If all proposers are correct, $learned_l$ can only contain proposed commands \par
\end{theorem} 
\textbf{Proof:} \par
\parbox{\linewidth-\algorithmicindent}{\strut1. At any given instant, $\forall l_i \in \mathcal{L}, \forall seq \in \mathcal{P}, learned(l_i,seq) \implies \forall x \in \mathcal{P}, \exists \sigma_1,\sigma_2 \in \mathcal{P} \cup \{\bot\},\forall b \in \mathcal{B},  acc\_send(N-s, seq,b) \lor (acc\_send(N-u, seq,b) \land wait\_learn(3T,l_i,seq,b)) \lor (min\_accepted(seq,b) \land seq \bullet \sigma_1 \thicksim x \bullet \sigma_2)$ }\par
\indent\indent\parbox{\linewidth-\algorithmicindent*2}{\strut\textbf{Proof:} By Algorithm \ref{VFT-Acc} lines \{18-23,33-34,44-45\} and Algorithm \ref{VFT-Learn} lines \{1-17,19-23\}, if a correct learner learned a sequence $seq$ at any given instant then either a quorum was gathered or $f+1$ (if $seq$ is universally commutative) acceptors must have executed \textit{phase 2b} for $seq$.}\par
\parbox{\linewidth-\algorithmicindent}{\strut2. At any given instant, $\forall seq,x \in \mathcal{P}, \exists \sigma_1,\sigma_2 \in \mathcal{P}, \forall b \in \mathcal{B}, acc\_send(N-s, seq,b) \lor (acc\_send(N-u, seq,b) \land wait\_learn(3T,l_i,seq,b)) \lor (min\_accepted(seq,b) \land seq \bullet \sigma_1 \thicksim x \bullet \sigma_2) \implies proposed(seq)$ }\par
\indent\indent\parbox{\linewidth-\algorithmicindent*2}{\strut\textbf{Proof:} By Algorithm \ref{VFT-Acc} lines \{10-13, 15-16\}, for either a quorum to be gathered or $f+1$ acceptors to accept a proposal, it must have been proposed by a proposer (note that the leader is considered a distinguished proposer).}\par
\parbox{\linewidth}{\strut3. At any given instant, $\forall l_i \in \mathcal{L},\forall seq \in \mathcal{P}, learned(l_i,seq) \implies proposed(seq)$}\par
\indent\indent\parbox{\linewidth}{\strut\textbf{Proof:} By 1 and 2.}\par
\parbox{\linewidth}{\strut4. Q.E.D.}\par

\subsubsection{Stability}
\begin{theorem}
If $learned_l = seq$ then, at all later times, $seq \sqsubseteq learned_l$, for any sequence $seq$ and correct learner $l$ \par \label{S-T1}
\end{theorem} 
\textbf{Proof:} By Algorithm \ref{VFT-Learn} lines \{17,23,28\}, a correct learner can only append new commands to its $learned$ command sequence.

\subsubsection{Liveness}
\begin{theorem}
For any correct learner $l$ and any proposal $seq$ from a correct proposer, eventually $learned_l$ contains $seq$ \label{L-T1} \par
\end{theorem} 
\parbox{\linewidth}{\textbf{Proof:}} \par
\parbox{\linewidth-\algorithmicindent}{\strut1. $\forall\ l_i \in \mathcal{L},\forall seq,x \in \mathcal{P}, \exists \sigma_1,\sigma_2 \in \mathcal{P} \cup \{\bot\}, \forall b \in \mathcal{B}, acc\_send(N-s, seq,b) \lor (acc\_send(N-u, seq,b) \land wait\_learn(3T,l_i,seq,b)) \lor (min\_accepted(seq,b) \land seq \bullet \sigma_1 \thicksim x \bullet \sigma_2) \overset{e}{\implies} learned(l_i,seq)$}\par
\indent\indent\parbox{\linewidth-\algorithmicindent*2}{\strut\textbf{Proof:} By Algorithm \ref{VFT-Acc} lines \{17-22,32-33,45-46\} and Algorithm \ref{VFT-Learn} lines \{1-17,19-23\}, if either a quorum is gathered or $f+1$ (if $seq$ is universally commutative) acceptors accept a sequence $seq$ (or some equivalent sequence), eventually $seq$ will be learned by any correct learner.}\par
\parbox{\linewidth-\algorithmicindent}{\strut2. $\forall seq \in \mathcal{P}, proposed(seq) \overset{e}{\implies} \forall x \in \mathcal{P}, \exists \sigma_1,\sigma_2 \in \mathcal{P} \cup \{\bot\}, \forall b \in \mathcal{B}, acc\_send(N-s, seq,b) \lor (acc\_send(N-u, seq,b) \land wait\_learn(3T,l_i,seq,b)) \lor (min\_accepted(seq,b) \land seq \bullet \sigma_1 \thicksim x \bullet \sigma_2)$} \par
\indent\indent\parbox{\linewidth-\algorithmicindent*2}{\strut\textbf{Proof:} A proposed sequence is either conflict-free when it's incorporated into every acceptor's current sequence or it creates conflicting sequences at different acceptors. In the first case, it's accepted by a quorum (Algorithm \ref{VFT-Acc}, lines \{18-23\}) and learned after the votes reach the learners. In the second case, eventually the next ballot is initiated and the sequence is sent in \textit{phase 1b} messages to the leader (Algorithm \ref{VFT-Acc}, lines \{1-4\}). After being sent to the leader, the sequence is incorporated in the next proposal and sent to the acceptors a long with others proposed commands (Algorithm \ref{VFT-Lead}, lines \{24-39,41-48\}).} \par
\parbox{\linewidth}{\strut3. $\forall l_i \in \mathcal{L}, \forall seq \in \mathcal{P}, proposed(seq) \overset{e}{\implies} learned(l_i,seq)$} \par
\indent\indent\parbox{\linewidth}{\strut\textbf{Proof:} By 1 and 2.} \par
\parbox{\linewidth}{\strut4. Q.E.D.}