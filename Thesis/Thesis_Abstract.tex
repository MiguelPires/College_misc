%%%%%%%%%%%%%%%%%%%%%%%%%%%%%%%%%%%%%%%%%%%%%%%%%%%%%%%%%%%%%%%%%%%%%%%%
%                                                                      %
%     File: Thesis_Abstract.tex                                        %
%     Tex Master: Thesis.tex                                           %
%                                                                      %
%     Author: Andre C. Marta                                           %
%     Last modified :  2 Jul 2015                                      %
%                                                                      %
%%%%%%%%%%%%%%%%%%%%%%%%%%%%%%%%%%%%%%%%%%%%%%%%%%%%%%%%%%%%%%%%%%%%%%%%

\section*{Abstract}

% Add entry in the table of contents as section
\addcontentsline{toc}{section}{Abstract}

One of the most recent members of the Paxos family of
protocols is Generalized Paxos. This variant of Paxos has the characteristic that it departs from the original specification of consensus, allowing for a
weaker safety condition where different processes can have different views of a sequence being agreed upon. However, much like its original Paxos counterpart, Generalized Paxos does not have a simple implementation. Furthermore, with the recent practical adoption of Byzantine fault tolerant protocols, it is timely and important to understand how Generalized Paxos can be implemented in the Byzantine model. The same point can be made with respect to the Visigoth model which targets datacenter-like environments by allowing parameterizable fault and synchrony assumptions. This dissertation makes several contributions. First, we provide a description of Generalized Paxos that is easier to understand, based on a simpler specification of consensus, along with a pseudocode specification that can be readily implemented. Second, we extend the protocol to the Byzantine fault model, providing also a pseudocode description to ease its mapping into a code implementation. This contribution is supplemented  with correctness proofs and a discussion of relevant extensions and optimizations. Third, we extend this implementation to the Visigoth fault model, providing with it an accessible pseudocode description and correctness proofs. 

\vfill

\textbf{\Large Keywords:} Paxos, Consensus, Fault Tolerance, Byzantine Fault Model

