\vspace{-0.3cm}

\section{Conclusion and discussion}
\vspace{-0.2cm}
% and Concluding remarks}
\label{sec:disc}
%
We presented a simplified description of the Generalized Paxos specification and protocol, 
and an implementation of Generalized Paxos that is resilient against Byzantine faults.
We now draw some lessons and outline some extensions to our protocol that present interesting directions for future work and hopefully
a better understanding of its practical applicability.

%\shortnegspace

\noindent \textbf{Handling faults in the fast case.}
A result that was stated in the original Generalized Paxos
paper~\cite{Lamport2005} is that to tolerate $f$ crash faults and
allow for fast ballots whenever there are up to $e$ crash faults, the
total system size $N$ must uphold two conditions:
$N > 2f$ and $N > 2e+f$.
Additionally, the fast and classic quorums must be of size $N-e$ and $N-f$, respectively. This implies that there is a price to pay in terms of number of replicas and quorum size for being able to run fast operations during faulty periods.
An interesting observation is that since Byzantine fault tolerance already requires a total system size of $3f+1$ and a quorum size of $2f+1$, we are able to amortize the cost of both features, i.e., we are able to tolerate the maximum number of faults for fast execution without paying a price in terms of the replication factor and quorum size.

%\shortnegspace

\noindent \textbf{Extending the protocol to universally commutative commands.}
% Generalized Paxos vs Paxos spec.
A downside of the use of commutative commands in the
context of Generalized Paxos is that the commutativity check is done
at runtime, to determine if non-commutative commands have
been proposed concurrently.
This raises the possibility of extending the protocol to handle
commands that are universally commutative, i.e., commute with every
other command. For these commands, it is known before executing them
that they will not generate any conflicts, and therefore it is not
necessary to check them against concurrently executing commands.  This
allows us to optimize the protocol by decreasing the number of phase
$2b$ messages required to learn to a smaller $f+1$ quorum. Since, by
definition, these sequences are guaranteed to never produce conflicts,
the $N-f$ quorum is not required to prevent learners from learning
conflicting sequences. Instead, a quorum of $f+1$ is sufficient to
ensure that a correct acceptor saw the command and will eventually
propagate it to a quorum of $N-f$ acceptors. This optimization is particularly useful in the context of 
geo-replicated systems, since it can be significantly faster
to wait for the $f+1$st message instead of the $N-f$th one.\looseness=-1

% Weakly consistent replication - also diff order but ok to have different v

\noindent \textbf{Generalized Paxos and weak consistency.}
%The Byzantine Generalized Paxos protocol tackles two challenges in two different avenues of research, fault tolerance and relaxed consistency models. By specifying the generalized consensus problem,
The key distinguishing feature of the specification of Generalized
Paxos~\cite{Lamport2005} is allowing learners to learn concurrent
proposals in a different order, when the proposals commute. This idea
is closely related to the work on weaker consistency models like eventual or
causal consistency~\cite{Ahamad1995}, or consistency models that mix
strong and weak consistency levels like RedBlue~\cite{Li2012}, which attempt
to decrease the cost of executing operations by reducing coordination
requirements between replicas. 
The link between the two models becomes clearer with the introduction of 
universally commutative commands in the previous paragraph.
In the case of weakly consistent replication,
weakly consistent requests can be executed as if they were universally
commutative, even if in practice that may not be the case. E.g., checking 
the balance of a bank account and making a deposit do not commute since
the output of the former depends on their relative order. However,
some systems prefer to run both as weakly consistent operations, even
though it may cause executions that are not explained by a sequential
execution, since the semantics are still acceptable given
that the final state that is reached is the same and no invariants 
of the application are violated~\cite{Li2012}.



% Extension to diff replica grps


%\textbf{Optimizations} One possible optimization of the Byzantine Generalized Paxos protocol leverages universally commutative commands or sequences of commands, which we define as sequences which commute with any other. Universally commutative sequences allows us to reduce latency by decreasing the number of phase $2b$ messages required to learn to a smaller $f+1$ quorum. Since, by definition, these sequences are guaranteed to never produce conflicts, the $N-f$ quorum isn't required to prevent learners from learning conflicting sequences. Instead, a $f+1$ quorum is sufficient for the learner to be sure that the proposed sequence was proposed by a correct proposer. 


% \section{Conclusion}
% \label{sec:conc}
% In this paper, we presented a simplified description of the Generalized Paxos protocol and specification, 
% which is meant to pave the way for new avenues of research in the area. 
% In addition, we modify the protocol to tolerate Byzantine faults, and we prove the correctness of this protocol. 
% In the future, we would like to implement and evaluate the protocol, in addition to gaining a better understanding of its practical applicability.
