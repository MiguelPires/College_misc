\subsection{Correctness Proofs}

This section argues for the correctness of the Byzantine Generalized Paxos protocol in terms of the specified consensus properties.\par


\begin{table}[h!]
	\renewcommand{\arraystretch}{1.5}
	\centering
	\begin{tabularx}{\linewidth}{ |c|X|}
		%\hline
		%\multicolumn{2}{|c|}{Notation}\\
		\hline
		Invariant/Symbol & Definition \\
		\hline
		$\thicksim$ & Equivalence relation between sequences \\
		\hline
		$\mathcal{L}$ & Set of learner processes \\
		\hline
		$X \overset{e}{\implies} Y$ & $X$ implies that $Y$ is eventually true \\
		\hline
		$learned_{l_i}$ & Learner $l_i$'s $learned$ sequence of commands \\
		\hline
		$learned(l_i,b,s)$ & $learned_{l_i}$ contains sequence $s$ at the end of ballot $b$  \\
		\hline
		$maj\_accepted(b,s)$ & At least $N-f$ acceptors broadcasted their acceptance votes to the learners\\
		\hline
		$verified(b,c)$ & At least $N-f$ learners have verified that command $c$ commutes with any other command learned in ballot $b$\\
		\hline
		$fast(b)$ & Ballot $b$ is a fast ballot\\
		\hline
	\end{tabularx} 
	\caption{Proof notation} 
	\label{table:1}
\end{table}

\subsubsection{Consistency}
\begin{theorem}
At any time and for any two learners $l_i$ and $l_j$, $learned_{l_i}$ and $learned_{l_j}$ can subsequently be extended to equivalent sequences \label{C-T1}
\end{theorem} 
\textbf{Proof:} \par
1. At the end of a ballot, $learned_{l_i} \thicksim learned_{l_j},\ \forall\ i,j \in \mathcal{L}$\par
\indent\indent\textbf{Proof:} Lemma \ref{C-L4} \par
2. At any moment of a ballot, $learned_{l_i}$ and $learned_{l_j}$ are extendable to equivalent sequences \par
\indent\indent\textbf{Proof:} By 1, if any two learned sequences are equivalent at the end of a ballot then, by definition, they are extendable to equivalent sequences\par
3. Q.E.D \par
\indent\indent\textbf{Proof:} By 1 and 2\par

\begin{lemma}
At the end of a ballot, $learned_{l_i} \thicksim learned_{l_j},\ \forall\ i,j \in \mathcal{L}$ \label{C-L4} \par
\end{lemma}
\textbf{Proof:} \par
1.  $learned(l_j,b,s) \overset{e}{\implies} learned(l_i,b,s),\ \forall\ lj, l_i \in \mathcal{L}$\par
\indent\indent\textbf{Proof:} Lemma \ref{C-L3} \par
2. Eventually, $learned_{l_i} \thicksim learned_{l_j}$  \par
\indent\indent\textbf{Proof:} If two learned sequences contain the same commands (1) and they are all commutative (Lemma \ref{C-L5}), they sequences are equivalent.\par
3. Q.E.D \par
\indent\indent\textbf{Proof:} By 1 and 2\par

\begin{lemma}
	$learned(l_j,b,s) \overset{e}{\implies} learned(l_i,b,s),\ \forall\ lj, l_i \in \mathcal{L}$ \label{C-L3} \par
\end{lemma}
\textbf{Proof:} \par
1. $learned(l_j,b,s) \implies maj\_accepted(b,s)$\par
\indent\indent\textbf{Proof:} Lemma \ref{C-L1} \par
2. $maj\_accepted(b,s) \overset{e}{\implies} learned(l_i,b,s)$ \par
\indent\indent\textbf{Proof:} Lemma \ref{C-L2} \par
3. $learned(l_j,b,s) \overset{e}{\implies} learned(l_i,b,s),\ \forall\ l_j,l_i \in \mathcal{L}$ \par
\indent\indent\textbf{Proof:} By 1 and 2\par
4. Q.E.D \par


\begin{lemma}
$learned(l_i,b,s) \land learned(l_j,b,s') \land \mathcal{S} = \mathcal{S'} \implies s \thicksim s'$ \todo{What I mean with this is that $s$ and $s'$ must have the same elements, just not necessarily in the same order} \label{C-L5} \par
\end{lemma}
\textbf{Proof:} By Algorithm 4 lines \{10-16, 35-42\}, two commands or sequences of commands are only learned in the same ballot if they commute.


\begin{lemma}
$learned(l_i,b,s) \implies maj\_accepted(b,s),\ \forall\ l_i \in \mathcal{L}$ \label{C-L1} \par
\end{lemma} 

\textbf{Proof:} By Algorithm \ref{BFT-Acc} lines \{14-16\} and Algorithm \ref{BFT-Learn} lines \{8-14\}, if a learner learned a sequence $s$ in a ballot $b$ then a majority of acceptors must have executed phase $2b$ for $s$ in the same ballot.

\begin{lemma}
$maj\_accepted(b,s) \overset{e}{\implies} learned(l_i,b,s),\ \forall\ l_i \in \mathcal{L}$ \label{C-L2} \par
\end{lemma}
\textbf{Proof:} By Algorithm \ref{BFT-Acc} lines \{10-16, 26-42\} and Algorithm \ref{BFT-Learn} lines \{8-14\}, when a quorum of $N-f$ acceptors accepts a sequence $s$ in ballot $b$, eventually s will be learned by any correct learner during ballot $b$.


\subsubsection{Non-Triviality}
\begin{theorem}
$learned_l$ can only contain proposed commands \label{N-T1} \par
\end{theorem} 

\textbf{Proof:} \par
1. $learned(l_i,b,s) \implies maj\_accepted(b,s),\ \forall\ l_i \in \mathcal{L}$ \par
\indent\indent\textbf{Proof:} Lemma \ref{C-L1} \par
2. $maj\_accepted(b,s) \implies proposed(b,s)$ \par
\indent\indent\textbf{Proof:} Lemma \ref{N-L1}\par
3. $learned(l_i,b,s) \implies proposed(b,s)$ \par
\indent\indent\textbf{Proof:} By 1 and 2 \par
4. Q.E.D \par

\begin{lemma}
$maj\_accepted(b,s) \implies proposed(b,s)$ \label{N-L1} \par
\end{lemma}

\textbf{Proof:} By Algorithm \ref{BFT-Acc} lines \{14-16, 26-42\}, for $N-f$ acceptors to accept a proposal it must have been proposed by a proposer.

\subsubsection{Stability}
\begin{theorem}
If $learned_l = v$ then, at all later times, $v \sqsubseteq learned_l$, for any $l$ and $v$ \par \label{S-T1}
\end{theorem} 
\textbf{Proof:} By Algorithm \ref{BFT-Learn} lines \{8-14\}, a learner can only append new commands to its $learned$ command sequence

%\begin{lemma}
%$fast(b+1) \implies learned_{l,b} \sqsubseteq learned_{l,b+1} ,\ %\forall\ l \in \mathcal{L}$ \label{S-L1} \par
%\end{lemma} 
