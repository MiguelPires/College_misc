\documentclass[9pt,pdftex,a4paper]{article}%
%\documentclass{llncs}
\pdfoutput=1
%This is a template for producing LIPIcs articles. 
%See lipics-manual.pdf for further information.
%for A4 paper format use option "a4paper", for US-letter use option "letterpaper"
%for british hyphenation rules use option "UKenglish", for american hyphenation rules use option "USenglish"
% for section-numbered lemmas etc., use "numberwithinsect"
 \usepackage{enumitem}
\setlist{nolistsep}
%\setenumerate{itemsep=0pt}
\usepackage{microtype}%if unwanted, comment out or use option "draft"

%\graphicspath{{./graphics/}}%helpful if your graphic files are in another directory
\usepackage{amssymb}
%\bibliographystyle{plainurl}% the recommended bibstyle
\usepackage{multicol}
\usepackage{amsmath}
%\usepackage{amsthm}
\usepackage{framed}
\usepackage{float}
\floatstyle{boxed} 
\restylefloat{figure}
%\expandafter\def\csname ver@subfig.sty\endcsname{}
\usepackage{tikz}
\usepackage{tabularx}
\usepackage{cleveref}
\crefname{section}{\S}{\S\S}
\usepackage{hyperref}
\usepackage{macros}
\usepackage[ruled]{algorithm}
\usepackage{algorithmicx}
\usepackage{multicol}
\usepackage{caption}

\usepackage{algpseudocode}
\usepackage{ioa_code}
\usepackage{graphicx}
\usepackage[textsize=tiny]{todonotes}
\usepackage{listings}
\usepackage{authblk}
\lstset{numbers=left,numberblanklines=false}
% \newcommand\Small{\fontsize{7.6}{7.6}\selectfont}
% \newcommand*\LSTfont{\Small\ttfamily\SetTracking{encoding=*}{-60}\lsstyle}

\newcommand\shortnegspace{\vspace{-.75em}}

\lstdefinestyle{customc}{
    belowcaptionskip=1\baselineskip,
    breaklines=true,
    frame=L,	
    xleftmargin=\parindent,
    language=C,
    showstringspaces=false,
    escapeinside={//}{\^^M},
    basicstyle=\LSTfont, %\scriptsize\ttfamily,
    keywordstyle=\bfseries\color{green!40!black},
    commentstyle=\itshape\color{gray!60!black},
    identifierstyle=\color{blue!50!black},
    stringstyle=\color{orange},
    numbers=left,                    % where to put the line-numbers; possible values are (none, left, right)
    numbersep=5pt,                   % how far the line-numbers are from the code
    numberstyle=\tiny\color{black},  % the style that is used for the line-numbers
    otherkeywords={then,word,process_local,type,xbegin,xabort,xend}
}

\def\P{\ensuremath{\mathcal{P}}}
\def\DP{\ensuremath{\Diamond\mathcal{P}}}
\def\DS{\ensuremath{\Diamond\mathcal{S}}}
%\def\T{\ensuremath{\mathcal{T}}}
\def\Time{\mathbb{T}}
\def\S{\ensuremath{\mathcal{S}}}
\def\D{\ensuremath{\mathcal{D}}}
\def\W{\ensuremath{\mathcal{W}}}
\def\A{\ensuremath{\mathcal{A}}}
\def\B{\ensuremath{\mathcal{B}}}
\def\F{\ensuremath{\mathcal{F}}}
\def\R{\ensuremath{\mathcal{R}}}
\def\N{\ensuremath{\mathcal{N}}}
\def\I{\ensuremath{\mathcal{I}}}
\def\O{\ensuremath{\mathcal{O}}}
\def\Q{\ensuremath{\mathcal{Q}}}
\def\K{\ensuremath{\mathcal{K}}}
\def\L{\ensuremath{\mathcal{L}}}
\def\M{\ensuremath{\mathcal{M}}}
\def\V{\ensuremath{\mathcal{V}}}
\def\E{\ensuremath{\mathcal{E}}}
\def\C{\ensuremath{\mathcal{C}}}
\def\T{\ensuremath{\mathcal{T}}}
\def\X{\ensuremath{\mathcal{X}}}
\def\Y{\ensuremath{\mathcal{Y}}}
\def\Nat{\ensuremath{\mathbb{N}}}
\def\Om{\ensuremath{\Omega}}
\def\ve{\varepsilon}
\def\fd{failure detector}
\def\cfd{\ensuremath{?\P+\DS}}
\def\afd{timeless}
\def\env{\ensuremath{\mathcal{E}}}
%\def\faulty{unreliable}
\def\bounded{one-shot}
\def\cons{\textit{cons}}
\def\val{\textit{val}}
\def\code{\textit{code}}

\def\HSS{\mathit{h}}
\def\argmin{\mathit{argmin}}
\def\proper{\mathit{proper}}
\def\content{\mathit{content}}
\def\Level{\mathit{L}}
\def\Blocked{\mathit{Blocked}}
\def\Set{\mathit{Set}}
\def\TS{\mathit{TS}}
\def\shared{\mathit{shared}}
\def\exclusive{\mathit{exclusive}}

\newcommand{\correct}{\mathit{correct}}
\newcommand{\RNum}[1]{\uppercase\expandafter{\romannumeral #1\relax}}
\newcommand	{\faulty}{\mathit{faulty}}
\newcommand{\infi}{\mathit{inf}}
\newcommand{\live}{\mathit{live}}
\newcommand{\true}{\mathit{true}}
\newcommand{\false}{\mathit{false}}
\newcommand{\stable}{\mathit{Stable}}
\newcommand{\setcon}{\mathit{setcon}}
\newcommand{\remove}[1]{}

\newcommand{\Wset}{\textit{Wset}}
\newcommand{\Rset}{\textit{Rset}}
\newcommand{\Dset}{\textit{Dset}}

%\newcommand{\parts}{\textit{parts}}
\newcommand{\txns}{\textit{txns}}

\newcommand{\Read}{\textit{read}}
\newcommand{\Write}{\textit{write}}
\newcommand{\TryC}{\textit{tryC}}
\newcommand{\TryA}{\textit{tryA}}
\newcommand{\ok}{\textit{ok}}
%%%%%%%%%%%%%%%%%%%%%%%%%%%%%%%%%%%%%%%%%%%%%%%%%%%%%%%%%%%%%%%%%%%%%%%

%\newtheorem{theorem}{Theorem}
%\newtheorem{reptheorem}[]{Theorem \ref{th:sl}}
%\newreptheorem{theorem}{Theorem}
%\newtheorem{axiom}[theorem]{Axiom}
%\newtheorem{case}[theorem]{Case}
%\newtheorem{claim}[theorem]{Claim}
%\newtheorem{proposition}{Proposition}
%\newtheorem{explanation}[theorem]{Explanation}
%\newtheorem{remark}[theorem]{Remark}
%\newtheorem{fact}[theorem]{Fact}
% \newtheorem{conclusion}[theorem]{Conclusion}
% \newtheorem{condition}[theorem]{Condition}
%\newtheorem{conjecture}[theorem]{Conjecture}
%\newtheorem{corollary}[theorem]{Corollary}
% \newtheorem{criterion}[theorem]{Criterion}
%\newtheorem{definition}{Definition}
% \newtheorem{exercise}[theorem]{Exercise}
%\newtheorem{lemma}[theorem]{Lemma}
% \newtheorem{notation}[theorem]{Notation}
% \newtheorem{problem}[theorem]{Problem}
%\newtheorem{proposition}[theorem]{Proposition}
% \newtheorem{remark}[theorem]{Remark}
% \newtheorem{solution}[theorem]{Solution}
% \newtheorem{summary}[theorem]{Summary}
%\newtheorem{observation}[theorem]{Observation}
%\newenvironment{proof}[1][Proof]{\noindent\textbf{#1.} }{\hfill $\Box$\\[2mm]} %\rule{0.5em}{0.5em}\\}
%\newenvironment{proofsketch}[1][Proof sketch]{\noindent\textbf{#1.} }{\hfill $\Box$\\[2mm]} %\rule{0.5em}{0.5em}\\}

\newenvironment{reptheorem}[1][Theorem]{\noindent\textbf{#1}}{}
\def\lf{\tiny}
\def\rrnnll{\setcounter{linenumber}{0}}
\def\nnll{\refstepcounter{linenumber}\lf\thelinenumber}
\newcounter{linenumber}
\newenvironment{algorithmfig}{\hrule\vskip 3mm}{ \vskip 3mm \hrule }

% \newenvironment{proofsketch}[1][Proof sketch]{\noindent\textbf{#1.} }{\hfill $\Box$\\[2mm]}
% \newtheorem{claim}[theorem]{Claim}
% Author macros::begin %%%%%%%%%%%%%%%%%%%%%%%%%%%%%%%%%%%%%%%%%%%%%%%%
\title{Generalized Paxos Made Byzantine\\ (and Less Complex)}

% \thanks{This paper is a regular submission and is eligible for the best student paper award (Miguel Pires is a full-time student).}

%% Please provide for each author the \author and \affil macro, even when authors have the same affiliation, i.e. for each author there needs to be the  \author and \affil macros
\usepackage{url}

\author{
Miguel Pires$^1$~~~Srivatsan Ravi$^2$~~~Rodrigo Rodrigues$^1$ \\
$^1$\normalsize INESC-ID and Instituto Superior T\'{e}cnico (U.\ Lisboa \\
$^2$\normalsize University of Southern California
}

% \author{Miguel Pires\inst{1} \and Srivatsan Ravi\inst{2} \and Rodrigo Rodrigues\inst{1}}
% 
% \institute{INESC-ID and Instituto Superior T\'{e}cnico (U.\ Lisboa) \\
% \email{miguel.pires@tecnico.ulisboa.pt, rodrigo.rodrigues@inesc-id.pt} 
% \and 
% University of Southern California \\
% \email{srivatsr@usc.edu}}
% \authorrunning{M. Pires, S. Ravi and R. Rodrigues}
% 

\begin{document}
%\thanks{This paper is a regular submission and is eligible for the best student paper award}
%\bibliographystyle{abbrv}

\maketitle

%
%
\begin{abstract}
%
One of the most recent members of the \emph{Paxos} family of protocols is \emph{Generalized} Paxos. 
This variant of Paxos has the characteristic that it departs from the original specification of consensus, allowing for a weaker safety condition where different processes can have a different views on a sequence being agreed upon. 
However, much like the original Paxos counterpart, Generalized Paxos does not have a simple implementation.
Furthermore, with the recent practical adoption of Byzantine fault tolerant protocols, it is timely and important to understand how Generalized Paxos can be implemented in the Byzantine model.
In this paper, we make two main contributions. First, we provide a description of Generalized Paxos that is easier to understand, based on a simpler specification and the pseudocode for a solution that can be readily implemented. Second, we extend the protocol to the Byzantine fault model.
%, and we propose a variant of the protocol that leverage the weaker specification to allow for a more efficient protocol in wide area settings.
\end{abstract}
%
%\begin{center}
%\textbf{Regular and student paper}
%\end{center}
%

% NOTE: Uncomment to decrease algorithm size
%\makeatletter
%\algrenewcommand\ALG@beginalgorithmic{\tiny}
%\makeatother
%\captionsetup[algorithm]{font=footnotesize}


% 

\chapter{Introduction}
\section{Motivation}
One of the fundamental challenges for processes participating in a distributed computation is achieving \emph{consensus}: processes initially propose a value and must \emph{eventually agree} on one of the proposed values~\cite{vukolic2012quorum}. Despite being a theoretical problem, its solution is critical to many modern large-scale system since it allows data to be replicated among distributed processes. Systems such as Google's Chubby~\cite{Burrows2006}, Spanner~\cite{Corbett2012} and Apache's ZooKeeper~\cite{Hunt2010} use techniques like \acrfull{smr}~\cite{time-clocks,Schneider1990} to allow them to remain highly available even in the presence of faults and asynchronous communication channels. \acrshort{smr} implements a fault tolerant system by modeling its processes as state machines that must receive the same inputs, execute the same state transitions and output the same results. Usually, to ensure that every state machine transitions to the same states, consensus is used to guarantee that inputs are processed in the same order. This technique is widely used to implement fault tolerant services and it's one of the reasons why consensus is so important.\par
One of the most important contributions to this field is the \acrfull{flp} impossibility result that states that the \textit{wait-free} consensus problem is unsolvable in an asynchronous system even if only one process can fail~\cite{Fischer1985}. Lamport's Paxos algorithm is able to solve consensus, circumventing the \acrshort{flp} impossibility result, by ensuring that values are always safely decided while progress is only guaranteed when the system is synchronous for a sufficient amount of time~\cite{Lamport2001}. In other words, Paxos overcomes the \acrshort{flp} result by weakening the liveness condition since it can't guarantee that correct processes will decide a value in a finite number of steps. In Paxos, there are 3 types of processes: \textit{proposers}, which propose values to be committed; \textit{acceptors}, which vote on proposed values; and \textit{learners}, that learn values voted on by a quorum of acceptors.\par
Paxos~\cite{Lamport:1998}, arguably, is one of the most popular protocols for solving the consensus problem among fault-prone processes. The evolution of the Paxos protocol represents a unique chapter in the history of Computer Science. It was first described in 1989 through a technical report~\cite{paxos:tr}, and was only published a decade later~\cite{Lamport:1998}. Another long wait took place until the protocol started to be studied in depth and used by researchers in various fields, namely the distributed algorithms~\cite{Prisco:1997} and the distributed systems~\cite{petal} research communities. And finally, another decade later, the protocol made its way to the core of the implementation of the services that are used by millions of people over the Internet, in particular since Paxos-based state machine replication is the key component of Google's Chubby lock service~\cite{Burrows2006} and Megastore storage system~\cite{36971}, or the open source ZooKeeper project~\cite{Hunt2010}, used by Yahoo!\ among others. Arguably, the complexity of the presentation may have stood in the way of a faster
adoption of the protocol, and several attempts have been made at writing more concise explanations of it~\cite{Lamport2001,Renesse2011}.\par

More recently, several variants of Paxos have been proposed and studied. Two important lines of research can be highlighted in this regard. First, a series of papers hardened the protocol against malicious adversaries by solving consensus in a Byzantine fault
model~\cite{Martin2006,Lamport2011}. The importance of this line of research is now being confirmed as these protocols are now in widespread use in the context of cryptocurrencies and distributed ledger schemes such as blockchain~\cite{bitcoin}. Second, many proposals target improving the Paxos protocol by eliminating communication costs~\cite{Lamport2006}, including an important evolution of the protocol called Generalized
Paxos~\cite{Lamport2005}, which has the noteworthy aspect of having lower communication costs by leveraging a more general specification than traditional consensus 
that can lead to a weaker requirement in terms of ordering of commands across replicas. In particular, instead of forcing all processes to agree on the same value (as with traditional consensus), it allows processes to pick an increasing sequence of commands that differs from process to process in that commutative commands may appear in a different order. The practical importance of such weaker specifications is underlined
by significant research activity on the corresponding weaker consistency models for replicated systems~\cite{Ladin:1990,dynamo}.\par

Generalized consensus is a generalization of traditional consensus that abstracts the problem of agreeing on a single value to a problem of agreeing on a monotonically increasing set of values. This problem is defined in terms of a set of values called command structures, \textit{c-structs}~\cite{Lamport2005}. These structures allow for the formulation of different consensus problems, including specifications where commutative operations are allowed to be ordered differently at different replicas (i.e., command histories). The advantage of such a problem becomes clear when considering the optimization proposed in a protocol called Fast Paxos, where fast ballots are executed by having proposers propose directly to acceptors~\cite{Lamport2006}. By avoiding sending the proposal to the leader, values can be learned in the optimal number of two message delays. However, if two proposers concurrently propose different values to acceptors, a conflict arises and at least an additional message delay is required for the leader to solve it. This is the cost that Fast Paxos pays in order to commit values in a single round trip.  Generalized consensus allows us to reduce this cost, if we define the problem as one of agreeing on command histories. Since histories are considered equivalent if non-commutative operations are totally ordered, the only operations that force the leader to intervene are non-commutative ones. An additional advantage of the generalized consensus formulation stems from its generality and from the fact that we can use the Generalized Paxos protocol to solve it despite its high level of abstraction. This protocol can be used to solve any consensus problem that can be defined in terms of generalized consensus, not only the command history problem. The reason why Generalized Paxos can take advantage of the possibility of reordering commutative commands is that it allows acceptors to accept different but compatible \textit{c-structs}. Two \textit{c-structs} are considered to be compatible if they can later be extended to equivalent \textit{c-structs}. In command histories, if all non-commutative commands are totally ordered, then two \textit{c-structs} are considered equivalent. \par
One application of the generalized consensus specification can be to implement SMR using command histories to agree on equivalent sequences of operations. For instance, consider a system with four operations $\{A, B, C, D\}$ where $C$ and $D$ are non-commutative. If two proposers concurrently propose the operations $A$ and $B$, some acceptors could accept $A$ first and then $B$ and other acceptors could accept the operations in the inverse order. However, this would not be considered a conflict and the leader wouldn't have to intervene since the operations commute. If two proposers tried to commit $C$ and $D$, acceptors could accept them in different orders which would be considered a conflict since these operations are non-commutative. In this situation, no \textit{c-struct} would be chosen and the leader would be forced to intervene by initiating a higher-numbered ballot to commit either $w \bullet C \bullet D$ or $w \bullet D \bullet C$. It's important to note that this is only one possible application of the Generalized Paxos protocol and that this protocol solves any problem that can be defined by the generalized consensus specification. \par
\iffalse{\color{red} If this is already mentioned in the discussion section, remove from intro}
The approach of allowing the system to reorder operations may feel familiar to the reader, since it resembles how weak consistency models relax consistency guarantees to allow replicas to reorder operations \cite{Ladin1992}. By relaxing consistency and allowing operations to be reordered, these models reduce coordination requirements which results in decreased latency and better operation concurrency. However, relaxing consistency guarantees also introduces the chance of state divergence which can be tolerable or not depending on the application. These approaches are critical to geo-replicated scenarios where it's important to reduce round trips between data centers and maintaining strong consistency incurs in an unacceptable latency cost. \par\fi
Despite generalized consensus' potential, it's still an understudied problem and its formulation is rather complex and abstract which makes it hard to understand and reason about. This complexity also makes the algorithm hard to implement and adapt to different scenarios. There are several symptoms of this complexity. One of them is that only the original Generalized Paxos protocol exists for this problem and few works make use of it. Another consequence of the lack of knowledge about generalized consensus is that there are several potentially interesting research questions that researchers haven't answered. For instance, despite the connection between the commutativity observation that motivated generalized consensus and the reduced coordination requirements made possible by weak consistency, it is unclear how Generalized Paxos would function in geo-replicated scenarios where the goal is to minimize cross-datacenter round trips. Similarly, there is not much research on what effects the Byzantine assumption would have on the solution of the generalized consensus problem.
\par

\section{Contributions}
The goal of this work is to perform a thorough study of the generalized consensus problem to gain a deeper knowledge about the applicable protocols, such as Generalized Paxos, and how they behave in different scenarios. One of the greatest barriers in the comprehension and adoption of Generalized Paxos is the complexity of its description which, in turn, is caused by a very generic specification of consensus. Much in the same way that the clarification of the Paxos protocol contributed to its practical adoption, it's also important to simplify the description of Generalized Paxos. \par
Furthermore, we believe it's also relevant to extend this protocol to non-crash fault models, such as the Byzantine and the Visigoth fault models, since it will open the possibility of adopting Generalized Paxos in different scenarios. In particular, the Byzantine fault model has recently gained traction in the blockchain community given the rise in popularity of cryptocurrencies like Bitcoin~\cite{bitcoin}. The Visigoth fault model targets environments like datacenters where a large number of servers are connected through a network with high security barriers, which makes it both unlikely that multiple processes will act maliciously in a coordinated way and also likely that arbitrary behavior will stem from state corruption faults due to the sheer number of components in the datacenter~\cite{Porto2015}. In the Visigoth model fault and synchrony assumptions are parameterizable in order to allow for any amount of synchrony, ranging from full asynchrony to full synchrony, and any amount of Byzantine or crash faults, allowing the system to support any combination of faults within the spectrum between crash and Byzantine faults. This allows the system administrator to parameterize the model to fit the network's characteristics which are more likely to be predictable in a datacenter. This model allows us to study command history consensus from different perspectives and propose a solution that can solve this problem across a broad spectrum of system models. This is an important aspect because, although it adds complexity, it also adds the ability to develop a generic protocol that can be used in different environments. This can be seen as removing some generality in the problem specification while retaining the original motivating scenario while, at the same time, generalizing the fault model.
\par
Concretely, this work makes the following contributions:
\begin{itemize}
	\item A simplified version of generalized consensus, which preserves its motivating scenario of agreeing on command histories;
	\item a protocol derived from the Generalized Paxos protocol that solves the aforementioned consensus problem while improving the original protocol's understandability and ease of mapping into a code implementation;
	\item an extension of the Generalized Paxos protocol to the Byzantine model;
	\item a description of the Byzantine Generalized Paxos protocol that is more accessible than the original description, namely including pseudocode;
	\item a correctness proof for the Byzantine Generalized Paxos protocol;
	\item an extension of Generalized Paxos to the Visigoth model;
	\item a description of Visigoth Generalized Paxos complete with pseudocode to ease its translation into an actual implementaton;
	\item a correctness proof for the Visigoth Generalized Paxos protocol;
	\item a discussion of several extensions and optimizations to the previous protocols.
\end{itemize}

\iffalse
Therefore, as our first contribution, we simplify the problem specification from the full generalized consensus to a simpler specification that preserves the motivating scenario. Our simpler specification is similar to the consensus problem of command histories, where the goal is to agree upon sequences of commands that, when executed, result in the same state. Two histories don't have to contain the same commands in the same order to produce the same final state, it's sufficient to ensure that non-commutative operations are totally ordered. This means that commutative operations can be differently ordered without causing divergence in the state produced by executing the command histories. \par
In order to better understand how the generalized consensus problem can be used in different environments, we adapted Generalized Paxos to several models while still preserving the advantageous properties which motivated its creation. The \acrfull{cft} protocol starts by adapting the original protocol to the previously described command history problem. The second version of the protocol, \acrfull{bgp}, solves the same problem in the Byzantine fault model. In addition to the full consensus protocol, several contributions are proposed to extend and optimize \acrshort{bgp}. Namely, a checkpointing sub-protocol is proposed to manage the accumulation of data at the acceptors in a safe way and an optimization is proposed to reduce quorum requirements for certain sequences of commands at no additional cost. Lastly, the third version of the protocol, adapts Generalized Paxos to the Visigoth fault model \cite{Porto2015}, where the fault and synchrony assumptions are parameterizable in order to allow for any amount of synchrony, ranging from full asynchrony to full synchrony, and any amount of Byzantine or crash faults, allowing the system to support any combination of faults within the spectrum between crash and Byzantine faults. With this model we can study command history consensus from different perspectives and propose a solution that allows it to be solved across a broad spectrum of system models. This is an important aspect because, although it adds complexity, it also adds the ability to develop a generic protocol that can be used in different environments. As such, this work removes some generality of the problem specification while retaining the original motivating scenario but, in turn, generalizes the model, allowing the system administrator to tune the protocol.  One particularly interesting application is similar to that of weakly consistent systems and geo-replicated datacenters. By specifying a consensus problem akin to command histories, our protocol could take advantage of operation commutativity to minimize the number of ballots that incur in a higher coordination cost. In this scenario, the parameterizability of Visigoth model also plays an important role since we can specify a number of slow processes that allows us to reduce the number of replicas that are required to safely commit values. Another potentially interesting scenario, related to the fault model, could be one where we tolerate arbitrary, uncorrelated faults. This is an interesting scenario since it's enough to deal with arbitrary state corruption faults, which are common within datacenters \cite{AmazonS32}, but avoids the cost of dealing with coordinated Byzantine behavior.\par 
\fi

\section{Document outline}
The remainder of this document is structured as follows: Chapter \ref{Related Work} is divided in five subsections and surveys works that are related to our own and may provide relevant insights into the problem we're trying to solve. Each subsection describes scientific works in a specific area of interest to us. Chapter \ref{problem} has a mainly pedagogical purpose. It describes the original generalized consensus problem as well as components of Generalized Paxos that are vital for the functioning of the protocol but whose reasoning can be opaque to the reader. Chapter \ref{Crash Fault Model} describes a simplified version of generalized consensus and proposes a protocol to implement its solution. Extensions to the protocol are also discussed along with possible scenarios in which they may be helpful. Chapter \ref{Byzatine Fault Model} adapts the simplified consensus problem to the Byzantine fault model and presents its solution, \acrlong{bgp}. Similarly, to its counterpart in the crash fault model, we discuss extensions to the protocol as well as how it differs from the most similar protocol in the literature, \acrfull{fab}~\cite{Martin2006}. Chapter \ref{vft} uses the same consensus problem defined for the Byzantine fault model but makes use of the Visigoth model to adapt Byzantine Generalized Paxos to a model with parameterizable fault and synchrony assumptions. Both Chapter \ref{Byzatine Fault Model} and Chapter \ref{vft} also present correctness proofs for their respective protocols with respect to our proposed Byzantine command history problem. Chapter \ref{conclusion} concludes this work by discussing what was learned and what unexplored avenues of research are left for future work.

% %
\section{Background and related work}
\label{sec:related} 
\subsection{Paxos and its variants} \label{Paxos} 

The Paxos protocol family solves consensus by finding an equilibrium in face of the well-known FLP impossibility result~\cite{FLP85}. It does this by always guaranteeing safety despite asynchrony, but at the same time making the observation that most of the time systems have periods during which they can be considered synchronous, since long delays are often sporadic and temporary. Therefore, Paxos only foregoes progress during the temporary periods of asynchrony, or if more than $f$ faults occur for a system of $N=2f+1$ replicas~\cite{L01}. The classic form of Paxos uses a set of proposers, acceptors and learners, runs in a sequence of ballots, and employs two phases (numbered 1 and 2), with a similar message pattern: proposer to acceptors, acceptors to proposer (and, in phase 2, also acceptors to learners). To ensure progress during synchronous periods, proposals are serialized by a distinguished proposer, which is called the leader.\par
Paxos is most commonly deployed as Multi (Decree)-Paxos, which provides an optimization of the basic message pattern by omitting the first phase of messages from all but the first ballot for each leader~\cite{Renesse2011}. This means that a leader only needs to send a \textit{phase 1a} message once and subsequent proposals may be sent directly in \textit{phase 2a} messages. This reduces the message pattern in the common case from five message delays to just three (from proposal to learning). Since there are no implications on the quorum size or guarantees provided by Paxos, the reduced latency comes at no additional cost. \par
Fast Paxos observes that it is possible to improve on the previous latency (in terms of common case message steps) by allowing proposers to propose values directly to acceptors \cite{L06}. To this end, the protocol distinguishes between fast and classic ballots, where fast ballots bypass the leader by sending proposals directly to acceptors and classic ballots work as in the original Paxos protocol. The reduced latency of fast ballots comes at the additional cost of using a quorum size of $N-e$ instead of a classic majority quorum, where $e$ is the number of faults that can be tolerated while using fast ballots. In addition, instead of the usual requirement that $N> 2f$, to ensure that fast and classic quorums intersect, a new requirement must be met: $N > 2e+f$. This means that if we wish to tolerate the same number of faults for classic and fast ballots (i.e., $e=f$), then the total number of replicas is $3f+1$ instead of the usual $2f+1$ and the quorum size for fast and classic ballots is the same. The optimized commit scenario occurs during fast ballots, in which only two messages broadcasts are necessary: \textit{phase 2a} messages between a proposer and the acceptors, and \textit{phase 2b} messages between acceptors and learners. This creates the possibility of two proposers concurrently proposing values to the acceptors and generating a conflict, which must be resolved by falling back to a recovery protocol. \par
Generalized Paxos improves the performance of Fast Paxos by addressing the issue of collisions. More precisely, it allows acceptors to accept different sequences of commands as long as non-commutative operations are totally ordered \cite{Lamport2005}. In the original description, non-commutativity between operations is generically represented as an interference relation. In this context, Generalized Paxos abstracts the traditional consensus problem of agreeing on a single value to the problem of agreeing on an increasing set of values. \textit{C-structs} provide this increasing sequence abstraction and allow the definition of different consensus problems. If we define the sequence of learned commands of a learner $l_i$ as a \textit{c-struct} $learned_{l_i}$, then the consistency requirement for generalized consensus can be defined as: $learned_{l_1}$ and $learned_{l_2}$ must have a \textit{common upper bound}, for all learners $l_1$ and $l_2$. This means that, for any $learned_{l_1}$ and $learned_{l_2}$, there must exist some \textit{c-struct} of which they are both prefixes. This prohibits interfering commands from being concurrently accepted because no subsequent \textit{c-struct} would extend them both. 
Defining \textit{c-structs} as command histories enables acceptors to agree on different sequences of commands and still preserve consistency as long as dependence relationships are not violated. This means that commutative commands can be ordered differently regarding each other but interfering commands must preserve the same order across each sequence at any learner. This guarantees that solving the consensus problem for histories is enough to implement a state-machine replicated system. \par
Mencius is a variant of Paxos that tries to address the wide area latency penalty caused by having a single leader, through which every proposal must go through. In Mencius, the leader of each round rotates between every process: the leader of round $i$ is process $p_k$, such that $k = n\ mod\ i$.  Leaders with nothing to propose can skip their turn by proposing a \textit{no-op}. If a leader is slow or faulty, the other replicas can execute \textit{phase 1} to revoke the leader's right to propose a value, but they can only propose a \textit{no-op} instead \cite{Mao2008}. Considering that non-leader replicas can only propose \textit{no-ops}, a \textit{no-op} command from the leader can be accepted in a single message delay since there is no chance of another value being accepted. If some non-leader server revokes the leader's right to propose and suggests a \textit{no-op}, then the leader can still suggest a value $v \neq$ \textit{no-op}, which will eventually be accepted as long as $l$ is not permanently suspected. Mencius also takes advantage of commutativity by allowing out-of-order commits, where values $x$ and $y$ can be learned in different orders by different learners if there does not exist a dependence relationship between them.

Egalitarian Paxos (EPaxos) extends the goal of Mencius of achieving a better throughput than Paxos by removing the bottleneck caused by having a leader \cite{Moraru2013}. To avoid choosing a leader, the proposal of commands for a command slot is done in a decentralized manner, taking advantage of the commutativity observations made by Generalized Paxos \cite{Lamport2005}. If two replicas unknowingly propose commands concurrently, one will commit its proposal in one round trip after getting replies from a quorum of replicas. However, some replica will see that another command was concurrently proposed and may interfere with the already committed command. If the commands are non-commutative then the replica must reply with a dependency between the commands, committing its command in two rounds trips. This commit latency is achieved by using a \textit{fast-path quorum} of $f+\lfloor\frac{f+1}{2}\rfloor$ replicas. Similarly to Mencius, EPaxos achieves a substantially higher throughput than Multi-Paxos.

\subsection{Byzantine fault tolerant replication} \label{Non-Crash}
%Non-crash fault models emerged to cope with the effect of malicious attacks and software errors. These models (e.g., the arbitrary fault model) assume a stronger adversary than previous crash fault models. 
The Byzantine Generals Problem is defined as a set of Byzantine generals that are camped in the outskirts of an enemy city and have to coordinate an attack. Each general can either decide to attack or retreat and there may be $f$ traitors among the generals that try to prevent the loyal generals from agreeing on the same action. The problem is solved if every loyal general agrees on what action to take \cite{LSP82}. Like the traitorous generals, a process that suffers a Byzantine fault may display an arbitrary behaviour and, in case of multiple Byzantine faults, an adversary may even coordinate multiple faulty replicas in an attack. \par
PBFT is a protocol that solves consensus for state machine replication while tolerating up to $f$ Byzantine faults \cite{CL99}. The system moves through configurations called \textit{views} in which one replica is the primary and the remaining replicas are the backups. The safety property of the algorithm requires that operations be totally ordered. The protocol starts when a client sends a request for an operation to the primary, which in turn assigns a sequence number to the request and multicasts a \textit{pre-prepare} message to the backups. If a backup replica accepts the pre-prepare message, it multicasts a \textit{prepare} message and adds both messages to its log. Both of these phases are needed to ensure that the requested operation is totally ordered at every correct replica, therefore satisfying the protocol's safety property. After receiving $2f$ prepare messages, a replica multicasts a \textit{commit} message and commits the message to its log when it has received $2f$ commit messages from other replicas. The liveness property requires that clients must eventually receive replies to their requests, provided that there are at most $\lfloor\frac{n-1}{3}\rfloor$ faults and the transmission time does not increase continuously. Backups can trigger new views after increasingly long timeouts if they suspect the leader to be Byzantine. \par
The closest related work is Fast Byzantine Paxos (FaB), which solves consensus in the Byzantine setting within two message communication steps in the common case, while requiring $5f+1$ acceptors to ensure safety and liveness \cite{Martin2006}. A variant that is proposed in the same paper is the Parameterized FaB Paxos protocol, which generalizes FaB by decoupling replication for fault tolerance from replication for performance. As such, the Parameterized FaB Paxos requires $3f+2t+1$ replicas to solve consensus, preserving safety while tolerating up to $f$ faults and completing in two steps despite up to $t$ faults. Therefore, FaB Paxos is a special case of Parameterized FaB Paxos where $t=f$. It has also been shown that $N>5f$ is a lower bound on the number of acceptors required to guarantee 2-step execution in the Byzantine model. In this sense, the FaB protocol is tight since it requires $5f+1$ acceptors to guarantee 2-step execution while preserving both safety and liveness. \par
In comparison to FaB Paxos, our protocol, Byzantine Generalized Paxos (BGP), requires a lower number of acceptors than what is stipulated by FaB's lower bound. However, this does not constitute a violation of the result since BGP does not guarantee a 2-step execution in the Byzantine scenario. Instead, BGP only provides a two communication step latency when proposed sequences are commutative with any other concurrently proposed sequence. In other words, Byzantine Generalized Paxos leverages a weaker performance guarantee to decrease the requirements regarding the minimum number of processes. In particular, a proposed sequence may not gather enough votes to be learned in the ballot in which it is proposed, either due to Byzantine behaviour or contention between non-commutative commands. However, any sequence is guaranteed to eventually be learned in a way such that non-commutative commands are totally ordered at any correct learner.
\iffalse One potentially dangerous scenario occurs when the leader is faulty. In this case, the leader could split the acceptors votes or run a fast ballot in some and a classic ballot in others. In this scenario, since learners require $N-f$ votes for any given sequence and non-commutative sequences aren't considered equivalent, at most one sequence can be learned by correct learners. This also applies to the case where two non-commutative sequences are proposed in a fast ballot. It's possible for the votes to be split in a way such that no sequence is learned. In this case, for the remainder of the ballot, since every possible subsequent sequence at each acceptor will be an extension of the previous non-commutative sequences, no new sequences will be learned. This prevents safety from being violated but also precludes liveness until a new classic ballot is initiated by a correct leader. Once a correct leader gathers the previous votes from the acceptors, it will propose a serialization of previous sequences that solves the conflict caused by the non-commutative sequences.
\fi


% %
\section{Model}
\label{sec:model}
%
We consider an \emph{asynchronous} system in which
a set of $n \in \mathbb{N}$ processes communicate by 
\emph{sending} and \emph{receiving} messages.
Each process executes an algorithm assigned to it, but may stop executing it by \emph{crashing}.
If a process does not follow the algorithm assigned to it during an execution, then it is \emph{Byzantine}; otherwise, we say that process is \emph{correct}.
This paper considers the \emph{authenticated} Byzantine model: every process can produce cryptographic digital signatures~\cite{quorum}. 
Furthermore, for clarity of exposition, we assume authenticated perfect links~\cite{cgr:book}, 
where a message that is sent by a non-faulty sender is eventually received and messages cannot be forged 
(such links can be implemented trivially using retransmission, elimination of duplicates, and point-to-point message authentication codes~\cite{cgr:book}.)
A process may be a \emph{learner}, \emph{proposer} or \emph{acceptor}.
Informally, proposers provide input values that must be agreed upon by learners, the acceptors help the learners \emph{agree} on a value, and learners learn commands by appending them to a local sequence of commands to be executed, $learned_l$ .
Our protocols require a minimum number of acceptor processes ($N$) that is a function of the maximum number of tolerated Byzantine faults ($f$), namely $N \ge 3f+1$. We assume that acceptor processes have identifiers in the set $\{0,...,N-1\}$. In contrast, the number of proposer and learner processes can be set arbitrarily.\par
\noindent\textbf{Problem Statement.}
In our simplified specification of Generalized Paxos, each learner $l$ maintains a monotonically increasing sequence of commands $learned_l$. 
We define these learned sequences of commands to be equivalent ($\thicksim$) 
if one can be transformed into the other by permuting the elements in a way such that the order of non-commutative pairs is preserved. A sequence $x$ is defined to be an \textit{eq-prefix} of another sequence $y$ ($x \sqsubseteq y$), if the subsequence of $y$ that contains all the elements in $x$ is equivalent ($\thicksim$) to $x$. 
We present the requirements for this consensus problem, stated in terms of learned sequences of commands for a learner $l$, $learned_l$. 
To simplify the original specification, instead of using c-structs (as explained in Section~\ref{sec:related}), we specialize to agreeing on equivalent sequences of commands:\par
%
\begin{enumerate}
\item \textbf{Nontriviality.} If all proposers are correct, $learned_l$ can only contain proposed commands.
\item \textbf{Stability.} If $learned_l = s$ then, at all later times, $s \sqsubseteq learned_l$, for any sequence $s$ and correct learner $l$.
\item \textbf{Consistency.} At any time and for any two correct learners $l_i$ and $l_j$, $learned_{l_i}$ and $learned_{l_j}$.
can subsequently be extended to equivalent sequences.
\item \textbf{Liveness.} For any proposal $s$ from a correct proposer, and correct learner $l$, eventually $learned_l$ contains $s$.
\end{enumerate}

%
\section{Protocol}
%
\begin{definition}
(Multi-valued wait-free consensus with adversary $\mathcal{A}$)

A process is \emph{correct} with respect to crash adversary 
in an execution $E$ if it takes infinitely many steps in $E$.
%
\begin{itemize}
\item (Agreement): No two processes agree on different values  
\item (Liveness): Every correct process (w.r.t $\mathcal{A}$) 
eventually decides on a value previously proposed by a correct process (w.r.t $\mathcal{A}$).
\end{itemize}
%
\end{definition}
%
\begin{algorithm}
\caption{Generalized Paxos - Proposer p}
\textbf{Local variables:} $ballot_l = 0,\ maxTried_l = \bot,\ C_l = \bot,\ timer = \bot,\ quorumSize = 0,\ messages = \bot,\ previousMessages = \bot,\ collected = False,\ verification = False, \ currentPhase = \bot$
\begin{algorithmic}[1]

    \Function{Propose}{\textit{C}}
    \If{fast\_ballot}
        \State \textbf{run} \Call{phase\_1a}{fast,\textit{C}};
    \Else
        \State \textbf{run} \Call{send}{\textit{propose, C}} to Leader;
    \EndIf
    \EndFunction
        
    \State
    \State \textbf{upon} \textit{receive(propose, C)} from proposer $p_i$ \textbf{do} 
        \State \hspace{\algorithmicindent} \textbf{if} $p = Leader$ \textbf{then}
            \State \hspace{\algorithmicindent}\hspace{\algorithmicindent} \textbf{run} \Call{phase\_1a}{slow, C};
    
    \State
    \State \textbf{upon} \textit{receive(statement)} from proposer $p_i$ \textbf{do}
    \State \hspace{\algorithmicindent} $messages[ballot_l][p_i] = statement$;
    
    \State
    \State \textbf{upon} $timer$ \textbf{do} 
    \State \hspace{\algorithmicindent} $quorumSize = n-u$;

    \State     
    \State \textbf{upon} $\#(messages) \geq quorumSize \land collected[ballot_l][round] = False$ \textbf{do} 
        \State \hspace{\algorithmicindent} \textbf{if }{$\#(messages) < n-s \land verification = False$} \textbf{then}
            \State \hspace{\algorithmicindent}\hspace{\algorithmicindent} $quorumSize = n-u$;
            \State \hspace{\algorithmicindent}\hspace{\algorithmicindent}
            $previousMessages = messages$;
            \State \hspace{\algorithmicindent}\hspace{\algorithmicindent}
            $messages = \bot$;
            \State \hspace{\algorithmicindent}\hspace{\algorithmicindent} $verification = True$;
            \State \hspace{\algorithmicindent}\hspace{\algorithmicindent} $timer = \textbf{run}\ \Call{start\_timer}{2T}$;
            \State \hspace{\algorithmicindent}\hspace{\algorithmicindent} \textbf{if} $fast\_ballot$ \textbf{then}
            \State \hspace{\algorithmicindent}\hspace{\algorithmicindent}\hspace{\algorithmicindent} \textbf{run} \Call{phase\_1a}{$fast,C$};
            \State\hspace{\algorithmicindent}\hspace{\algorithmicindent} \textbf{else}
            \State\hspace{\algorithmicindent}\hspace{\algorithmicindent}\hspace{\algorithmicindent} \textbf{run} \Call{Phase\_1a}{slow, C};
            
            
        \State \hspace{\algorithmicindent} \textbf{else}
            \State \hspace{\algorithmicindent}\hspace{\algorithmicindent} $collected[ballot_l][currentPhase] = True$;
            \State \hspace{\algorithmicindent}\hspace{\algorithmicindent} \textbf{if}\ $currentPhase = p1b$ \textbf{then}
            \State \hspace{\algorithmicindent}\hspace{\algorithmicindent}\hspace{\algorithmicindent} 
            \textbf{run} \Call{phase\_2a}{$ballot_l, Q$};
            \State \hspace{\algorithmicindent}\hspace{\algorithmicindent} \textbf{else}
            \State \hspace{\algorithmicindent}\hspace{\algorithmicindent}\hspace{\algorithmicindent} \textbf{run} \Call{send}{$p2b, message.bal, message.val$} to learners;
            
    \State
    \State \textbf{upon} $verification = True \land collected[ballot_l][currentPhase] = False \land (\#(messages) \geq quorumSize \lor\ \#(quorum(messages)\ \cup\ quorum(previousMessages)) \geq n-s)$ \textbf{do}
    \State \hspace{\algorithmicindent} $Q = quorum(messages)\ \cup\ quorum(previousMessages)$;
    \State \hspace{\algorithmicindent} $collected[ballot_l][lastPhase] = True$;
    \State\ \hspace{\algorithmicindent}\textbf{if}\ $currentPhase = p1b$ \textbf{then}
            \State \hspace{\algorithmicindent}\hspace{\algorithmicindent} 
            \textbf{run} \Call{phase\_2a}{$ballot_l, Q$};
            \State \hspace{\algorithmicindent} \textbf{else}
            \State \hspace{\algorithmicindent}\hspace{\algorithmicindent} \textbf{run} \Call{send}{$p2b, message.bal, message.val$} to learners;

\end{algorithmic}
\end{algorithm}

\begin{algorithm}
\caption{Generalized Paxos - Proposer p (continued)}
\begin{algorithmic}[1]
    %\item[] % unnumbered empty line
    \Function{Phase\_1a}{ballot\_type, C}
        \State $C_l = C$;
        \State $quorumSize = n-s$;
        \State $currentPhase = p1a$;
        \State $timer$ = \textbf{run} \Call{start\_Timer}{2T};
        
        \State
        \If{ballot\_type = fast}
            \State \textbf{run} \Call{send}{$fast, ballot_l$} to Acceptors;
        \Else
            \State \textbf{run} \Call{send}{$p1a, ballot_l$} to Acceptors;
        \EndIf
    \EndFunction
    
    \State
    \Function{Phase\_2a}{$bal, Q$}
        \State $maxTried_l$ = \textbf{run} \Call{Proved\_Safe}{$Q, bal$};
        \State $maxTried_l = maxTried_l \bullet C_l$;
        \State $currentPhase = p2a$;
        \State \textbf{run} \Call{send}{$p2a,ballot_l, maxTried_l$} to Acceptors;
    \EndFunction
    
    \State
    \Function{Proved\_Safe}{Q, m}
        \State $k = max(i\ |\ (i < m) \wedge (\exists a \in Q :\ val_a[i]\ \neq null))$;
        \State $RS = \{R \in k$-$quorum\ |\ \forall a \in R \cap Q : val_a[k] \neq null\}$;
        
        \State
        \If{$RS = \varnothing$}
            \State \textbf{return} $\{val_a[k]\ |\ (a \in Q) \wedge (val_a[k] \neq null)\}$;
        \Else
            \State \textbf{return} \textbf{run} $lowerUpperBound(val_a[k]\ |\ a \in Q \cap R$);
        \EndIf
    \EndFunction
        
\end{algorithmic}
\end{algorithm}

\begin{algorithm}
\caption{Generalized Paxos - Acceptor a}
\textbf{Local variables: } $bal_a = 0,\ mbal_a = 0, \val_a = \bot$ 
\begin{algorithmic}[1]
  
  \State \textbf{upon} \textit{receive(fast, val)} from proposer \textit{p} \textbf{do}
    \State \hspace{\algorithmicindent} \textbf{run} \Call{phase\_2b\_fast}{$val, p$};
    
    \State
    \State \textbf{upon} \textit{receive(p1a, ballot)} from leader \textit{l} \textbf{do}
    \State \hspace{\algorithmicindent} \textbf{run} \Call{phase\_1b}{$ballot, l$};
    
    \State
    \State \textbf{upon} \textit{receive(p2a, ballot, value)} from leader \textit{l} \textbf{do}
    \State \hspace{\algorithmicindent} \textbf{run} \Call{phase\_2b\_classic}{$ballot, value, l$};
    
    \State
    \Function{Phase\_1b}{$m, l$}
        \If {$bal_a < m$}
            \State $bal_a = m$;
            \State \Call{send}{$p1b, m, mbal_a, val_a$} to leader l;
        \EndIf
    \EndFunction
    
    \State
    \Function{Phase\_2b\_Classic}{$m, v, l$}
        \State $k = max(i\ |\ (i < m) \wedge (\exists a \in Q :\ val_a[i]\ \neq null))$;
        \If {$m \geq k$}
            \State $val_a = v$;
            \State \Call{send}{$p2b, bal_a, val_a$} to leader l;
        \EndIf
    \EndFunction
    
    \State
    \Function{Phase\_2b\_Fast}{$v, p$}
        \State $k = max(i\ |\ (i < m) \wedge (\exists a \in Q :\ val_a[i]\ \neq null))$;
        \If {$bal_a == k$}
            \State $val_a = val_a \bullet v$;
            \State \Call{send}{$p2b, bal_a, val_a$} to process p;
        \EndIf
    \EndFunction
    
\end{algorithmic}
\end{algorithm}

\begin{algorithm}
\caption{Generalized Paxos - Learner l}
\textbf{Local variables: } $learned = \varnothing$ 
\begin{algorithmic}[1]
  
    \State \textbf{upon} $receive (p2b, bal, val)$ from proposer p \textbf{do}
        \State \hspace{\algorithmicindent} $learned = learned \cup val$;
\end{algorithmic}
\end{algorithm}
%
%

\vspace{1em}

\begin{normalsize}
\noindent {\bf Acknowledgements.} This work was supported by the European Research Council (ERC-2012-StG-307732) and FCT (UID/CEC/50021/2013).
\end{normalsize}

\bibliographystyle{plain}
\bibliography{references}


%\appendix
%\clearpage
%\begin{algorithm}[!t] 
	\caption{Byzantine Generalized Paxos - Proposer p}
	\label{BFT-Prop}
	\textbf{Local variables:} $ballot\_type = \bot$
	\begin{algorithmic}[1]
			
		\State \textbf{upon} \textit{receive(BALLOT, type)} \textbf{do} 
		\State \hspace{\algorithmicindent} $ballot\_type = type$;
		\State
		
		\State \textbf{upon} \textit{command\_request(c)} \textbf{do}   \hspace{\algorithmicindent}\hspace{\algorithmicindent}\hspace{\algorithmicindent}\hspace{\algorithmicindent}\# receive request from application
		\State \hspace{\algorithmicindent} \textbf{if} $ballot\_type = fast\_ballot$ \textbf{then}
		\State \hspace{\algorithmicindent}\hspace{\algorithmicindent} \Call{send}{$P2A\_FAST, c$} to acceptors;
		\State \hspace{\algorithmicindent} \textbf{else} 
		\State \hspace{\algorithmicindent}\hspace{\algorithmicindent} \Call{send}{\textit{PROPOSE, c}} to leader;		
	\end{algorithmic}
\end{algorithm}

\begin{algorithm} 
	\caption{Byzantine Generalized Paxos - Process p}
	\label{BFT-Proc}
	\textbf{Local variables:} $suspicions = \bot,\ new\_view = \bot,\ leader = \bot,\ view = 0$
	\begin{algorithmic}[1]
		
		\State \textbf{upon} \textit{suspect\_leader} \textbf{do} 
		\State \hspace{\algorithmicindent} \textbf{if} $suspicions[p] \neq true$ \textbf{then}
		\State\hspace{\algorithmicindent}\hspace{\algorithmicindent} $suspicions[p] = true$;
		\State\hspace{\algorithmicindent}\hspace{\algorithmicindent} $proof = \langle suspicion, view \rangle_{priv_p}$;
		\State\hspace{\algorithmicindent}\hspace{\algorithmicindent} \Call{send}{$SUSPICION, view,proof$};	
		\State
		
		\State \textbf{upon} \textit{receive($SUSPICION, view_i, proof$)} from process $p_i$ \textbf{do} 
		\State \hspace{\algorithmicindent} \textbf{if} $view_i \neq view$ \textbf{then}
		\State \hspace{\algorithmicindent}\hspace{\algorithmicindent} \textbf{return};
		\State
		\State \hspace{\algorithmicindent} $suspicions[p_i] = proof$;
		\State \hspace{\algorithmicindent} \textbf{if} $\#(suspicions) > f$ and $new\_view[p] = \bot$ \textbf{then}
		\State\hspace{\algorithmicindent}\hspace{\algorithmicindent} $new\_view[p] = \langle last\_bal,\ last\_val \rangle$;
		\State\hspace{\algorithmicindent}\hspace{\algorithmicindent} \Call{send}{$VIEW\_CHANGE, view+1, last\_bal, last\_val, suspicions$};
		\State
		
		\State\textbf{upon} \textit{receive($VIEW\_CHANGE, view_i, last\_bal_i, last\_val_i, suspicions$)} 
		\item[] from process $p_i$ \textbf{do} 
		\State \hspace{\algorithmicindent} \textbf{if} $view_i \leq view$ \textbf{then}
		\State \hspace{\algorithmicindent}\hspace{\algorithmicindent}\textbf{return};
		\State
		\State \hspace{\algorithmicindent} \textbf{for} $p$ \textbf{in} $processes$ \textbf{do} 
		\State \hspace{\algorithmicindent}\hspace{\algorithmicindent} $proof = suspicions[p]$;
		\State \hspace{\algorithmicindent}\hspace{\algorithmicindent} \textbf{if} $proof \neq \langle suspicion, view_i \rangle_{pub_{p_i}}$ \textbf{then}
		\State \hspace{\algorithmicindent}\hspace{\algorithmicindent}\hspace{\algorithmicindent} \textbf{return};
		
		\State
		\State\hspace{\algorithmicindent} $new\_view[view_i][p_i] = \langle last\_bal_i,\ last\_val_i \rangle$;
		\State\hspace{\algorithmicindent} \textbf{if} $new\_view[view_i][p] = \bot$ \textbf{then}
		\State\hspace{\algorithmicindent}\hspace{\algorithmicindent} $new\_view[view_i][p] = \langle last\_bal,\ last\_val \rangle$;
		\State\hspace{\algorithmicindent}\hspace{\algorithmicindent}  \Call{send}{$VIEW\_CHANGE, view_i, last\_bal, last\_val, suspicions_{\sigma_i}$};
		\State
		\State\hspace{\algorithmicindent} \textbf{if} $\#(new\_view[view_i]) > N-f$ \textbf{then}
		\State\hspace{\algorithmicindent}\hspace{\algorithmicindent} $view = view_i$;
		\State\hspace{\algorithmicindent}\hspace{\algorithmicindent} $leader = n\ mod\ view$;
		\State\hspace{\algorithmicindent}\hspace{\algorithmicindent} $suspicions = \bot$;
		\State\hspace{\algorithmicindent}\hspace{\algorithmicindent} $new\_view = \bot$;

		\State
		\Function{merge\_sequences}{$old\_seq, new\_seq$}
		\State \textbf{for} $c$ \textbf{in} $new\_seq$ \textbf{do} 
		\State \hspace{\algorithmicindent} \textbf{if} $!\Call{contains}{old\_seq,c}$ \textbf{then}
		\State \hspace{\algorithmicindent}\hspace{\algorithmicindent}\hspace{\algorithmicindent} $old\_seq =  old\_seq \bullet c$;
		\State \textbf{end for}
		\State \textbf{return} $old\_seq$;
		\EndFunction

	\end{algorithmic}
\end{algorithm}

\begin{algorithm} 
	\caption{Byzantine Generalized Paxos - Leader l}
	\label{BFT-Lead}
	\textbf{Local variables:} $ballot_l = 0,\ maxTried_l = \bot,\ proposals = \bot,\ accepted = \bot$
	\begin{algorithmic}[1]
		\State \textbf{upon} \textit{trigger\_next\_ballot(type)} \textbf{do}
		\State \hspace{\algorithmicindent} $ballot_l \mathrel{+{=}} 1$;
		\State \hspace{\algorithmicindent} \Call{send}{$BALLOT,type}$ to proposers;
		\State
		\State \hspace{\algorithmicindent} \textbf{if} $type = fast$ \textbf{then}
		\State \hspace{\algorithmicindent}\hspace{\algorithmicindent} \Call{send}{$FAST,ballot_l}$ to acceptors;
		\State \hspace{\algorithmicindent} \textbf{else}
		\State \hspace{\algorithmicindent}\hspace{\algorithmicindent} \Call{send}{$P1A, ballot_l$} to acceptors;
		
		\State
		\State \textbf{upon} \textit{receive(PROPOSE, prop)} from proposer $p_i$ \textbf{do} 
		\State \hspace{\algorithmicindent} $proposals = proposals \bullet prop$;
		\State
		\State \textbf{upon} \textit{receive($P1B, m, bal_a,val_a$)} from acceptor $a$ \textbf{do}
		\State \hspace{\algorithmicindent} \textbf{if} $m = ballot_l$ \textbf{then}
		\State \hspace{\algorithmicindent}\hspace{\algorithmicindent} $accepted[ballot_l][a] = val_a$;
		\State
		\State \hspace{\algorithmicindent}\hspace{\algorithmicindent} \textbf{if} $\#(accepted[ballot_l]) \geq N-f$ \textbf{then} 
		\State \hspace{\algorithmicindent}\hspace{\algorithmicindent}\hspace{\algorithmicindent} \Call{phase\_2a}{$ $};

		\State
		\Function{phase\_2a}{$ $}
		\State $maxTried_l = \Call{proved\_safe}{ballot_l}$;
		\State $maxTried_l = maxTried_l \bullet proposals$;
		\State \Call{send}{$P2A,ballot_l, maxTried_l$} to acceptors;
		\EndFunction
		
		\State
		\Function{proved\_safe}{$ballot$}
		\State $safe\_seq = \bot$;
		\State \textbf{for} $seq$ \textbf{in} $accepted[ballot]$ \textbf{do}
		\State \hspace{\algorithmicindent} $safe\_seq = \Call{merge\_sequences}{safe\_seq, seq}$;
		\State \textbf{end for}
		\State \textbf{return} $safe\_seq$;
		\EndFunction
	\end{algorithmic}
\end{algorithm}

\begin{algorithm} 
	\caption{Byzantine Generalized Paxos - Acceptor a}
	\label{BFT-Acc}
	\textbf{Local variables:} $bal_a = 0,\ val_a = \bot,\ fast\_bal = \bot$ 
	\begin{algorithmic}[1]
		\State \textbf{upon} \textit{receive(P1A, ballot)} from leader \textit{l} \textbf{do}
		\State \hspace{\algorithmicindent} \Call{phase\_1b}{$ballot, l$};
		
		\State
		\State \textbf{upon} \textit{receive(FAST,ballot)} from leader \textit{l} \textbf{do}
		\State \hspace{\algorithmicindent} $fast\_bal[ballot] = true$;
		
		\State
		\State \textbf{upon} \textit{receive(P2A\_CLASSIC, ballot, value)} from leader \textit{l} \textbf{do}
		\State \hspace{\algorithmicindent} \Call{phase\_2b\_classic}{$ballot, value$}; 

		\State		
		\State \textbf{upon} \textit{receive(P2A\_FAST, value)} from proposer \textit{p} \textbf{do}
		\State \hspace{\algorithmicindent} \Call{phase\_2b\_fast}{$value$};
				
		\State
		\Function{phase\_1b}{$ballot, l$}
		\If {$bal_a < ballot$}
		\State \Call{send}{$P1B, ballot, bal_a, val_a$} to leader l;
		\State $bal_a = ballot$;	
		\State $val_a[bal_a] = \bot$;	
		\EndIf
		\EndFunction
	
		\State
		\Function{phase\_2b\_classic}{$ballot, value$}
		\If {$ballot \geq bal_a$ and $val_a = \bot$}
		\State $bal_a = ballot$;
		\State $val_a[ballot] = value$;
		\State \Call{send}{$P2B, ballot, value$} to acceptors;
		\EndIf
		\EndFunction
		
		\State
		\Function{phase\_2b\_fast}{$value$}
		\If {$fast\_bal[bal_a]$}
		\State $val_a[bal_a] =  \Call{merge\_sequences}{val_a[bal_a], value}$;
		\State \Call{send}{$P2B, bal_a, value$} to learners;
		\EndIf
		\EndFunction
	\end{algorithmic}
\end{algorithm}


\begin{algorithm}
	\caption{Byzantine Generalized Paxos - Learner l}
	\label{BFT-Learn}
	\textbf{Local variables:} $learned = \bot,\ messages = \bot$ 
	\begin{algorithmic}[1]
		\State \textbf{upon} \textit{receive($P2B, ballot, value$)} from acceptor $a$ \textbf{do}
		\State \hspace{\algorithmicindent} $messages[ballot][value][a] = true$;
		
		\State \hspace{\algorithmicindent} \textbf{if} $\#(messages[ballot][value]) \geq N-f$ \textbf{then}
		\State \hspace{\algorithmicindent}\hspace{\algorithmicindent}\hspace{\algorithmicindent}\hspace{\algorithmicindent}
		$learned = \Call{merge\_sequences}{learned, value}$;
	\end{algorithmic}
\end{algorithm}
%\clearpage
%\section{Correctness Proofs} \label{proof}

This section argues for the correctness of the Byzantine Generalized Paxos protocol in terms of the specified consensus properties.\par

\begin{table}[h!]
	\renewcommand{\arraystretch}{1.5}
	\centering
	\begin{tabularx}{\linewidth}{ |c|X|}
		%\hline
		%\multicolumn{2}{|c|}{Notation}\\
		\hline
		Invariant/Symbol & Definition \\
		\hline
		$\thicksim$ & Equivalence relation between sequences \\
		\hline
		$X \overset{e}{\implies} Y$ & $X$ implies that $Y$ is eventually true \\
		\hline
		$X \sqsubseteq Y$ & The sequence $X$ is a prefix of sequence $Y$ \\
		\hline
		$\mathcal{L}$ & Set of learner processes \\
		\hline
		$\mathcal{P}$ & Set of proposals (commands or sequences of commands) \\
		\hline
		$learned_{l_i}$ & Learner $l_i$'s $learned$ sequence of commands \\
		\hline
		$learned(l_i,s)$ & $learned_{l_i}$ contains the sequence $s$ \\
		\hline
		$maj\_accepted(s)$ & $N-f$ acceptors sent phase 2b messages to the learners for sequence $s$ \\
		\hline
		$min\_accepted(s)$ & $f+1$ acceptors sent phase 2b messages to the learners for sequence $s$ \\
		\hline
		$proposed(s)$ & A proposer proposed $s$ \\
		\hline
		
	\end{tabularx} 
	\vspace{\smallskipamount}
	\caption{Proof notation} 
	\label{table:1}
\end{table}

\subsubsection{Consistency}
\begin{theorem}At any time and for any two correct learners $l_i$ and $l_j$, $learned_{l_i}$ and $learned_{l_j}$ can subsequently be extended to equivalent sequences \par
\end{theorem} 
\textbf{Proof:} \par
\parbox{\linewidth}{\strut1. At any given instant, $\forall s,s' \in \mathcal{P}, \forall l_i,l_j \in \mathcal{L}, learned(l_j,s) \land learned(l_i,s') \implies \exists \sigma_1,\sigma_2 \in \mathcal{P}, s \thicksim s' \bullet \sigma_1 \lor s' \thicksim s \bullet \sigma_2$}  \par
\indent\indent\parbox{\linewidth}{\strut\textbf{Proof:} }\par
\indent\indent\indent\parbox{\linewidth-\algorithmicindent*3}{\strut1.1. At any given instant, $\forall s,s' \in \mathcal{P}, \forall l_i,l_j \in \mathcal{L}, learned(l_i,s) \land learned(l_j,s') \implies (maj\_accepted(s) \lor (min\_accepted(s) \land (s \thicksim x \bullet \sigma_1 \lor x \thicksim s \bullet \sigma_2))) \land (maj\_accepted(s') \lor (min\_accepted(s') \land (s' \thicksim x \bullet \sigma_1 \lor x \thicksim s' \bullet \sigma_2))), \exists \sigma_1, \sigma_2 \in \mathcal{P}, \forall x \in \mathcal{P}$} \par
\indent\indent\indent\indent\parbox{\linewidth-\algorithmicindent*4}{\strut\textbf{Proof:} A sequence can only be learned if the learner gathers $N-f$ votes (i.e., $maj\_accepted(s)$) or if it is universally commutative (i.e., $s \thicksim x \bullet \sigma_1 \lor x \thicksim s \bullet \sigma_2,\ \exists \sigma_1, \sigma_2 \in \mathcal{P}, \forall x \in \mathcal{P}$) and the learner gathers $f+1$ votes (i.e., $min\_accepted(s)$). The first case includes both gathering $N-f$ votes directly from each acceptor (Algorithm \ref{BFT-Learn} lines \{1-4\}) and gathering $N-f$ proofs of vote from only one acceptor, as is the case when the sequence contains a special checkpointing command (Algorithm \ref{BFT-Learn} \{6-11\}). The second case requires that the sequence must be commutative with any other (Algorithm \ref{BFT-Learn} \{1-4\}). This is encoded in the logical expression $s \thicksim x \bullet \sigma_1 \lor x \thicksim \bullet \sigma_2$ which is true if either the learned sequence $s$ can be extended with a suffix $\sigma_1$ to any other sequence $x$ or if any sequence $x$ can be extended with some prefix $\sigma_2$ to be equivalent to $s$.}
\indent\indent\indent\parbox{\linewidth-\algorithmicindent*3}{\strut1.2. At any given instant, $\forall s,s' \in \mathcal{P}, maj\_accepted(s) \land maj\_accepted(s') \implies \exists \sigma_1,\sigma_2 \in \mathcal{P}, s \thicksim s' \bullet \sigma_1 \lor s' \thicksim s \bullet \sigma_2$}\par
\indent\indent\indent\indent\parbox{\linewidth}{\strut\textbf{Proof:} Proved by contradiction.}\par
\indent\indent\indent\indent\indent\parbox{\linewidth-\algorithmicindent*5}{\strut1.2.1.~At any given instant, $\exists s,s' \in \mathcal{P}, \forall \sigma_1,\sigma_2 \in \mathcal{P}, maj\_accepted(s) \land maj\_accepted(s') \wedge s \not\thicksim s' \bullet \sigma_1 \land s' \not\thicksim s \bullet \sigma_2$} \par
\indent\indent\indent\indent\indent\indent\parbox{\linewidth}{\strut\textbf{Proof:} Contradiction assumption.}\par
\indent\indent\indent\indent\indent\parbox{\linewidth-\algorithmicindent*5}{\strut1.2.2. Take $s$ and $s'$ which are certain to exist by 1.2.1, $s$ and $s'$ are non-commutative }\par
\indent\indent\indent\indent\indent\indent\parbox{\linewidth-\algorithmicindent*6}{\strut\textbf{Proof:} If $\forall \sigma_1,\sigma_2 \in \mathcal{P}, s \not\thicksim s' \bullet \sigma_1 \land s' \not\thicksim s \bullet \sigma_2$ then $s$ and $s'$ must contain non-commutative commands differently ordered. Otherwise, they would be possible to extend to equivalent sequences.}\par
\indent\indent\indent\indent\indent\parbox{\linewidth}{\strut1.2.3. At any given instant, $\neg (maj\_accepted(s) \land maj\_accepted(s'))$ } \par
\indent\indent\indent\indent\indent\indent\parbox{\linewidth-\algorithmicindent*6}{\strut\textbf{Proof:} Since $s$ and $s'$ are non-commutative, therefore not equivalent, and each correct acceptor only votes once for a new proposal (Algorithm \ref{BFT-Acc}, lines \{28-43\}), any learner will only obtain $N-f$ votes for one of the sequences (Algorithm \ref{BFT-Learn}, lines \{1-4\}).}\par
\indent\indent\indent\indent\indent\parbox{\linewidth}{\strut1.2.4. Q.E.D. }\par
\indent\indent\indent\parbox{\linewidth-\algorithmicindent*3}{\strut1.3. Take $s$ and $s'$, at any given instant, $\forall x \in \mathcal{P}, \exists \sigma_1,\sigma_2 \in \mathcal{P}, (maj\_accepted(s) \lor (min\_accepted(s) \land (s \thicksim x \bullet \sigma_1 \lor x \thicksim s \bullet \sigma_2))) \land (maj\_accepted(s') \lor (min\_accepted(s') \land (s' \thicksim x \bullet \sigma_1 \lor x \thicksim s' \bullet \sigma_2))) \implies s \thicksim x \bullet \sigma$}\par
\indent\indent\indent\indent\parbox{\linewidth}{\strut\textbf{Proof:} By 1.2 and by definition of $(s \thicksim x \bullet \sigma \lor x \thicksim s \bullet \sigma)$.}\par
\indent\indent\indent\parbox{\linewidth-\algorithmicindent*3}{\strut1.4. At any given instant, $\forall s,s' \in \mathcal{P}, \forall l_i,l_j \in \mathcal{L}, learned(l_i,s)\ \land\ learned(l_j,s') \implies \exists \sigma_1,\sigma_2 \in \mathcal{P}, s \thicksim s' \bullet \sigma_1 \lor s' \thicksim s \bullet \sigma_2$ }\par
\indent\indent\indent\indent\parbox{\linewidth}{\strut\textbf{Proof:} By 1.1 and 1.3.}\par
\indent\indent\indent\parbox{\linewidth}{\strut1.5. Q.E.D. }\par
\parbox{\linewidth-\algorithmicindent*3}{\strut2. At any given instant, $\forall l_i,l_j \in \mathcal{L}, learned(l_j,learned_j) \land learned(l_i,learned_i) \implies \exists \sigma_1,\sigma_2 \in \mathcal{P}, learned_i \thicksim learned_j \bullet \sigma_1 \lor learned_j \thicksim learned_i \bullet \sigma_2$}\par
\indent\indent\parbox{\linewidth}{\strut\textbf{Proof:} By 1.}\par
\parbox{\linewidth}{\strut3. Q.E.D.} \par

\subsubsection{Non-Triviality}
\begin{theorem}
If all proposers are correct, $learned_l$ can only contain proposed commands. \label{N-T1} \par
\end{theorem} 
\textbf{Proof:} \par
\parbox{\linewidth}{\strut1. At any given instant, $\forall l_i \in \mathcal{L}, \forall s \in \mathcal{P}, learned(l_i,s) \implies \forall x \in \mathcal{P}, \exists \sigma \in \mathcal{P},  maj\_accepted(s) \lor (min\_accepted(s) \land  (s \thicksim x \bullet \sigma \lor x \thicksim s \bullet \sigma))$ }\par
\indent\indent\parbox{\linewidth}{\strut\textbf{Proof:} By Algorithm \ref{BFT-Acc} lines \{28-43\} and Algorithm \ref{BFT-Learn} lines \{1-4,6-11\}, if a correct learner learned a sequence $s$ at any given instant then either $N-f$ or $f+1$ (if $s$ is universally commutative) acceptors must have executed phase $2b$ for $s$.}\par
\parbox{\linewidth}{\strut2. At any given instant, $\forall s \in \mathcal{P}, maj\_accepted(s) \lor min\_accepted(s) \implies proposed(s)$ }\par
\indent\indent\parbox{\linewidth}{\strut\textbf{Proof:} By Algorithm \ref{BFT-Acc} lines \{15-19, 28-43\}, for either $N-f$ or $f+1$ acceptors to accept a proposal it must have been proposed by a proposer.}\par
\parbox{\linewidth}{\strut3. At any given instant, $\forall s \in \mathcal{P}, learned(l_i,s) \implies proposed(s),\forall l_i \in \mathcal{L}$}\par
\indent\indent\parbox{\linewidth}{\strut\textbf{Proof:} By 1 and 2.}\par
\parbox{\linewidth}{\strut4. Q.E.D.}\par

\subsubsection{Stability}
\begin{theorem}
If $learned_l = s$ then, at all later times, $s \sqsubseteq learned_l$, for any sequence $s$ and correct learner $l$ \par \label{S-T1}
\end{theorem} 
\textbf{Proof:} By Algorithm \ref{BFT-Learn} lines \{1-4,6-11\}, a correct learner can only append new commands to its $learned$ command sequence.

\subsubsection{Liveness}
\begin{theorem}
For any proposal $s$ and correct learner $l$, eventually $learned_l$ contains $s$ \label{L-T1} \par
\end{theorem} 
\parbox{\linewidth}{\textbf{Proof:}} \par
\parbox{\linewidth}{\strut1. $\forall\ l_i \in \mathcal{L},\forall s,x \in \mathcal{P}, \exists \sigma \in \mathcal{P}, maj\_accepted(s) \lor (min\_accepted(s) \land  (s \thicksim x \bullet \sigma \lor x \thicksim s \bullet \sigma))\overset{e}{\implies} learned(l_i,s)$}\par
\indent\indent\parbox{\linewidth}{\strut\textbf{Proof:} By Algorithm \ref{BFT-Acc} lines \{28-43\} and Algorithm \ref{BFT-Learn} lines \{1-4,6-11\}, when either $N-f$ or $f+1$ (if $s$ is universally commutative) acceptors accepts a sequence $s$ (or some equivalent sequence), eventually $s$ will be learned by any correct learner.}\par
\parbox{\linewidth}{\strut2. $\forall s \in \mathcal{P}, proposed(s) \overset{e}{\implies} \forall x \in \mathcal{P}, \exists \sigma \in \mathcal{P}, maj\_accepted(s) \lor (min\_accepted(s) \land  (s \thicksim x \bullet \sigma \lor x \thicksim s \bullet \sigma))$} \par
\indent\indent\parbox{\linewidth}{\strut\textbf{Proof:} A proposed sequence is either conflict-free when its incorporated into every acceptor's current sequence or it creates conflicting sequences at different acceptors. In the first case, it's accepted by a quorum (Algorithm \ref{BFT-Acc} lines \{38-43\}) and, in the second case, it's sent in phase $1b$ messages to the in leader in the next ballot (Algorithm \ref{BFT-Acc} lines \{20-26\}) and incorporated in the next proposal (Algorithm \ref{BFT-Lead} lines \{13-18,20-32\}).} \par
\parbox{\linewidth}{\strut3. $\forall l_i \in \mathcal{L}, \forall s \in \mathcal{P}, proposed(s) \overset{e}{\implies} learned(l_i,s)$} \par
\indent\indent\parbox{\linewidth}{\strut\textbf{Proof:} By 1 and 2.} \par
\parbox{\linewidth}{\strut4. Q.E.D.}
% %
%\begin{appendices}

\section{Crash-fault Tolerant Protocol}
\section{Crash Fault Model} \label{Crash Fault Model}

This section describes the crash fault tolerant version of the Generalized Paxos protocol for our simplified problem. The only modifications applied to the protocol were made to make it simpler while still ensuring its correctness. The protocol should still be recognizable as Generalized Paxos since its message pattern and control flow remain the same. However, we chose to describe it in detail, both in the interest of clarity and also so that in the description of later iterations of the protocol we can focus on the aspects that differ. 

\subsection{Agreement protocol} 

The consensus protocol allows learner processes to agree on equivalent sequences of commands (according to our previous definition of equivalence).
An important conceptual distinction between the original Paxos protocol and Generalized Paxos is that, in the original Paxos, each instance of consensus is called a ballot and agrees upon a single value, whereas in Generalized Paxos, instead of being separate instances of consensus, ballots correspond to an extension to the sequence of learned commands of a single ongoing consensus instance.
In both protocols, ballots can either be \textit{classic} or \textit{fast}. \par

In classic ballots, a leader proposes a single sequence of commands, such that it can be appended to the commands learned by the learners. 
A classic ballot in Generalized Paxos follows a protocol that is very similar to the one used by classic Paxos~\cite{Lam98}. This protocol comprises a first phase where each acceptor conveys to the leader the sequences that the acceptor has already voted for (so that the leader can resend commands that may not have gathered enough votes).
This is followed by a second phase where the leader picks an extension to the sequence of previously proposed commands and broadcasts it to the acceptors. The acceptors send their votes to the learners, who then, after gathering enough support for a given extension to the current sequence, append the new commands to their own sequences of learned commands and discard the already learned ones.\par

In fast ballots, multiple proposers can concurrently propose either single commands or sequences of commands by sending them directly to the acceptors. (We use the term \textit{proposal} to denote either the command or sequence of commands that was proposed.)
In this case, concurrency implies that acceptors may receive proposals in a different order. If the resulting sequences are equivalent, then the fast ballots are successfully learned in two message delays. If not, the protocol must fall back to using a classic ballot.

Next, we present the protocol for each type of ballot in detail.

\subsection{Classic ballots} 

Classic ballots work in a way that is very close to the original Paxos protocol~\cite{Lam98}. Therefore, throughout our description, we will highlight the points where Generalized Paxos departs from that original protocol, in particular where it's due to behaviors that are particular to the our simplified specification of Generalized Paxos.

In this part of the protocol, the leader continuously collect proposals by assembling commands that received from the proposers in a sequence. This sequence is built by appending arriving proposals to a sequence containing every proposal received since the previous ballot. (This differs from classic Paxos, where it suffices to keep a single proposed value that the leader attempts to reach agreement on.)

When the next ballot is triggered, the leader starts the first phase by sending phase $1a$ messages to all acceptors containing just the ballot number. Similarly to classic Paxos, acceptors reply with a phase $1b$ message to the leader, which reports all sequences of commands they voted for. This message also carries a promise not to participate in lower-numbered ballots, in order to prevent safety violations~\cite{Lam98}.

After gathering a quorum of $N-f$ phase $1b$ messages, the leader initiates phase $2a$ by sending a message with a proposal to the acceptors. This proposal is assembled by appending the sequences gathered from the proposers to a sequence that contains every command in the sequences that were previously accepted by the acceptors in the quorum (instead of sending a single value with the highest ballot number in the classic specification).

The acceptors reply to phase $2a$ messages by sending phase $2b$ messages to the learners, containing the ballot and the proposal from the leader. After receiving $N-f$ votes for a sequence, a learner learns it by extracting the commands that are not contained in his $learned$ sequence and appending them in order. Note that for a sequence to be learned, a learner doesn't have to receive $N-f$ votes for the exact same sequence but for equivalence sequences (in accordance to our previous definition of equivalence).

\subsection{Fast ballots} 

In contrast to classic ballots, fast ballots are able to leverage the weaker specification of generalized consensus (compared to classic consensus) in terms of command ordering at different replicas, to allow for the faster execution of commands in some cases.

The basic idea of fast ballots is that proposers contact the acceptors directly, bypassing the leader, and then the acceptors send their vote for the current sequence directly to the learners, where this sequence now incorporates the proposed value. If a learner can gather $N-f$ votes for a sequence (or an equivalent one), then it is learned. If, however, a conflict exists between sequences then they will not be considered equivalent and at most one of them will gather enough votes to be learned. Conflicts are dealt with by maintaining the proposals at the acceptors so they can be sent to the leader and learned in the next classic ballot. This differs from Fast Paxos where recovery is performed through an additional round-trip. \par
Next, we explain each of these steps in more detail.


\noindent {\bf Step 1: Proposer to acceptors.}
To initiate a fast ballot, the leader informs both proposers and acceptors that the proposals may be sent directly to the acceptors. Unlike classic ballots, where the sequence proposed by the leader consists of the commands received from the proposers appended to previously proposed commands, in a fast ballot, proposals can be sent to the acceptors in the form of either a single command or a sequence to be appended to the command history. These proposals are sent directly from the proposers to the acceptors.

\noindent {\bf Step 2: Acceptors to learners.}
Acceptors append the proposals they receive to the proposals they have previously accepted in the current ballot and broadcast the result to the learners. Similarly to what happens in classic ballots, the fast ballot equivalent of the phase $2b$ message, which is sent from acceptors to learners, contains the current ballot number and the command sequence. However, since commands (or sequences of commands) are concurrently proposed, acceptors can receive and vote for non-commutative proposals in different orders. To ensure safety, correct learners must learn non-commutative commands in a total order. To this end, a learner must gather $N-f$ votes for equivalent sequences. That is, sequences do not necessarily have to be equal in order to be learned since commutative commands may be reordered. Recall that a sequence is equivalent to another if it can be transformed into the second one by reordering its elements without changing the order of any pair of non-commutative commands. (Note that, in the pseudocode, equivalent sequences are being treated as belonging to the same index of the \emph{messages} variable, to simplify the presentation.) By requiring $N-f$ votes for a sequence of commands, we ensure that, given two sequences where non-commutative commands are differently ordered, only one sequence will receive enough votes. {\color{red} Depends on the system size? Is it 2f+1 or what?} Since each acceptor will only vote for a single sequence, there are only enough correct processes to commit one of them. Note that the fact that proposals are sent as extensions of previous sequences is critical to the safety of the protocol. In particular, since the votes from acceptors can be reordered by the network before being delivered at the learners, if these values were single commands it would be impossible to guarantee that non-commutative commands would be learned in a total order. \par

\noindent \textbf{Arbitrating an order after a conflict.} When, in a fast ballot, non-commutative commands are  concurrently proposed, these commands may be incorporated into the sequences of various acceptors in different orders, and therefore the sequences sent by the acceptors in phase $2b$ messages will not be equivalent and will not be learned. In this case, the leader subsequently runs a classic ballot and gathers these unlearned sequences in phase $1b$. Then, the leader will assemble a single serialization for every previously proposed command, which it will then send to the acceptors. Therefore, if non-commutative commands are concurrently proposed in a fast ballot, they will be included in the subsequent classic ballot and the learners will learn them in a total order, thus preserving consistency. \par 
Note that the leader may assemble gathered sequences in an order that is non-commutative to some sequence that has previously gathered enough votes to be learned. Since the leader can only wait $N-f$ phase $1b$ messages from acceptors, there is no way guarantee which sequence may have been learned. However, this is not a problem because, by making learners process phase $2b$ messages in order, if some sequence gathered enough votes to be learned then its commands will be learned before the new non-commutative sequence is and the new-sequence's repeated commands will be discarded.
\begin{table}[h!]
	\renewcommand{\arraystretch}{1.5}
	\centering
	\begin{tabular}{ |c|c|}
		\hline
		\multicolumn{2}{|c|}{Notation}\\
		\hline
		Symbol & Definition \\
		\hline
		$\sqcup$ & least upper bound \\
		\hline
		$\sqcap$ & greatest lower bound \\
		\hline
		$\bullet$ & append operator \\
		\hline
	\end{tabular} 
	\caption{Notation for the pseudocode} 
	\label{table:1}
\end{table}

\begin{algorithm}
	\caption{Generalized Paxos - Proposer p}
	\begin{algorithmic}[1]
		
		\Function{propose}{\textit{C}}
		\If{fast\_ballot}
		\State \Call{send}{$fast, C$} to Acceptors;
		\Else
		\State \Call{send}{\textit{propose, C}} to Leader;
		\EndIf
		\EndFunction
		
		
	\end{algorithmic}
\end{algorithm}

\begin{algorithm}
	\caption{Generalized Paxos - Leader l}
	\textbf{Local variables:} $ballot_l = 0,\ maxTried_l = \bot,\ C_l = \bot, messages = \bot$
	\begin{algorithmic}[1]
		\State \textbf{upon} \textit{receive(propose, C)} from proposer $p_i$ \textbf{do} 
		\State \hspace{\algorithmicindent} $C_l = C$;
		\State \hspace{\algorithmicindent} \Call{phase\_1a}{$ $};
		\State
		\State \textbf{upon} \textit{receive(statement)} from acceptor $a_i$ \textbf{do}
		\State \hspace{\algorithmicindent}  $messages[ballot_l][a_i] = statement$;
		\State \hspace{\algorithmicindent} \textbf{if} $\#(messages) \geq \lfloor \frac{N}{2} \rfloor +1$ \textbf{then} 
		\State \hspace{\algorithmicindent}\hspace{\algorithmicindent} \Call{phase\_2a}{$ballot_l, Q$};
		\State
		%\item[] % unnumbered empty line
		\Function{phase\_1a}{}
		\State \Call{send}{$p1a, ballot_l$} to Acceptors;
		\EndFunction
		
		\State
		\Function{phase\_2a}{$bal, Q$}
		\State $maxTried_l$ = \Call{proved\_safe}{$Q, bal$};
		\State $maxTried_l = maxTried_l \bullet C_l$;
		\State \Call{send}{$p2a,ballot_l, maxTried_l$} to Acceptors;
		\EndFunction
		
		\State
		\Function{proved\_safe}{$ballot$}
		\State $maxTried = \bot$;
		\State \textbf{for} $val$ in $message[ballot]$ \textbf{do}
		\State \hspace{\algorithmicindent} \textbf{if} $maxTried = \bot$ or $\Call{is\_prefix}{val, maxTried}$ \textbf{then}
		\State\hspace{\algorithmicindent}\hspace{\algorithmicindent} $maxTried = val$;
		\State \textbf{end for}
		\State \textbf{return} $maxTried$;
		\EndFunction
		
			\State
		\Function{is\_prefix}{$new\_sequence, old\_sequence$}
		\If {$\lVert old\_sequence \rVert > \lVert new\_sequence \rVert$}
		\State \textbf{return} \textit{false};
		\EndIf
		
		\State
		\item[] outer:	
		\State \textbf{for} $c_{old}$ in $old\_sequence$ \textbf{do}
		\State \hspace{\algorithmicindent} \textbf{for} $c_{new}$ in $new\_sequence$ \textbf{do}
		\State\hspace{\algorithmicindent}\hspace{\algorithmicindent} \textbf{if} $c_{old} = c_{new}$ \textbf{then}
		\State \hspace{\algorithmicindent}\hspace{\algorithmicindent}\hspace{\algorithmicindent} \textbf{continue} \textit{outer};
		
		\State \hspace{\algorithmicindent}\hspace{\algorithmicindent} \textbf{else if} !\Call{are\_commutative}{$c_{old}, c_{new}$} \textbf{then}
		\State\hspace{\algorithmicindent}\hspace{\algorithmicindent}\hspace{\algorithmicindent} \textbf{return} \textit{false};
		\State \hspace{\algorithmicindent}\textbf{end for}
		\State \hspace{\algorithmicindent}\textbf{return} \textit{false};
		\State \textbf{end for}
		\State
		\State \textbf{return} \textit{true};
		\EndFunction
		
	\end{algorithmic}
\end{algorithm}

\begin{algorithm}
	\caption{Generalized Paxos - Acceptor a}
	\textbf{Local variables: } $bal_a = 0,\ val_a = \bot$ 
	\begin{algorithmic}[1]
		
		\State \textbf{upon} \textit{receive(fast, val)} from proposer \textit{p} \textbf{do}
		\State \hspace{\algorithmicindent} \Call{phase\_2b\_fast}{$val$};
		
		\State
		\State \textbf{upon} \textit{receive(p1a, ballot)} from leader \textit{l} \textbf{do}
		\State \hspace{\algorithmicindent} \Call{phase\_1b}{$ballot, l$};
		
		\State
		\State \textbf{upon} \textit{receive(p2a, ballot, value)} from leader \textit{l} \textbf{do}
		\State \hspace{\algorithmicindent} \Call{phase\_2b\_classic}{$ballot, value$};
		
		\State
		\Function{phase\_1b}{$m, l$}
		\If {$bal_a < m$}
		\State \Call{send}{$p1b, m, bal_a, val_a$} to leader l;
		\State $bal_a = m$;
		\EndIf
		\EndFunction
		
		\State
		\Function{phase\_2b\_classic}{$m, v$}
		\State $k = max(i\ |\ (i < m) \wedge (\exists a \in Q :\ val_a[i]\ \neq null))$;
		\If {$m \geq k$}
		\State $val_a = v$;
		\State \Call{send}{$p2b, bal_a, val_a$} to learners;
		\EndIf
		\EndFunction
		
		\State
		\Function{phase\_2b\_fast}{$v$}
		\State $k = max(i\ |\ (i < m) \wedge (\exists a \in Q :\ val_a[i]\ \neq null))$;
		\If {$bal_a == k$}
		\State $val_a = val_a \bullet v$;
		\State \Call{send}{$p2b, bal_a, val_a$} to learners;
		\EndIf
		\EndFunction
		
	\end{algorithmic}
\end{algorithm}

\begin{algorithm}
	\caption{Generalized Paxos - Learner l}
	\textbf{Local variables: } $learned = \bot, messages = \bot$ 
	\begin{algorithmic}[1]
		\State \textbf{upon} \textit{receive($p2b, bal, val$)} from acceptor $a_i$ \textbf{do}
		\State \hspace{\algorithmicindent} $messages[bal][a_i] = val$;
		\State \hspace{\algorithmicindent} \textbf{if} {($fast\_ballot$ and $\#(messages[bal]) \geq \lceil \frac{3N}{4} \rceil$) or
			\State \hspace{\algorithmicindent} \hspace{\algorithmicindent}	($classic\_ballot$ and $\#(messages[bal]) \geq \lfloor \frac{N}{2}\rfloor+1$)} \textbf{then}
		\State \hspace{\algorithmicindent} \hspace{\algorithmicindent} \hspace{\algorithmicindent} $learned = learned \sqcup val$;
	\end{algorithmic}
\end{algorithm}


\clearpage
\section{Byzantine Fast Quorums - OUTDATED}
To adapt Generalized Paxos to the Byzantine setting, the fast quorums must be recalculated to ensure agreement. Generalized Consensus' Approximate Theorem 3 states that any two fast quorums, $Q_{f1}$ and $Q_{f2}$, and a classic quorum $Q_c$, must have a non-empty intersection. To adapt the protocol to the Byzantine scenario, this intersection can't only be non-empty, it also needs to be larger than $f$ replicas. We obtain the minimum quorum size by forcing the intersection between quorums to be larger than $f$ in the worst case scenario. In the worst case, given two fast quorums of size $Q_f$ and a classic quorum of size $Q_c$, $Q_c$ would intersect with all the replicas of the fast quorums that don't intersect with each other and with some replicas of the fast quorums that do intersect. We name $x$ as the intersection between the three quorums. To ensure agreement, we need to ensure that the intersection between the classic quorum and the two fast quorums, $x$, will have to be larger than $f$. Therefore, the following statements must always hold:

\begin{gather*}
\begin{cases}
Q_c \geq N - Q_f + N-Q_f + x \\
x > f \\
Q_c = \lceil \frac{N+f}{2}\rceil
\end{cases} \\ 
\end{gather*}

The first equation states that the classic quorum is composed of the replicas of the fast quorums that don't intersect with each other plus some replicas that do intersect. The second equation forces the intersection to always be larger than $f$ and the third equation is the minimum quorum size for classic quorums. To solve the system, we can substitute the $Q_c$ in the first equation by $\lceil \frac{N+f}{2}\rceil$ and we obtain:
\begin{gather*} \\
Q_c \geq N - Q_f + N-Q_f + x \label{eq_1} \tag{1} \\ 
\lceil\frac{N+f}{2}\rceil \geq 2N - 2Q_f + x \label{eq_2} \tag{2} \\
N+f \geq 4N - 4Q_f + 2x \label{eq_3} \tag{3} \\
Q_f \geq \frac{3N+2x-f}{4} \label{eq_4} \tag{4} \\ 
\text{Since $x > f$} \implies Q_f \geq \frac{3N+f+1}{4} \label{eq_5} \tag{5}  \\
\end{gather*}

Note that when transitioning from \eqref{eq_2} to \eqref{eq_3}, the ceiling operator was ignored. However, this is safe because, for any $x$, $y$ and $z$, if $\frac{x}{y} > z$ then $\lceil \frac{x}{y} \rceil > z$ holds.\par

\end{appendices}

\end{document}
%%% Local Variables:
%%% mode: latex
%%% mode: flyspell
%%% Local IspellDict: "american"
%%% mode: outline-minor
%%% End:

