\section{Model}
\label{sec:model}
%
We consider an \emph{asynchronous} system in which
a set of $n \in \mathbb{N}$ processes communicate by 
\emph{sending} and \emph{receiving} messages.
Each process executes an algorithm assigned to it, but may stop executing it by \emph{crashing}.
If a process does not follow the algorithm assigned to it, then it is \emph{byzantine}.
This paper considers the \emph{authenticated} Byzantine model: every process can produce cryptographic digital signatures~\cite{quorum}.
A process may be a \emph{learner}, \emph{proposer}, \emph{leader} or \emph{acceptor}.
Informally, proposers provide input values that must be agreed upon by learners and the acceptors help the learners \emph{agree} on a value.

\paragraph{Problem Statement: Generalized Paxos}
In Generalized Paxos, each learner $l$ maintains a monotonically increasing sequence of commands $learned_l$. 
We define these learned sequences of commands to be equivalent ($\thicksim$) 
if one can be transformed into the other by permuting the elements in a way such that the order of non-commutative pairs is preserved. A sequence $x$ is defined to be a prefix of another sequence $y$ ($x \sqsubseteq y$), if the subsequence of $y$ that contains all the elements in $x$ is equivalent ($\thicksim$) to $x$. 
We present the requirements for this consensus problem, stated in terms of learned sequences of commands for a learner $l$, $learned_l$:\par
\textbf{Nontriviality} $learned_l$ can only contain proposed commands \par
\textbf{Stability} If $learned_l = v$ then, at all later times, $v \sqsubseteq learned_l$\footnote{$a \sqsubseteq b$ means that $a$ is a prefix of $b$}, for any $l$ and $v$ \par
\textbf{Consistency} At any time and for any two correct learners $l_i$ and $l_j$, $learned_{l_i}$ and $learned_{l_j}$ can subsequently be extended to equivalent sequences\par
\textbf{Liveness} For any proposal $s$ and correct learner $l$, eventually $learned_l$ contains $s$\par
