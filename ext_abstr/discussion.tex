\vspace{-0.3cm}

\section{Conclusion and discussion}
\vspace{-0.2cm}
% and Concluding remarks}
\label{sec:disc}
%
We presented a simplified description of the Generalized Paxos specification and protocol, as well as its extension to be resilient against Byzantine faults.
We now draw some lessons and outline some extensions to our protocol that present interesting directions for future work.

%\shortnegspace

\noindent \textbf{Handling faults in the fast case.}
A result that was stated in the original Generalized Paxos
paper~\cite{Lamport2005} is that to tolerate $f$ crash faults and
allow for fast ballots whenever there are up to $e$ crash faults, the
total system size $N$ must uphold two conditions:
$N > 2f$ and $N > 2e+f$.
Additionally, the fast and classic quorums must be of size $N-e$ and $N-f$, respectively. This implies that there is a price to pay in terms of number of replicas and quorum size for being able to run fast operations during faulty periods.
An interesting observation is that since Byzantine fault tolerance already requires a total system size of $3f+1$ and a quorum size of $2f+1$, we are able to amortize the cost of both features, i.e., we are able to tolerate the maximum number of faults for fast execution without paying a price in terms of the replication factor and quorum size.

%\shortnegspace

\noindent \textbf{Universally commutative commands.}
% Generalized Paxos vs Paxos spec.
A downside of the use of commutative commands in the
context of Generalized Paxos is that the commutativity check is done
at runtime, to determine if non-commutative commands have
been proposed concurrently.
This raises the possibility of extending the protocol to handle
commands that are universally commutative, i.e., commute with every
other command. For these commands, it is known before executing them
that they will not generate any conflicts, and therefore it is not
necessary to check them against concurrently executing commands.  This
allows us to optimize the protocol by decreasing the number of phase
$2b$ messages required to learn to a smaller $f+1$ quorum since $N-f$ votes aren't required to prevent learners from learning conflicting sequences. A quorum of $f+1$ is sufficient to
ensure that a correct acceptor will eventually
propagate the command to a quorum of $N-f$ acceptors. This optimization is particularly useful in the context of 
geo-replicated systems, since it can be significantly faster to wait for the $f+1$st message instead of the $N-f$th one. 
The link between Generalized Paxos and weak consistency models  becomes clearer with the introduction of universally commutative commands. In the case of weakly consistent replication, weakly consistent requests can be executed as if they were universally commutative, even if in practice that may not be the case. E.g., checking the balance of a bank account and making a deposit do not commute since the output of the former depends on their relative order. However, some systems prefer to run both as weakly consistent operations~\cite{Li2012}. \looseness=-1