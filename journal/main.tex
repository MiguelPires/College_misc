%  LaTeX support: latex@mdpi.com 
%  In case you need support, please attach all files that are necessary for compiling as well as the log file, and specify the details of your LaTeX setup (which operating system and LaTeX version / tools you are using).

% You need to save the "mdpi.cls" and "mdpi.bst" files into the same folder as this template file.

%=================================================================
\documentclass[algorithms,article,submit,moreauthors,pdftex,10pt,a4paper]{mdpi} 
%
%--------------------
% Class Options:
%--------------------
% journal
%----------
% Choose between the following MDPI journals:
% actuators, addictions, admsci, aerospace, agriculture, agronomy, algorithms, animals, antibiotics, antibodies, antioxidants, applsci, arts, asi, atmosphere, atoms, axioms, batteries, bdcc, behavsci, beverages, bioengineering, biology, biomedicines, biomimetics, biomolecules, biosensors, brainsci, buildings, carbon, cancers, catalysts, cells, ceramics, challenges, chemengineering, chemosensors, children, chromatography, climate, coatings, colloids, computation, computers, condensedmatter, cosmetics, cryptography, crystals, cybersecurity, data, dentistry, designs, diagnostics, diseases, diversity, econometrics, economies, education, electrochem, electrochemistry, electronics, energies, entropy, environments, epigenomes, est, fermentation, fibers, fire, fishes, fluids, foods, forecasting, forests, fractalfract, futureinternet, galaxies, games, gastrointestdisord, gels, genealogy, genes, geosciences, geriatrics, hazardousmatters, healthcare, heritage, highthroughput, horticulturae, humanities, hydrology, informatics, information, infrastructures, inorganics, insects, instruments, ijerph, ijfs, ijms, ijgi, ijtpp, inventions, j, jcdd, jcm, jcs, jdb, jfb, jfmk, jimaging, jof, jintelligence, jlpea, jmmp, jmse, jpm, jrfm, jsan, land, languages, laws, life, literature, logistics, lubricants, machines, magnetochemistry, make, marinedrugs, materials, mathematics, mca, mti, medsci, medicines, membranes, metabolites, metals, microarrays, micromachines, microorganisms, minerals, modelling, molbank, molecules, mps, nanomaterials, ncrna, neonatalscreening, neuroglia, nitrogen, nutrients, ohbm, particles, pathogens, pharmaceuticals, pharmaceutics, pharmacy, philosophies, photonics, plants, plasma, polymers, polysaccharides, preprints, proceedings, processes, proteomes, publications, quaternary, qubs, reactions, recycling, religions, remotesensing, reports, resources, risks, robotics, safety, scipharm, sensors, separations, sexes, sinusitis, socsci, societies, soils, sports, standards, stats, surfaces, surgeries, sustainability, symmetry, systems, technologies, toxics, toxins, tropicalmed, universe, urbansci, vaccines, vetsci, vibration, viruses, vision, water, wem
%---------
% article
%---------
% The default type of manuscript is article, but can be replaced by: 
% abstract, addendum, article, benchmark, book, bookreview, briefreport, casereport, changes, comment, commentary, communication, conceptpaper, correction, conferenceproceedings, conferencereport, expressionofconcern, meetingreport, creative, datadescriptor, discussion, editorial, essay, erratum, hypothesis, interestingimages, letter, meetingreport, newbookreceived, opinion, obituary, projectreport, reply, reprint, retraction, review, perspective, protocol, shortnote, supfile, technicalnote, viewpoint
% supfile = supplementary materials
% protocol: If you are preparing a "Protocol" paper, please refer to http://www.mdpi.com/journal/mps/instructions for details on its expected structure and content.
%----------
% submit
%----------
% The class option "submit" will be changed to "accept" by the Editorial Office when the paper is accepted. This will only make changes to the frontpage (e.g. the logo of the journal will get visible), the headings, and the copyright information. Also, line numbering will be removed. Journal info and pagination for accepted papers will also be assigned by the Editorial Office.
%------------------
% moreauthors
%------------------
% If there is only one author the class option oneauthor should be used. Otherwise use the class option moreauthors.
%---------
% pdftex
%---------
% The option pdftex is for use with pdfLaTeX. If eps figures are used, remove the option pdftex and use LaTeX and dvi2pdf.

%=================================================================
\firstpage{1} 
\makeatletter 
\setcounter{page}{\@firstpage} 
\makeatother 
\articlenumber{x}
\doinum{10.3390/------}
\pubvolume{xx}
\pubyear{2018}
\copyrightyear{2018}
%\externaleditor{Academic Editor: name}
\history{Received: date; Accepted: date; Published: date}
%\ISSN{xxxx--xxxx}
%\issuenum{x}
%\updates{yes} % If there is an update available, un-comment this line
%-----------------------

\usepackage{amssymb}
%\bibliographystyle{plainurl}% the recommended bibstyle
\usepackage{multicol}
\usepackage{amsmath}
\usepackage{amsthm}
\usepackage[ruled]{algorithm}
\usepackage{algpseudocode}
%\usepackage{algorithmic}
\usepackage{tabularx}
\usepackage[colorinlistoftodos,prependcaption,textsize=small]{todonotes}\setlength{\marginparwidth}{2cm}
\usepackage{csquotes}

%------------------------------------------------------------------
% The following line should be uncommented if the LaTeX file is uploaded to arXiv.org
%\pdfoutput=1

%=================================================================
% Add packages and commands here. The following packages are loaded in our class file: fontenc, calc, indentfirst, fancyhdr, graphicx, lastpage, ifthen, lineno, float, amsmath, setspace, enumitem, mathpazo, booktabs, titlesec, etoolbox, amsthm, hyphenat, natbib, hyperref, footmisc, geometry, caption, url, mdframed, tabto, soul, multirow, microtype, tikz

%=================================================================
%% Please use the following mathematics environments: Theorem, Lemma, Corollary, Proposition, Characterization, Property, Problem, Example, ExamplesandDefinitions, Hypothesis, Remark, Definition
%% For proofs, please use the proof environment (the amsthm package is loaded by the MDPI class).

%=================================================================
% Full title of the paper (Capitalized)
\Title{Generalized Paxos Made Byzantine (and Less Complex)}

% Author Orchid ID: enter ID or remove command
\newcommand{\orcidauthorA}{0000-0000-000-000X} % Add \orcidA{} behind the author's name
%\newcommand{\orcidauthorB}{0000-0000-000-000X} % Add \orcidB{} behind the author's name

% Authors, for the paper (add full first names)
\Author{Miguel Pires $^{1}$, Srivatsan Ravi $^{2}$ and Rodrigo Rodrigues $^{1,}$*}

% Authors, for metadata in PDF
\AuthorNames{Miguel Pires, Srivatsan Ravi and Rodrigo Rodrigues}

% Affiliations / Addresses (Add [1] after \address if there is only one affiliation.)
\address{%
$^{1}$ \quad INESC-ID and Instituto Superior Técnico, R. Alves Redol 9, 1000-029, Lisbon, Portugal; miguel.pires@tecnico.ulisboa.pt; rodrigo.miragaia.rodrigues@tecnico.ulisboa.pt\\
$^{2}$ \quad University of Southern California, Los Angeles, CA 90007, USA; srivatsan@srivatsan.in}

% Contact information of the corresponding author
\corres{Correspondence: e-mail@e-mail.com; Tel.: +x-xxx-xxx-xxxx}

% Current address and/or shared authorship
%\firstnote{Current address: Affiliation 3} 

% The commands \thirdnote{} till \eighthnote{} are available for further notes

% Simple summary
%\simplesumm{}

% Abstract (Do not insert blank lines, i.e. \\) 
\abstract{One of the most recent members of the \emph{Paxos} family of protocols is \emph{Generalized} Paxos. This variant of Paxos has the characteristic that it departs from the original specification of consensus, allowing for a weaker safety condition where different processes can have a different views on a sequence being agreed upon. However, much like the original Paxos counterpart, Generalized Paxos does not have a simple implementation. Furthermore, with the recent practical adoption of Byzantine fault tolerant protocols in the context of blockchain protocols, it is timely and important to understand how Generalized Paxos can be implemented in the Byzantine model.	In this paper, we make two main contributions. First, we attempt to provide a simpler description of Generalized Paxos, based on a simpler specification and the pseudocode for a solution that can be readily implemented. Second, we extend the protocol to the Byzantine fault model, and provide the respective correctness proof.}

% Keywords
\keyword{Byzantine Fault Tolerance; Consensus; Paxos}

% The fields PACS, MSC, and JEL may be left empty or commented out if not applicable
%\PACS{J0101}
%\MSC{}
%\JEL{}

\begin{document}

\section{Introduction}
Distributed protocols for building stateful fault-tolerant applications in data centers typically strive to provide strong consistency \`a la \emph{linearizability}~\cite{Her91}.
However, the complexity of achieving consistent replication of state while ensuring availability makes it hard to justify its deployment in \emph{geo-distributed data centers}.
The increasing popularity of replicating online services across diverse data centers emphasizes 
the need for low-latency protocols that cater to failure patterns specific to such a setting while still maintaining 
some \emph{relaxed} form of \emph{cross-site consistency}. There is a need for constructing modular protocols for state-machine replication
in such geo-distributed settings that are tunable to specific consistency levels and failure patterns.
%


% %
\section{Background and Overview}
\label{sec:related} 
\subsection{Paxos and its variants} \label{Paxos} 
\paragraph{Classic Paxos.}
The Paxos protocol family solves consensus by finding an equilibrium in face of the well-known FLP impossibility result~\cite{FLP85}. It does this by always guaranteeing safety in an asynchronous system, but at the same time making the observation that most of the time systems have periods during which they can be considered synchronous, since long delays are often sporadic and temporary. Therefore, Paxos only foregoes progress during the temporary periods of asynchrony, or if more than $f$ faults occur for a system of $n=2f+1$ replicas~\cite{L01}. The classic form of Paxos employs a set of proposers, acceptors and learners, runs in a sequence of ballots, and employs two phases (numbered 1 and 2), with a similar message pattern: proposer to acceptors, acceptors to proposer (and, in phase 2, also acceptors to learners). To ensure progress during synchronous periods, proposals are serialized by a distinguished proposer, which is called the leader.\par
Paxos is most commonly deployed as Multi (Decree)-Paxos, which provides an optimization of the basic message pattern by omitting the first phase of messages from all but the first ballot for each leader~\cite{Renesse2011}. This means that a leader only needs to send a \textit{phase 1a} message once and subsequent proposals may be sent directly in \textit{phase 2a} messages. This reduces the message pattern in the common case from five message delays to just three (from proposal to learning). Since there are no implications on the quorum size or guarantees provided by Paxos, the reduced latency comes at no additional cost. \par

\paragraph{Fast Paxos.}
Fast Paxos observes that it is possible to improve on the previous latency (in terms of common case message steps) by allowing proposers to propose values directly to acceptors \cite{L06}. To this end, the protocol distinguishes between fast and classic ballots, where fast ballots bypass the leader by sending proposals directly to acceptors and classic ballots work as in Basic Paxos. The reduced latency of fast ballots comes at the additional cost of using a quorum size of $n-e$ instead of a classic majority quorum, where $e$ is the number of faults that can be tolerated while using fast ballots. In addition, instead of the usual requirement that $n> 2f$, to ensure that fast and classic quorums intersect, a new requirement must be met: $n > 2e+f$. This means that if we wish to tolerate the same number of faults for classic and fast ballots (i.e., $e=f$), then the total number of replicas is $3f+1$ instead of the usual $2f+1$ and the quorum size for fast and classic ballots is the same. The optimized commit scenario occurs during fast ballots, in which only two messages broadcasts are necessary: \textit{phase 2a} messages between a proposer and the acceptors, and \textit{phase 2b} messages between acceptors and learners. This creates the possibility of two proposers concurrently proposing values to the acceptors and generating a conflict, which must be resolved by falling back to a recovery protocol. \par

\paragraph{Generalized Paxos.}
Generalized Paxos addresses Fast Paxos' shortcomings regarding collisions. More precisely, it allows acceptors to accept different sequences of commands as long as non-commutative operations are totally ordered \cite{Lamport2005}.  Non-commutativity between operations is generically represented as an interference relation. Generalized Paxos abstracts the traditional consensus problem of agreeing on a single value to the problem of agreeing on an increasing set of values. \textit{C-structs} provide this abstraction of an increasing set of values and allow us to define different consensus problems. If we define the sequence of learned commands of a learner $l_i$ as a \textit{c-struct} $learned[l_i]$, then the consistency requirement for consensus can be defined as:
\begin{itemize}
\item \textbf{Consistency} -- $learned[l_1]$ and $learned[l_2]$ are always compatible, for all learners $l_1$ and $l_2$.
\end{itemize}
For two \textit{c-structs} to be compatible, they must have a \textit{common upper bound}. This means that, for any two learned \textit{c-structs} such as $learned[l_1]$ and $learned[l_2]$, there must exist some \textit{c-struct} to which they are both prefixes. This prohibits non-commutative commands from being concurrently accepted because no subsequent \textit{c-struct} would extend them both since it would not have a total order of non-commutative operations. For instance, consider a set of commands $\lbrace A$, $B$, $C\rbrace$ and an interference relation between commands $A$ and $B$ (i.e., they are non-commutative with respect to each other). If proposers propose $A$ and $C$ concurrently, some learners may learn one command before the other and the resulting \textit{c-structs} would be either $C \bullet A$ or $A \bullet C$. These are compatible because there are \textit{c-structs} that extend them, namely $A \bullet C \bullet B$ and $C \bullet A \bullet B$. These \textit{c-structs} that extend them are valid because the interfering commands are totally ordered. However, if two proposers propose $A$ and $B$, learners could learn either one in the first ballot and these \textit{c-structs} would not be compatible because no \textit{c-struct} extends them. Any \textit{c-struct} would start either by $A \bullet B$ or $B \bullet A$, which means that an interference relation would be violated. In the Generalized Paxos protocol, when such a collision occurs, no value is chosen and the leader intervenes by starting a new ballot and proposing a \textit{c-struct}. Defining \textit{c-structs} as command histories enables acceptors to agree on different sequences of commands and still preserve consistency as long as dependence relationships are not violated. This means that commutative commands can be ordered differently regarding each other but interfering commands must preserve the same order across each sequence at any learner. This guarantees that solving the consensus problem for histories is enough to implement a state-machine replicated system. \par

\paragraph{Mencius.}
Mencius is also a variant of Paxos that tries to address the bottleneck of having a single leader, through which every proposal must go through. In Mencius, the leader of each round rotates between every process: the leader of round $i$ is the process $p_k$, such that $k = n\ mod\ i$.  Leaders with nothing to propose can skip their turn by proposing a \textit{no-op}. If a leader is slow or faulty, the other replicas can execute \textit{phase 1} to revoke the leader's right to propose a value, but they can only propose a \textit{no-op} instead \cite{Mao2008}. Considering that non-leader replicas can only propose \textit{no-ops}, a \textit{no-op} command from the leader can be accepted in a single message delay since there is no chance of another value being accepted. If some non-leader server revokes the leader's right to propose and suggests a \textit{no-op}, then the leader can still suggest a value $v \neq$ \textit{no-op}, which will eventually be accepted as long as $l$ is not permanently suspected. Mencius also takes advantage of commutativity by allowing out-of-order commits, where values $x$ and $y$ can be learned in different orders by different learners if there does not exist a dependence relationship between them.

\paragraph{Egalitarian Paxos.}
Egalitarian Paxos (EPaxos) extends the goal of Mencius of achieving a better throughput than Paxos by removing the bottleneck caused by having a leader \cite{Moraru2013}. To avoid choosing a leader, the proposal of commands for a command slot is done in a decentralized manner, taking advantage of the commutativity observations made by Generalized Paxos \cite{Lamport2005}. If two replicas unknowingly propose commands concurrently, one will commit its proposal in one round trip after getting replies from a quorum of replicas. However, some replica will see that another command was concurrently proposed and may interfere with the already committed command. If the commands are non-commutative then the replica must reply with a dependency between the commands, committing its command in two rounds trips. This commit latency is achieved by using a \textit{fast-path quorum} of $f+\lfloor\frac{f+1}{2}\rfloor$ replicas. Similarly to Mencius, EPaxos achieves a substantially higher throughput than Multi-Paxos.

\subsection{Byzantine fault tolerant replication} \label{Non-Crash}
%Non-crash fault models emerged to cope with the effect of malicious attacks and software errors. These models (e.g., the arbitrary fault model) assume a stronger adversary than previous crash fault models. 
The Byzantine Generals Problem is defined as a set of Byzantine generals that are camped in the outskirts of an enemy city and have to coordinate an attack. Each general can either decide to attack or retreat and there may be $f$ traitors among the generals that try to prevent the loyal generals from agreeing on the same action. The problem is solved if every loyal general agrees on what action to take \cite{LSP82}. Like the traitorous generals, a process that suffers a Byzantine fault may display an arbitrary behaviour and, in case of multiple Byzantine faults, an adversary may even coordinate multiple faulty replicas in an attack. \par

\paragraph{Practical Byzantine Fault Tolerance (PBFT).}
PBFT is a protocol that solves consensus while tolerating up to $f$ Byzantine faults \cite{CL99}. The system moves through configurations called \textit{views} in which one replica is the primary and the remaining replicas are the backups. The safety property of the algorithm requires that operations be totally ordered. The protocol starts when a client sends a request for an operation to the primary, which in turn assigns a sequence number to the request and multicasts a \textit{pre-prepare} message to the backups. This message contains the timestamp, the digest of the client's message, the view and the actual request. If a backup replica accepts the pre-prepare message, after verifying that the view number and timestamp are correct, it multicasts a \textit{prepare} message and adds both messages to its log. The prepare message is similar to the pre-prepare message except that it does not contain the client's request message. Both of these phases ensure that the requested operation is totally ordered at every correct replica (note that the two phases described informally are not necessary for safety as demonstrated by \cite{Dolev1983}). The protocol's safety property requires that the replicated service must satisfy linearizability and, therefore, operations must be totally ordered. After receiving $2f$ prepare messages, a replica multicasts a \textit{commit} message and commits the message to its log when it has received $2f$ commit messages from other replicas. The liveness property requires that clients must eventually receive replies to their requests, provided that there are at most $\lfloor\frac{N-1}{3}\rfloor$ faults and the transmission time doesn't increase continuously. This property represents a weak liveness condition but one that is enough to circumvent the FLP impossibility result \cite{FLP85}. A Byzantine leader may try to prevent progress by omitting pre-prepare messages when it receives operation requests from clients, but backups can trigger new views after waiting for a certain period of time. The description of the Byzantine Generalized Paxos protocol presented in this paper largely follows from the PBFT protocol.
\par


% %
\section{Model}
\label{sec:model}
%
We consider an \emph{asynchronous} system in which
a set of $n \in \mathbb{N}$ processes communicate by 
\emph{sending} and \emph{receiving} messages.
Each process executes an algorithm assigned to it, but may stop executing it by \emph{crashing}.
If a process does not follow the algorithm assigned to it, then it is \emph{byzantine}.
This paper considers the \emph{authenticated} Byzantine model: every process can produce cryptographic digital signatures~\cite{quorum}. 
Furthermore, for clarity of exposition, we assume authenticated perfect links~\cite{cgr:book}, 
where a message that is sent by a non-faulty sender is eventually received and messages cannot be forged 
(such links can be implemented trivially using retransmission, elimination of duplicates, and point-to-point message authentication codes~\cite{cgr:book}.)
A process may be a \emph{learner}, \emph{proposer} or \emph{acceptor}.
Informally, proposers provide input values that must be agreed upon by learners and the acceptors help the learners \emph{agree} on a value.

\paragraph*{Problem Statement: Generalized Paxos}
In Generalized Paxos, each learner $l$ maintains a monotonically increasing sequence of commands $learned_l$. 
We define these learned sequences of commands to be equivalent ($\thicksim$) 
if one can be transformed into the other by permuting the elements in a way such that the order of non-commutative pairs is preserved. A sequence $x$ is defined to be a \textit{eq-prefix} of another sequence $y$ ($x \sqsubseteq y$), if the subsequence of $y$ that contains all the elements in $x$ is equivalent ($\thicksim$) to $x$. 
We present the requirements for this consensus problem, stated in terms of learned sequences of commands for a learner $l$, $learned_l$. To simplify the original specification, instead of using C-structs (as explained in Section~\ref{sec:related}), we specialize to agreeing on equivalent sequences of commands:\par
\textbf{Nontriviality.} $learned_l$ can only contain proposed commands. \par
\textbf{Stability.} If $learned_l = v$ then, at all later times, $v \sqsubseteq learned_l$, for any $l$ and $v$. \par
\textbf{Consistency.} At any time and for any two correct learners $l_i$ and $l_j$, $learned_{l_i}$ and $learned_{l_j}$ can subsequently be extended to equivalent sequences.\par
\textbf{Liveness.} For any proposal $s$ and correct learner $l$, eventually $learned_l$ contains $s$.\par

%
\section{Protocol}

%\subsection{Description}
This section presents our Byzantine fault tolerant Generalized Paxos
Protocol (or BGP, for short). Given our space constraints, we opted
for merging in a single description a novel presentation of
Generalized Paxos and its extension to the Byzantine model, even though
each represents an independent contribution in its own right.


\begin{algorithm}
	\caption{Byzantine Generalized Paxos - Proposer p}
	\label{BFT-Prop}
	\textbf{Local variables:} $ballot\_type = \bot$
	\begin{algorithmic}[1]	
		
		\State \textbf{upon} \textit{receive(BALLOT, type)} \textbf{do} 
		\State \hspace{\algorithmicindent} $ballot\_type = type$;
		\State
		
		\State \textbf{upon} \textit{command\_request(c)} \textbf{do}   \hspace{\algorithmicindent}\hspace{\algorithmicindent}\hspace{\algorithmicindent}\hspace{\algorithmicindent}
		\State \hspace{\algorithmicindent} \textbf{if} $ballot\_type = fast\_ballot$ \textbf{then}
		\State \hspace{\algorithmicindent}\hspace{\algorithmicindent} \Call{send}{$P2A\_FAST, c$} to acceptors;
		\State \hspace{\algorithmicindent} \textbf{else} 
		\State \hspace{\algorithmicindent}\hspace{\algorithmicindent} \Call{send}{\textit{PROPOSE, c}} to leader;		
	\end{algorithmic}
\end{algorithm}

\subsection{Overview}
We modularize our protocol explanation according to the following main components, which are also present in other protocols of the Paxos family:

\begin{itemize}

\item
  {\bf View change} -- The goal of this subprotocol is to ensure that, at any given moment, one of the proposers is chosen as a distinguished leader, who runs a specific version of the agreement subprotocol. To achieve this, the view change subprotocol continuously replaces leaders, until one is found that can ensure progress (i.e., commands are eventually appended to the current sequence).

\item
{\bf Agreement} -- Given a fixed leader, this subprotocol extends the current sequence with a new command or set of commands. Analogously to Fast Paxos~\cite{L06} and Generalized Paxos~\cite{Lamport2005}, choosing this extension can be done through two variants of the protocol: using either {\bf classic ballots} or {\bf fast ballots}, with the characteristic that fast ballots complete in fewer communication steps, but may have to fall back to using a classic ballot when there is contention among concurrent requests.

\end{itemize}

\subsection{View change} 

The goal of the view change subprotocol is to elect a distinguished acceptor process, called the leader, that carries through the agreement protocol, i.e., enables proposed commands to eventually be learned by all the learners. The overall design of this subprotocol is similar to the corresponding part of existing BFT state machine replication protocols~\cite{CL99}.\par

\begin{algorithm}
	\caption{Byzantine Generalized Paxos - Process p}
	\begin{algorithmic}[1]		
		\Function{merge\_sequences}{$old\_seq, new\_seq$}
		\State \textbf{for} $c$ \textbf{in} $new\_seq$ \textbf{do} 
		\State \hspace{\algorithmicindent} \textbf{if} $!\Call{contains}{old\_seq,c}$ \textbf{then}
		\State \hspace{\algorithmicindent}\hspace{\algorithmicindent}\hspace{\algorithmicindent} $old\_seq =  old\_seq \bullet c$;
		\State \textbf{end for}
		\State \textbf{return} $old\_seq$;
		\EndFunction
		
		\State
		\Function{signed\_commands}{$full\_seq$}
		\State $signed\_seq = \bot$;
		\State \textbf{for} $c$ \textbf{in} $full\_seq$ \textbf{do}
		\State \hspace{\algorithmicindent} \textbf{if} \Call{$verify\_command$}{c} \textbf{then}
		\State \hspace{\algorithmicindent}\hspace{\algorithmicindent} $signed\_seq = signed\_seq \bullet c$;
		\State \textbf{end for}
		\State \textbf{return} $signed\_seq$;
		\EndFunction
	\end{algorithmic}
\end{algorithm}

In this subprotocol, the system moves through sequentially numbered views, and the leader for each view is chosen in a rotating fashion using the simple equation $\textit{leader(view)}=\textit{view mod N}$. The protocol works continuously by having acceptor processes monitor whether progress is being made on adding commands to the current sequence, and, if not, they multicasting a signed {\sc suspicion} message for the current view to all acceptors suspecting the current leader. Then, if enough suspicions are collected, processes can move to the subsequent view. However, the required number of suspicions must be chosen in a way that prevents Byzantine processes from triggering view changes spuriously. To this end, acceptor processes will multicast a view change message indicating their commitment to starting a new view only after hearing that $f+1$ processes suspect the leader to be faulty. This message contains the new view number, the $f+1$ signed suspicions, and is signed by the acceptor that sends it. This way, if a process receives a view-change message without previously receiving $f+1$ suspicions, it can also multicast a view-change message, after verifying that the suspicions are correctly signed by $f+1$ distinct processes.
%As such, the signatures allow a process that receives this message to commit to the new view and multicast its own view-change message without receiving $f+1$ suspicions itself.
This guarantees that if one correct process receives the $f+1$ suspicions and multicasts the view-change message, then all correct processes, upon receiving this message, will be able to validate the proof of $f+1$ suspicions and also multicast the view-change message.\par
\begin{algorithm} 
	\caption{Byzantine Generalized Paxos - Leader l}
	\label{BFT-Lead}
	\textbf{Local variables:} $ballot_l = 0,maxTried_l = \bot,proposals = \bot, accepted = \bot, notAccepted = \bot, view = 0$
	\begin{algorithmic}[1]
		\State \textbf{upon} \textit{receive($LEADER,view_a,proofs$)} from acceptor \textit{a} \textbf{do}
		\State \hspace{\algorithmicindent} $valid\_proofs = 0$;
		\State \hspace{\algorithmicindent} \textbf{for} $p$ \textbf{in} $acceptors$ \textbf{do} 
		\State \hspace{\algorithmicindent}\hspace{\algorithmicindent} $view\_proof = proofs[p]$;
		
		\State \hspace{\algorithmicindent}\hspace{\algorithmicindent} \textbf{if} $view\_proof_{pub_p} = \langle view\_change, view_a \rangle$ \textbf{then}
		\State \hspace{\algorithmicindent}\hspace{\algorithmicindent}\hspace{\algorithmicindent}  $valid\_proofs \mathrel{+{=}} 1$;
		\State \hspace{\algorithmicindent} \textbf{if} $valid\_proofs > f$ \textbf{then}
		\State \hspace{\algorithmicindent}\hspace{\algorithmicindent} $view = view_a$;
	
		\State
		\State \textbf{upon} \textit{trigger\_next\_ballot(type)} \textbf{do}
		\State \hspace{\algorithmicindent} $ballot_l \mathrel{+{=}} 1$;
		\State \hspace{\algorithmicindent} \Call{send}{$BALLOT,type}$ to proposers;
		\State \hspace{\algorithmicindent} \textbf{if} $type = fast$ \textbf{then}
		\State \hspace{\algorithmicindent}\hspace{\algorithmicindent} \Call{send}{$FAST,ballot_l,view}$ to acceptors;
		\State \hspace{\algorithmicindent} \textbf{else}
		\State \hspace{\algorithmicindent}\hspace{\algorithmicindent} \Call{send}{$P1A, ballot_l, view$} to acceptors;
		
		\State
		\State \textbf{upon} \textit{receive(PROPOSE, prop)} from proposer $p_i$ \textbf{do} 
		\State \hspace{\algorithmicindent} $proposals = proposals \bullet prop$;
		
		\State
		\State \textbf{upon} \textit{receive($P1B, ballot, bal_a, proven,val_a, proofs$)} from acceptor $a$ \textbf{do}
		\State \hspace{\algorithmicindent} \textbf{if} $ballot \neq ballot_l$ \textbf{then}
		\State \hspace{\algorithmicindent}\hspace{\algorithmicindent} \textbf{return};
		\State
		\State \hspace{\algorithmicindent} $valid\_proofs = 0$; 
		\State \hspace{\algorithmicindent} \textbf{for} $i$ \textbf{in} $acceptors$ \textbf{do}
		\State \hspace{\algorithmicindent}\hspace{\algorithmicindent} $proof = proofs[proven][i]$;
		\State \hspace{\algorithmicindent}\hspace{\algorithmicindent} \textbf{if} $proof_{pub_i} = \langle bal_a, proven \rangle$ \textbf{then}
		\State \hspace{\algorithmicindent}\hspace{\algorithmicindent}\hspace{\algorithmicindent} 
		$valid\_proofs \mathrel{+{=}} 1$;
		\State
		\State \hspace{\algorithmicindent} \textbf{if} $valid\_proofs > N-f$ \textbf{then}
		\State \hspace{\algorithmicindent}\hspace{\algorithmicindent}\hspace{\algorithmicindent} $accepted[ballot_l][a] = proven$;
		\State \hspace{\algorithmicindent}\hspace{\algorithmicindent}\hspace{\algorithmicindent}		$notAccepted[ballot_l] = notAccepted[ballot_l] \bullet (val_a \setminus proven)$;		
		
		\State 
		\State \hspace{\algorithmicindent}\hspace{\algorithmicindent} \textbf{if} $\#(accepted[ballot_l]) \geq N-f$ \textbf{then} 
		\State \hspace{\algorithmicindent}\hspace{\algorithmicindent}\hspace{\algorithmicindent} \Call{phase\_2a}{$ $};
		
		\State
		\Function{phase\_2a}{$ $}
		\State $maxTried = \Call{largest\_seq}{accepted[ballot_l]}$;
		\State $previousProposals = \Call{remove\_duplicates}{notAccepted[ballot_l]}$;
		\State $maxTried = maxTried \bullet previousProposals \bullet proposals$;
		\State \textbf{if} $\Call{clean\_state?}{ }$ \textbf{then}
		\State \hspace{\algorithmicindent} $maxTried_l = maxTried_l \bullet C^*$;
		\State \Call{send}{\textit{P2A\_CLASSIC,$ballot_l$,view, $maxTried_l$}} to acceptors;
		\State $proposals = \bot$;
		\EndFunction

	\end{algorithmic}
\end{algorithm}

Finally, an acceptor process must wait for $N-f$ view-change messages to start participating in the new view, i.e., update its view number and the corresponding leader process. At this point, the acceptor also assembles the $N-f$ view-change messages proving that others are committing to the new view, and sends them to the new leader. This allows the new leader to start its leadership role in the new view once it validates the $N-f$ signatures contained in a single message.

\subsection{Agreement protocol} 

The consensus protocol allows learner processes to agree on equivalent sequences of commands (according to our previous definition of equivalence).
An important conceptual distinction between the original Paxos protocol and BGP is that, in the original Paxos, each instance of consensus is called a ballot, whereas in BGP, instead of being a separate instance of consensus, 
ballots correspond to an extension to the sequence of learned commands of a single ongoing consensus instance. Proposers can try to extend the current sequence by either single commands or sequences of commands. We use the term \textit{proposal} to denote either the command or sequence of commands that was proposed.

As mentioned, ballots can either be \textit{classic} or \textit{fast}. In classic ballots, a leader proposes a single proposal to be appended to the commands learned by the learners. The protocol is then similar to the one used by classic Paxos~\cite{Lam98}, with a first phase where each acceptor conveys to the leader the sequences that the acceptor has already voted for (so that the leader can resend commands that may not have gathered enough votes), followed by a second phase where the leader instructs and gathers support for appending the new proposal to the current sequence of learned commands. Fast ballots, in turn, allow any proposer to attempt to contact all acceptors in order to extend the current sequence within only two message delays (in case there are no conflicts between concurrent proposals).

\begin{algorithm} 
	\caption{Byzantine Generalized Paxos - Acceptor a (view-change)}
	\label{BFT-Proc}
	\textbf{Local variables:} $suspicions = \bot,\ new\_view = \bot,\ leader = \bot,\ view = 0, bal_a = 0,\ val_a = \bot,\ fast\_bal = \bot,\ checkpoint=\bot$
	\begin{algorithmic}[1]		
		\State \textbf{upon} \textit{suspect\_leader} \textbf{do} 
		\State\hspace{\algorithmicindent} \textbf{if} $suspicions[p] \neq true$ \textbf{then}
		\State\hspace{\algorithmicindent}\hspace{\algorithmicindent} $suspicions[p] = true$;
		\State\hspace{\algorithmicindent}\hspace{\algorithmicindent} $proof = \langle suspicion, view \rangle_{priv_a}$;
		\State \hspace{\algorithmicindent}\hspace{\algorithmicindent} \Call{send}{$SUSPICION, view,proof$};	
		\State
		
		\State \textbf{upon} \textit{receive(SUSPICION, $view_i$, proof)} from acceptor $i$ \textbf{do} 
		\State\hspace{\algorithmicindent} \textbf{if} $view_i \neq view$ \textbf{then}
		\State\hspace{\algorithmicindent}\hspace{\algorithmicindent} \textbf{return};
		\State\hspace{\algorithmicindent} \textbf{if} $proof_{pub_i} = \langle suspicion, view \rangle$ \textbf{then}
		\State\hspace{\algorithmicindent}\hspace{\algorithmicindent} $suspicions[i] = proof$;

		\State\hspace{\algorithmicindent} \textbf{if} $\#(suspicions) > f$ and $new\_view[view+1][p] = \bot$ \textbf{then}
		\State\hspace{\algorithmicindent}\hspace{\algorithmicindent} $change\_proof = \langle view\_change, view +1 \rangle_{priv_a}$;
		\State\hspace{\algorithmicindent}\hspace{\algorithmicindent} $new\_view[view+1][p] = change\_proof$;
		\State\hspace{\algorithmicindent}\hspace{\algorithmicindent} \Call{send}{\textit{$VIEW\_CHANGE$, view+1, suspicions, $change\_proof$}};
		\State
		
		\State\textbf{upon} \textit{receive($VIEW\_CHANGE$, $new\_view_i$, suspicions, $change\_proof_i$)} from acceptor $i$ \textbf{do} 
		\State\hspace{\algorithmicindent} \textbf{if} $new\_view_i \leq view$ \textbf{then}
		\State\hspace{\algorithmicindent}\hspace{\algorithmicindent}\textbf{return};
		\State
		\State\hspace{\algorithmicindent} $valid\_proofs = 0$;
		\State\hspace{\algorithmicindent} \textbf{for} $p$ \textbf{in} $acceptors$ \textbf{do} 
		\State\hspace{\algorithmicindent}\hspace{\algorithmicindent} $proof = suspicions[p]$;
		\State\hspace{\algorithmicindent}\hspace{\algorithmicindent} $last\_view = new\_view_i-1$;
		\State\hspace{\algorithmicindent}\hspace{\algorithmicindent} \textbf{if} $proof_{pub_p} = \langle suspicion, last\_view \rangle$ \textbf{then}
		\State\hspace{\algorithmicindent}\hspace{\algorithmicindent}\hspace{\algorithmicindent} $valid\_proofs \mathrel{+{=}} 1$;
		\State
		\State\hspace{\algorithmicindent} \textbf{if} $valid\_proofs \leq f$ \textbf{then}
		\State\hspace{\algorithmicindent}\hspace{\algorithmicindent} \textbf{return};
		\State
		\State\hspace{\algorithmicindent} $new\_view[new\_view_i][i] = change\_proof_i$;
		\State\hspace{\algorithmicindent} \textbf{if} $new\_view[view_i][a] = \bot$ \textbf{then}				
		\State\hspace{\algorithmicindent}\hspace{\algorithmicindent} $change\_proof = \langle view\_change, new\_view_i \rangle_{priv_a}$;
		\State\hspace{\algorithmicindent}\hspace{\algorithmicindent} $new\_view[view_i][a] = change\_proof$;
		\State\hspace{\algorithmicindent}\hspace{\algorithmicindent}  \Call{send}{\textit{$VIEW\_CHANGE$, $view_i$, suspicions, $change\_proof$}};
		\State
		\State\hspace{\algorithmicindent} \textbf{if} $\#(new\_view[new\_view_i]) \geq N-f$ \textbf{then}
		\State\hspace{\algorithmicindent}\hspace{\algorithmicindent} $view = view_i$;
		\State\hspace{\algorithmicindent}\hspace{\algorithmicindent} $leader = view\ mod\ N$;
		\State\hspace{\algorithmicindent}\hspace{\algorithmicindent} $suspicions = \bot$;
		\State\hspace{\algorithmicindent}\hspace{\algorithmicindent} $\Call{send}{LEADER, view, new\_view[view_i]}$ to leader;
	\end{algorithmic}
\end{algorithm}

Next, we present the protocol for each type of ballot in detail.

\subsection{Classic ballots} 

Classic ballots work in a way that is very close to the original Paxos protocol~\cite{Lam98}. Therefore, throughout our description, we will highlight the points where BGP departs from that original protocol, either due to the Byzantine fault model, or due to behaviors that are particular to the specification of Generalized Paxos.

In this part of the protocol, the leader continuously collects proposals by assembling all commands that are received from the proposers since the previous ballot in a sequence. (This differs from classic Paxos, where it suffices to keep a single proposed value that the leader attempts to reach agreement on.)

When the next ballot is triggered, the leader starts the first phase by sending phase $1a$ messages to all acceptors containing just the ballot number. Similarly to classic Paxos, acceptors reply with a phase $1b$ message to the leader, which reports all sequences of commands they voted for. In classic Paxos, acceptors also promise not to participate in lower-numbered ballots, in order to prevent safety violations~\cite{Lam98}.  However, in BGP this promise is already implicit, given (1) there is only one leader per view and it is the only process allowed to propose in a classic ballot and (2) acceptors replying to that message must be in the same view as that leader.

Upon receiving phase $1b$ messages, the leader checks that the commands are authentic by validating command signatures. (This is needed due to the Byzantine model.)  After gathering a quorum of $N-f$ responses, the leader initiates phase $2a$ by sending a message with a proposal to the acceptors (as in the original protocol, but with a quorum size adjusted for the Byzantine model). 
This proposal is constructed by appending the proposals received from the proposers to a sequence that contains every command in the sequences 
that were previously accepted by the acceptors in the quorum (instead of sending a single value with the highest ballot number in the classic specification).

	%\vspace*{-.6cm}
\begin{algorithm} 
	\caption{Byzantine Generalized Paxos - Acceptor a (agreement)}
	\label{BFT-Acc}
	\textbf{Local variables:} $new\_view = \bot,\ leader = \bot,\ view = 0, bal_a = 0,\ val_a = \bot,\ fast\_bal = \bot,\ values=\bot,\ proven = \bot$
		\begin{algorithmic}[1]
			\State \textbf{upon} \textit{receive(P1A, ballot, $view_l$)} from leader $l$ \textbf{do}
			\State \hspace{\algorithmicindent} \textbf{if} $view_l = view$ \textbf{then}
			\State \hspace{\algorithmicindent}\hspace{\algorithmicindent} \Call{phase\_1b}{$ballot$};
			
			\State
			\State \textbf{upon} \textit{receive($FAST,ballot,view_l$)} from leader \textbf{do}
			\State \hspace{\algorithmicindent} \textbf{if} $view_l = view$ \textbf{then}
			\State \hspace{\algorithmicindent}\hspace{\algorithmicindent} $fast\_bal[ballot] = true$;
			
\iffalse	\State
			\State \textbf{upon} \textit{receive(P2B,ballot,value,proof)} from acceptor $i$ \textbf{do}
			\State \hspace{\algorithmicindent} \textbf{if} $proof_{pub_i} \neq \langle ballot, value \rangle$ \textbf{then}
			\State \hspace{\algorithmicindent}\hspace{\algorithmicindent} \textbf{return};
			\State \hspace{\algorithmicindent} $checkpoint[ballot][i] = proof$;
			\State \hspace{\algorithmicindent} \textbf{if} $\#(checkpoint[ballot]) \geq N-f$ \textbf{then}
			\State \hspace{\algorithmicindent}\hspace{\algorithmicindent} $\Call{send}{P2B, ballot, value, checkpoint[ballot]}$ to learners;
			\State \hspace{\algorithmicindent}\hspace{\algorithmicindent} $val_a = \bot$;
\fi
			\State
			\State \textbf{upon} \textit{receive(VERIFY,$view_i$, $ballot_i$,$val_i$,proof)} from acceptor $i$ \textbf{do}
			\State \hspace{\algorithmicindent} \textbf{if} $proof_{pub_i} \neq \langle ballot_i, val_i \rangle$ or $view \neq view_i$ \textbf{then}
			\State \hspace{\algorithmicindent}\hspace{\algorithmicindent} \textbf{return};
			\State
			
			\State \hspace{\algorithmicindent} $proofs[ballot_i][val_i][i] = proof$;
			\State \hspace{\algorithmicindent} \parbox{\linewidth}{\textbf{if} $\#(proofs[ballot_i][val_i]) \geq N-f$ or ($\#(proofs[ballot_i][val_i]) > f$ and \Call{isUniversallyCommutative}{$val_i$}) \textbf{then}}

%			\State \hspace{\algorithmicindent}\hspace{\algorithmicindent} \textbf{if} $!\Call{Contains?}{val_a,val_i}$ \textbf{then}
%			\State \hspace{\algorithmicindent}\hspace{\algorithmicindent} $val_a = val_i$;
			\State \hspace{\algorithmicindent}\hspace{\algorithmicindent} $proven = val_i$;
			\State \hspace{\algorithmicindent}\hspace{\algorithmicindent} $\Call{send}{P2B, ballot_i, val_i, proofs[ballot_i][value_i]}$ to learners;

			\State
			\State \textbf{upon} \textit{receive(P2A\_CLASSIC, ballot, view, value)} from leader \textbf{do}
			\State \hspace{\algorithmicindent} \textbf{if} $view_l = view$ \textbf{then}
			\State \hspace{\algorithmicindent}\hspace{\algorithmicindent} \Call{phase\_2b\_classic}{$ballot, value$}; 
			
			\State		
			\State \textbf{upon} \textit{receive(P2A\_FAST, value)} from proposer \textbf{do}
			\State \hspace{\algorithmicindent} \Call{phase\_2b\_fast}{$value$};
			
			\State
			\Function{phase\_1b}{$ballot$}
			\If {$bal_a < ballot$}
			\State \Call{send}{$P1B, ballot,bal_a,proven, val_a[bal_a], proofs[bal_a]$} to leader;
			\State $bal_a = ballot$;	
			\State $val_a[bal_a] = \bot$;	
			\EndIf
			\EndFunction
			
			\State
			\Function{phase\_2b\_classic}{$ballot, value$}
			\If {$ballot \geq bal_a$ and $val_a = \bot$}
			\State $bal_a = ballot$;
			\State $val_a[bal_a] = value$;
			\State $proof = \langle ballot, val_a \rangle_{priv_a}$;
			\State $proofs[ballot][val_a][a] = proof$;
			\State \Call{send}{\textit{VERIFY, view, ballot, $val_a$, proof}} to acceptors;
			\EndIf
			\EndFunction
			
			\State
			\Function{phase\_2b\_fast}{$ballot, value$}
			\If {$ballot = bal_a$ and $fast\_bal[bal_a]$}
			\State $val_a[bal_a] = val_a[bal_a] \bullet value$;
			\State $proof = \langle ballot, val_a \rangle_{priv_a}$;
			\State $proofs[ballot][val_a][a] = proof$;
			\State \Call{send}{\textit{VERIFY, view, ballot, $val_a$, proof}} to acceptors;
			\EndIf
			\EndFunction
		\end{algorithmic}
\end{algorithm}

The acceptors reply to phase $2a$ messages by sending phase $2b$ messages to the learners, containing the ballot and the proposal from the leader. After receiving $N-f$ votes for a sequence, a learner learns it by extracting the commands that are not contained in his $learned$ sequence and appending them in order. (This differs from the original protocol in the quorum size, due to the fault model, and by the fact that learners would wait for a quorum of matching values, due to the consensus specification.)

\subsection{Fast ballots} 

In contrast to classic ballots, fast ballots leverage the weaker specification of generalized consensus (compared to classic consensus) in terms of command ordering at different replicas, to allow for the faster execution of commands in some cases.\par
The basic idea of fast ballots is that proposers contact the acceptors directly, bypassing the leader, and then the acceptors send directly to the learners their vote for the current sequence, where this
sequence now incorporates the proposed value. Learners then analyze whether conflicts exist, and, if so, revert to using a classic ballot. This is where generalized consensus allows for avoiding falling back to this slow path, namely in the case that the commands that are sequenced in a different order at different acceptors commute.\par
Next, we explain each of these steps in more detail.

\noindent {\bf Step 1: Proposer to acceptors.}
To initiate a fast ballot, the leader informs both proposers and acceptors that the proposals may be sent directly to the acceptors. Unlike classic ballots, where the sequence proposed by the leader consists of the commands received from the proposers appended to previously proposed commands, in a fast ballot, proposals can be sent to the acceptors in the form of either a single command or a sequence to be appended to the command history. These proposals are sent directly from the proposers to the acceptors.\par

\begin{algorithm}
	\caption{Byzantine Generalized Paxos - Learner l}
	\label{BFT-Learn}
	\textbf{Local variables:} $learned = \bot$
		\begin{algorithmic}[1]			
			\State \textbf{upon} \textit{receive($P2B, ballot, value, proofs$)} from acceptor $a$ \textbf{do}
			\State \hspace{\algorithmicindent} $valid\_proofs = 0$;
			\State \hspace{\algorithmicindent} \textbf{for} $i$ \textbf{in} $acceptors$ \textbf{do}
			\State \hspace{\algorithmicindent}\hspace{\algorithmicindent} $proof = proofs[i]$;
			\State \hspace{\algorithmicindent}\hspace{\algorithmicindent} \textbf{if} $proof_{pub_i} = \langle ballot, value \rangle$ \textbf{then}
			\State \hspace{\algorithmicindent}\hspace{\algorithmicindent}\hspace{\algorithmicindent} 
			$valid\_proofs \mathrel{+{=}} 1$;
			\State
			\State \hspace{\algorithmicindent} \parbox{\linewidth}{\textbf{if} $valid\_proofs \geq N-f$ or ($valid\_proofs > f$ and \Call{isUniversallyCommutative}{value}) \textbf{then}}
			\State \hspace{\algorithmicindent}\hspace{\algorithmicindent} $learned = \Call{merge\_sequences}{learned, value}$;
		\end{algorithmic}
\end{algorithm}

\noindent {\bf Step 2: Acceptors to learners.}
Acceptors append the proposals they receive to the proposals they have previously accepted in the current ballot and broadcast the result to the learners. Similarly to what happens in classic ballots, the fast ballot equivalent of the phase $2b$ message, which is sent from acceptors to learners, contains the current ballot number and the command sequence. However, since commands (or sequences of commands) are concurrently proposed, acceptors can receive and vote for non-commutative proposals in different orders. To ensure safety, correct learners must learn non-commutative commands in a total order. To this end, a learner must gather $N-f$ votes for equivalent sequences. That is, sequences do not necessarily have to be equal in order to be learned since commutative commands may be reordered. Recall that a sequence is equivalent to another if it can be transformed into the second one by reordering its elements without changing the order of any pair of non-commutative commands. (Note that, in the pseudocode, equivalent sequences are being treated as belonging to the same index of the \emph{messages} variable, to simplify the presentation.) By requiring $N-f$ votes for a sequence of commands, we ensure that, given two sequences where non-commutative commands are differently ordered, only one sequence will receive enough votes even if $f$ Byzantine acceptors vote for both sequences. Outside the set of (up to) $f$ Byzantine acceptors, the remaining $2f+1$ correct acceptors will only vote for a single sequence, which means there are only enough correct processes to commit one of them. Note that the fact that proposals are sent as extensions to previous sequences is critical to the safety of the protocol. In particular, since the votes from acceptors can be reordered by the network before being delivered to the learners, if these values were single commands it would be impossible to guarantee that non-commutative commands would be learned in a total order. \par
\noindent \textbf{Arbitrating an order after a conflict.} When, in a fast ballot, non-commutative commands are  concurrently proposed, these commands may be incorporated into the sequences of various acceptors in different orders, and therefore the sequences sent by the acceptors in phase $2b$ messages will not be equivalent and will not be learned. In this case, the leader subsequently runs a classic ballot and gathers these unlearned sequences in phase $1b$. Then, the leader will arbitrate a single serialization for every previously proposed command, which it will then send to the acceptors. Therefore, if non-commutative commands are concurrently proposed in a fast ballot, they will be included in the subsequent classic ballot and the learners will learn them in a total order, thus preserving consistency.

\iffalse\section{Byzantine Generalized Paxos Pseudocode} \label{bft_code}
This section presents the pseudocode for the Byzantine Generalized Paxos protocol. \par
\noindent \textbf{Checkpointing} The pseudocode includes a checkpointing feature that allows the leader to propose a special command $C^*$ that causes both the acceptors and learners to discard previously stored commands. This feature can be used to prevent commands from being stored indefinitely. However, since commands are kept at the acceptors to ensure that they will eventually be committed, special care has to be taken before discarding them. To prevent unlearned commands from being discarded, the checkpointing command must be sent within a sequence in a classic ballot. In phase 1 of that classic ballot, acceptors send every command they have to the leader, who waits for $N-f$ phase 1b messages. The leader can be sure that the commands were originated from proposers by verifying the signatures they contain. Since, when proposing to acceptors in fast ballots, proposers wait for acknowledgments from $N-f$ proposers, there's at least one correct acceptor in the intersection of a quorum that received a proposal and a quorum that sends commands to the leader. This means that any proposed value will be sent by some acceptor to the leader and included in the leader's sequence, along with the checkpointing command. Since acceptors must be certain that it's safe to discard previously stored commands, before sending phase 2b messages to learners, they first broadcast these messages among themselves. This round between acceptors is necessary because a Byzantine leader could send a checkpointing command to some acceptors but not others. After waiting for $N-f$ such messages, acceptors send phase 2b messages to the learners along with the cryptographic proofs exchanged in the acceptor-to-acceptor broadcast. After receiving just one message, the leader may simply validate the $N-f$ acceptor proofs contained in it and learn the commands. The learners discard previously stored state when they execute the checkpointing command.
\fi




%\clearpage
\subsection{Correctness Proofs}

This section argues for the correctness of the Byzantine Generalized Paxos protocol in terms of the specified consensus properties.\par


\begin{table}[h!]
	\renewcommand{\arraystretch}{1.5}
	\centering
	\begin{tabularx}{\linewidth}{ |c|X|}
		%\hline
		%\multicolumn{2}{|c|}{Notation}\\
		\hline
		Invariant/Symbol & Definition \\
		\hline
		$\thicksim$ & Equivalence relation between sequences \\
		\hline
		$X \overset{e}{\implies} Y$ & $X$ implies that $Y$ is eventually true \\
		\hline
		$X \sqsubseteq Y$ & The sequence $X$ is either equivalent or a strict prefix of sequence $Y$ according to the equivalence relation $\thicksim$ \\
		\hline
		$\oplus$ & Exclusive OR (XOR) logic operator \\
		\hline
		$\mathcal{L}$ & Set of learner processes \\
		\hline
		$\mathcal{P}$ & Set of proposals (commands or sequences of commands) \\
		\hline
		$learned_{l_i}$ & Learner $l_i$'s $learned$ sequence of commands \\
		\hline
		$learned(l_i,b,s)$ & $learned_{l_i}$ contains sequence $s$ at the end of ballot $b$  \\
		\hline
		$maj\_accepted(b,s)$ & At least $N-f$ acceptors broadcasted their acceptance votes to the learners\\
		\hline
		$proposed(b,s)$ & A proposer proposed $s$ in ballot $b$ \\
		\hline
		
  	\end{tabularx} 
	\caption{Proof notation} 
	\label{table:1}
\end{table}

\subsubsection{Consistency}
\begin{theorem}At any time and for any two learners $l_i$ and $l_j$, $learned_{l_i}$ and $learned_{l_j}$ can subsequently be extended to equivalent sequences
\end{theorem} 
\textbf{Proof:} By Lemma \ref{C-L}
\begin{lemma}
$learned_i \sqsubseteq learned_j \lor learned_j \sqsubseteq  learned_i$ \label{C-L} \par
\end{lemma}
\textbf{Proof:} \par
1. $learned(l_i,b,learned_i) \land learned(l_j,b,learned_j),\ \forall i,j \in \mathcal{L}$ \par
\indent\indent\textbf{Proof:} By definition, $learned(l_i,b,learned_i)$ \par
2. At any moment of a ballot, $learned_{l_i}$ and $learned_{l_j}$ are extendable to equivalent sequences \par
\indent\indent\textbf{Proof:} By 1, if any two learned sequences are equivalent at the end of a ballot then, by definition, at any moment they are extendable to equivalent sequences\par
3. Q.E.D \par

\begin{lemma}
$learned(l_j,b,s) \land learned(l_i,b,s') \implies s \sqsubseteq s' \lor s' \sqsubseteq s$ \par
\end{lemma}
\textbf{Proof:} \par
1. $learned(l_i,b,s) \land learned(l_j,b,s') \implies maj\_accepted(b,s) \land maj\_accepted(b,s')$ \par
\indent\indent\textbf{Proof:} By Lemma \ref{C-L1}.\par
2. $maj\_accepted(b,s)\ \land\ maj\_accepted(b,s') \implies s \sqsubseteq s' \lor s' \sqsubseteq s$ \par
\indent\indent\textbf{Proof:} By Lemma \ref{C-L3}.\par
3. $learned(l_i,b,s)\ \land\ learned(l_j,b,s') \implies s \sqsubseteq s' \lor s' \sqsubseteq s$ \par
\indent\indent\textbf{Proof:} By 1 and 2.\par
3. Q.E.D. \par

\begin{lemma}
$maj\_accepted(b,s) \land maj\_accepted(b,s') \implies s \sqsubseteq s' \lor s' \sqsubseteq s$ \label{C-L3} \par
\end{lemma}
\textbf{Proof:} Proved by contradiction.\par
1. $maj\_accepted(b,s) \land maj\_accepted(b,s') \implies s \not\sqsubseteq s' \land s' \not\sqsubseteq s$ \par
\indent\indent\textbf{Proof:} Contradiction assumption.\par
2. $s$ e $s'$ are assembled concurrently \par
\indent\indent\textbf{Proof:} By Algorithm \ref{BFT-Acc} lines \{27-31\}, if $\not\sqsubseteq s' \land s' \not\sqsubseteq s$ then they must contain non-commutative commands and have been assembled concurrently. \par
3. $maj\_accepted(b,s) \oplus maj\_accepted(b,s')$ \par
\indent\indent\textbf{Proof:} By 2 and Algorithm \ref{BFT-Learn} lines \{1-4\}, only $s$ or $s'$ can gather $N-f$ votes.  \par
4. Q.E.D.

\begin{lemma}
$learned(l_i,b,s) \implies maj\_accepted(b,s),\ \forall\ l_i \in \mathcal{L}$ \label{C-L1} \par
\end{lemma} 
\textbf{Proof:} By Algorithm \ref{BFT-Acc} lines \{20-31\} and Algorithm \ref{BFT-Learn} lines \{1-4\}, if a learner learned a sequence $s$ in a ballot $b$ then a majority of acceptors must have executed phase $2b$ for $s$ in the same ballot.

\subsubsection{Non-Triviality}
\begin{theorem}
$learned_l$ can only contain proposed commands \label{N-T1} \par
\end{theorem} 
\textbf{Proof:} \par
1. $learned(l_i,b,s) \implies maj\_accepted(b,s),\ \forall\ l_i \in \mathcal{L}$ \par
\indent\indent\textbf{Proof:} Lemma \ref{C-L1}. \par
2. $maj\_accepted(b,s) \implies proposed(b,s)$ \par
\indent\indent\textbf{Proof:} Lemma \ref{N-L1}.\par
3. $learned(l_i,b,s) \implies proposed(b,s)$ \par
\indent\indent\textbf{Proof:} By 1 and 2. \par
4. Q.E.D. \par

\begin{lemma}
$maj\_accepted(b,s) \implies proposed(b,s)$ \label{N-L1} \par
\end{lemma}
\textbf{Proof:} By Algorithm \ref{BFT-Acc} lines \{7-11, 20-31\}, for $N-f$ acceptors to accept a proposal it must have been proposed by a proposer.

\subsubsection{Stability}
\begin{theorem}
If $learned_l = v$ then, at all later times, $v \sqsubseteq learned_l$, for any $l$ and $v$ \par \label{S-T1}
\end{theorem} 
\textbf{Proof:} By Algorithm \ref{BFT-Learn} lines \{1-4\}, a learner can only append new commands to its $learned$ command sequence.

\subsubsection{Liveness}
\begin{theorem}
For any proposal $s$ and learner $l$, eventually $learned_l$ contains $s$ \label{L-T1} \par
\end{theorem} 
\textbf{Proof:} By Lemma \ref{L-L1}.
\begin{lemma}
$proposed(s) \overset{e}{\implies} learned(l_i,b,s),\ \forall l_i \in \mathcal{L}$ \label{L-L1} \par
\end{lemma}
\textbf{Proof:} \par
1. $maj\_accepted(b,s) \overset{e}{\implies} learned(l_i,b,s),\ \forall\ l_i \in \mathcal{L}$ \par
\indent\indent\textbf{Proof:} By Lemma \ref{L-L3}.\par
2. $proposed(s) \overset{e}{\implies} maj\_accepted(b,s)$ \par
\indent\indent\textbf{Proof:} By Lemma \ref{L-L2}. \par
3. Q.E.D.

\begin{lemma}
$proposed(s) \overset{e}{\implies} maj\_accepted(b,s)$ \label{L-L2} \par
\end{lemma}
\textbf{Proof:} A proposed sequence is either conflict-free when its incorporated into every acceptor's current sequence or it's creates conflicting sequences at different acceptors. In the first case, it's accepted by a quorum (Algorithm \ref{BFT-Acc} lines \{27-31\}) and, in the second case, it's sent in phase $1b$ messages to the in leader in the next ballot (Algorithm \ref{BFT-Acc} lines \{13-18\}) and incorporated in the next proposal (Algorithm \ref{BFT-Lead} lines \{8-14,27-36\}). 

\begin{lemma}
$maj\_accepted(b,s) \overset{e}{\implies} learned(l_i,b,s),\ \forall\ l_i \in \mathcal{L}$ \label{L-L3} \par
\end{lemma}
\textbf{Proof:} By Algorithm \ref{BFT-Acc} lines \{20-31\} and Algorithm \ref{BFT-Learn} lines \{1-4\}, when a quorum of $N-f$ acceptors accepts a sequence $s$ (or some equivalent sequence) in ballot $b$, eventually $s$ will be learned by any correct learner during ballot $b$.

%
\reftitle{References}
\bibliography{references}

\appendix
\clearpage
\begin{table}[h!]
	\renewcommand{\arraystretch}{1.5}
	\centering
	\begin{tabular}{ |c|c|}
		\hline
		\multicolumn{2}{|c|}{Notation}\\
		\hline
		Symbol & Definition \\
		\hline
		$\sqcup$ & least upper bound \\
		\hline
		$\sqcap$ & greatest lower bound \\
		\hline
		$\bullet$ & append operator \\
		\hline
	\end{tabular} 
	\caption{Notation for the pseudocode} 
	\label{table:1}
\end{table}

\begin{algorithm}[h] 
	\caption{Generalized Paxos - Proposer p}
	\textbf{Local variables:} $ballot\_type = \bot, ballot = 0 $
	\begin{algorithmic}[1]
		
		\State \textbf{upon} \textit{receive(BALLOT, bal, type)} \textbf{do} 
		\State \hspace{\algorithmicindent} 
		$ballot = bal$;
		\State \hspace{\algorithmicindent} 
		$ballot\_type = type$;
		\State
		
		\State \textbf{upon} \textit{command\_request(c)} \textbf{do}   \hspace{\algorithmicindent}\hspace{\algorithmicindent}\hspace{\algorithmicindent}\# receive request from application
		\State \hspace{\algorithmicindent} \textbf{if} $ballot\_type = fast\_ballot$ \textbf{then}

		\State \hspace{\algorithmicindent}\hspace{\algorithmicindent} \Call{send}{$P2A\_FAST, ballot, c$} to acceptors;
		\State \hspace{\algorithmicindent} \textbf{else} 
		\State \hspace{\algorithmicindent}\hspace{\algorithmicindent} \Call{send}{\textit{PROPOSE, c}} to leader;		
	\end{algorithmic}
\end{algorithm}

\begin{algorithm}
\caption{Visigoth Generalized Paxos - Process p}
\begin{algorithmic}[1]
	
	\Function{merge\_sequences}{$old\_seq, new\_seq$}
	\State \textbf{for} $c$ \textbf{in} $new\_seq$ \textbf{do} 
	\State \hspace{\algorithmicindent} \textbf{if} $!\Call{contains}{old\_seq,c}$ \textbf{then}
	\State \hspace{\algorithmicindent}\hspace{\algorithmicindent}\hspace{\algorithmicindent} $old\_seq =  old\_seq \bullet c$;
	\State \textbf{end for}
	\State \textbf{return} $old\_seq$;
	\EndFunction
\end{algorithmic}
\end{algorithm}

\begin{algorithm} 
\caption{Visigoth Generalized Paxos - Leader l}
\textbf{Local variables:} $ballot_l = 0,maxTried_l = \bot,proposals = \bot, accepted = \bot$
\begin{algorithmic}[1]
	\State \textbf{upon} \textit{trigger\_next\_ballot(type)} \textbf{do}
	\State \hspace{\algorithmicindent} $ballot_l \mathrel{+{=}} 1$;
	\State \hspace{\algorithmicindent} \Call{send}{$BALLOT,ballot_l,type}$ to proposers;
	\State
	\State \hspace{\algorithmicindent} \textbf{if} $type = fast$ \textbf{then}
	\State \hspace{\algorithmicindent}\hspace{\algorithmicindent} \Call{send}{$FAST,ballot_l,view}$ to acceptors;
	\State \hspace{\algorithmicindent} \textbf{else}
	\State \hspace{\algorithmicindent}\hspace{\algorithmicindent} \Call{send}{$P1A, ballot_l, view$} to acceptors;
	
	\State
	\State \textbf{upon} \textit{receive(PROPOSE, prop)} from proposer $p_i$ \textbf{do} 
	\State \hspace{\algorithmicindent} $proposals = proposals \bullet prop$;
	\State
	\State \textbf{upon} \textit{receive($P1B, bal_a,vals_a$)} from acceptor $a$ \textbf{do}
	\State \hspace{\algorithmicindent} \textbf{if} $bal_a = ballot_l$ \textbf{then}
	\State \hspace{\algorithmicindent}\hspace{\algorithmicindent} $accepted[ballot_l][a] = vals_a$;
	\State \hspace{\algorithmicindent}\hspace{\algorithmicindent} \textbf{if} $\#(accepted[ballot_l]) \geq N-f$ \textbf{then} 
	\State \hspace{\algorithmicindent}\hspace{\algorithmicindent}\hspace{\algorithmicindent} \Call{phase\_2a}{$ $};
	
	\State
	\Function{phase\_2a}{$ $}
	\State $maxTried_l = \Call{proved\_safe}{ballot_l}$;
	\State $maxTried_l = maxTried_l \bullet proposals$;
	\State \Call{send}{$P2A\_CLASSIC,view,ballot_l, maxTried_l$} to acceptors;
	\State $proposals = \bot$;
	\EndFunction
	
	\State
	\Function{proved\_safe}{$ballot$}
	\State $safe\_seq = \bot$;
	\State \textbf{for} $seq$ \textbf{in} $accepted[ballot]$ \textbf{do}
	\State \hspace{\algorithmicindent} $safe\_seq = \Call{merge\_sequences}{safe\_seq, seq}$;
	\State \textbf{end for}
	\State \textbf{return} $safe\_seq$;
	\EndFunction	
\end{algorithmic}
\end{algorithm}

\begin{algorithm} 
	\caption{Visigoth Generalized Paxos - Acceptor a}
	\label{VFT-Acc}
	\textbf{Local variables:} $leader = \bot, bal_a = 0,val_a = \bot,fast\_bal = \bot$
	\begin{algorithmic}[1]
		\State \textbf{upon} \textit{receive(P1A, ballot)} from leader \textbf{do}
		\State \hspace{\algorithmicindent} \Call{phase\_1b}{$ballot$};
		
		\State
		\State \textbf{upon} \textit{receive($FAST,ballot$)} from leader \textbf{do}
		\State \hspace{\algorithmicindent} $fast\_bal[ballot] = true$;
		
		\State
		\State \textbf{upon} \textit{receive(P2A\_CLASSIC, ballot, value)} from leader \textbf{do}
		\State \hspace{\algorithmicindent} \Call{phase\_2b\_classic}{$ballot, value$}; 
		
		\State		
		\State \textbf{upon} \textit{receive(P2A\_FAST,ballot,value,proof)} from proposer p \textbf{do}
		\State \hspace{\algorithmicindent} \Call{phase\_2b\_fast}{$ballot, value$};

		\State
		\Function{phase\_1b}{$ballot$}
		\If {$bal_a < ballot$}
		\State \Call{send}{$P1B, ballot, val_a$} to leader;
		\State $bal_a = ballot$;	
		\State $val_a[bal_a] = \bot$;	
		\EndIf
		\EndFunction
		
		\State
		\Function{phase\_2b\_classic}{$ballot, value$}
		\If {$ballot \geq bal_a$ and $val_a = \bot$}
		\State $bal_a = ballot$;
		\State $val_a[ballot] = value$;
		\State \Call{send}{$P2B, ballot, value$} to learners;

		\EndIf
		\EndFunction
		
		\State
		\Function{phase\_2b\_fast}{$ballot, value$}
		\If {$ballot = bal_a$ and $fast\_bal[bal_a]$}
		\State $val_a[bal_a] =  \Call{merge\_sequences}{val_a[bal_a], value}$;
		\State \Call{send}{$P2B, bal_a, val_a[bal_a]$} to learners;
		\EndIf
		\EndFunction
	\end{algorithmic}
\end{algorithm}

\begin{algorithm}
	\caption{Generalized Paxos - Learner l}
	\textbf{Local variables: } $learned = \bot, messages = \bot$ 
	\begin{algorithmic}[1]
		\State \textbf{upon} \textit{receive($p2b, bal, val$)} from acceptor $a_i$ \textbf{do}
		\State \hspace{\algorithmicindent} $messages[bal][val][a_i] = true$;
		\State \hspace{\algorithmicindent} \textbf{if} $\#(messages[bal][val]) \geq N-f$ \textbf{then}
		\State \hspace{\algorithmicindent} \hspace{\algorithmicindent} \hspace{\algorithmicindent} $learned = \Call{merge\_sequences}{learned, val}$;
	\end{algorithmic}
\end{algorithm}



\end{document}

