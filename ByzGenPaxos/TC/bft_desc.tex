%\subsection{Description}
This section presents our Byzantine fault tolerant Generalized Paxos
Protocol. Given our space constraints, we opted for merging in a
single description a novel presentation of Generalized Paxos and its
extension to the Byzantine model.
\subsection{Overview}

We modularize our protocol explanation according to the following main components, which are also present in other protocols of the Paxos family:

\begin{itemize}

\item
{\bf View change} -- The protocol that continuously replaces leaders, until one is found that can ensure progress (i.e., commands are eventually executed).

\item
{\bf Agreement} -- Given a fixed leader, this protocol extends the current sequence with a new command or set of commands. Analogously to Fast Paxos~\cite{L06} and Generalized Paxos~\cite{Lamport2005}, choosing this extension can be done using two different sub-protocols: using either {\bf classic ballots} or {\bf fast ballots}, with the characteristic that fast ballots complete in fewer communication steps, but may have to fall back to using a classic ballot when there is contention among concurrent requests.

\end{itemize}

\subsection{View change} 

The goal of the view change sub-protocol is to elect a leader that is able to carry through the agreement protocol, i.e., that enables proposed commands to eventually be learned by all the learners.

To allow this, whenever a subset of correct processes perceive that liveness is not being upheld, this must be sufficient for triggering a view change. However, even if up to $f$ Byzantine processes produce false suspicions, this should not suffice to trigger a view change. Therefore, this mechanism must ensure that when $f+1$ processes suspect the leader to be faulty (and only then), all correct processes commit to a new view and stop participating in lower-numbered views. In particular, processes start broadcasting suspicion messages if they believe that the leader is faulty. Suspicions contain the current view number, and are signed by the sending process, to avoid being forged. Furthermore, to ensure that view changes succeed if they have enough support but cannot be triggered by Byzantine processes, if a correct process receives $f+1$ suspicions, it stops participating in the current view and multicasts a view-change message. View-change messages contain a ballot number, the previously committed ballot and value, and the $f+1$ signed suspicions. If a process receives a view-change message without previously receiving $f+1$ suspicions, it will verify the suspicions by validating they are signed by $f+1$ distinct processes, commit to the new view and multicast a view-change message. (As such, the signatures allow a process that receives this message to commit to the new view and multicast its own view-change messages without receiving $f+1$ suspicions itself.)  This guarantees that if one correct process receives the $f+1$ suspicions and broadcasts the view-change message, then all correct processes, upon receving this message, will be able to validate the proof of $f+1$ suspicions and switch to the new view. Processes start participating in the new view as soon as they broadcast their view-change messages, since the multicast of one correct view-change message is all it takes to ensure that all correct process will eventually switch to the new view. The only process that must wait for $2f+1$ view-change messages is the leader of the new view since, to send new sequences, it needs to know the values voted for in the last ballot for a quorum of acceptors, as described next.


\subsection{Agreement protocol} 

The consensus protocol allows processes to agree on equivalent sequences of commands (according to our previous definition of equivalence). 
In the original Paxos, each instance of consensus is called a ballot, which can be \textit{classic} or \textit{fast}. 
In this algorithm, ballots are a similar concept, with the exception that instead of being a separate instance of consensus, 
it corresponds to an extension to the sequence of learned commands of a common consensus instance.

In classic ballots, a leader proposes a single sequence of commands, which is learned by the learners. 
A classic ballot in Generalized Paxos follows a protocol that is very similar to the one used by classic Paxos~\cite{Lam98}. This protocol comprises a first phase where the leader obtains a promise that the acceptors will not participate in lower-numbered ballots, and each acceptor conveys to the leader the sequences that the acceptor has already voted for. This first phase becomes unnecessary when a classic ballot is executed after a view change, since the view change protocol can include this first phase, i.e., it can gather previously committed sequences and extract that same promise from the acceptors. In this case, the aforementioned promise is implicitly valid because, for the remainder of the current view, the leader is the only process that can start a new ballot. This is followed by a second phase where the leader picks an extension to the sequence of previously proposed commands and broadcasts it to the acceptors. The acceptors send their votes to the learners, who, after receiving $N-f$ votes for a proposal, learn the sequences by appending the new commands to their own sequences of learned commands.

In fast ballots, multiple proposers can concurrently propose either single commands or sequences of commands (by sending them directly to the acceptors), and eventually a set of equivalent sequences must be learned by the learners. We use the term \textit{proposal} to denote either the command or sequence of commands that was proposed.
Concurrency implies that acceptors may send votes in different orders but the protocol ensures that, as long as the resulting sequences are equivalent, the commands will be learned in two message delays. 

Note that there is a subtle interplay between the two types of ballots, since the leader must execute the first phase if he wishes to run a classic ballot after a fast ballot. This is because the acceptors may have used a fast round to vote for new sequences that are unknown to the leader, since they were proposed directly by the proposers.

Next, we present the protocol for each type of ballot in detail.

\subsection{Classic ballots} 

In a classic ballot, proposers send their proposed commands to the leader who sends a single sequence of commands to the acceptors. This sequence is built by appending the new proposals to a sequence containing every proposal of the previous ballot. The leader starts by sending phase $1a$ messages to all acceptors, who respond with the last ballot in which they voted, and every sequence of commands they voted for. We can ensure that the commands were proposed by valid proposers, instead of being created by faulty acceptors, by validating command signatures. After gathering $N-f$ responses, the leader initiates phase $2a$ by sending a message with a proposal to the acceptors. This proposal is assembled by appending the proposers' sequences to a sequence that contains every command in the acceptors' previously accepted sequences. The acceptors send phase $2b$ messages to the learners, containing the ballot and the leader's proposal. After receiving $N-f$ votes for a sequence, a learner learns it by extracting the commands that aren't contained in his $learned$ sequence and appending them in order.

\subsection{Fast ballots} 
To initiate a fast ballot, the leader informs both proposers and acceptors that the proposals may be sent directly to the acceptors. Unlike classic ballots, where the sequence proposed by the leader consists of the commands received from the proposers appended to previously proposed commands, in a fast ballot, proposals can be sent to the acceptors in the form of either a single command or a sequence to be appended to the command history. These proposals are sent directly from the proposers to the acceptors, who then append them to the proposals they have previously accepted in the current ballot and broadcast the result to the learners. Like in the classic ballot, the phase $2b$ message sent to the learners contains the current ballot number and the command sequence. To learn a sequence, the learner must gather $N-f$ votes for equivalent sequences. That is, sequences don't necessarily have to be equal to be learned since commutative commands may be reordered. Recall that a sequence is equivalent to another if it can be transformed into the second one by reordering its elements without changing the order of any pair of non-commutative commands. (Note that, in the pseudocode, equivalent sequences are being treated as belonging to the same index of the \emph{messages} variable, to simplify the presentation.) \par
\textbf{Leader Value Picking.} Sequences sent by the acceptors in phase $1b$ messages may contain commands that were not learned. This is likely if, in a fast ballot, non-commutative commands are  concurrently proposed by proposers. In this situation, these commands may be incorporated into the sequences of various acceptors in different orders, and therefore the sequences sent by the acceptors in phase $2b$ messages will not be equivalent and will not be learned. However, after the leader gathers these unlearned sequences in phase $1b$, it will assemble a single serialization for every previously proposed command, which it will then send to the acceptors. Therefore, if non-commutative commands are concurrently proposed in a fast ballot, they will be included in the subsequent classic ballot and the learners will learn them in a total order, preserving consistency.
