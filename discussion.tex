\section{Discussion}
% and Concluding remarks}
\label{sec:disc}

\subsection{Extending the protocol to universally commutative commands}

% Generalized Paxos vs Paxos spec.
The Generalized Paxos specification dictates that the only commands
whose relative order can be reversed at different learners are those
that commute with each other. This restriction has the advantage of
allowing for running a replicated state machine on top of the
Generalized Paxos protocol, while still ensuring that the state
machine behaves in a way that is indistinguishable from running it on
top of the original Paxos protocol.

% Extension to ops that commute with all
One of the downsides of the Generalized Paxos protocol (in either
fault model) is that this commutativity check is done at runtime,
and, if two non-commutative operations are issued concurrently in
the same ballot, then we must fall back to the slower
classic Paxos protocol.

This raises the possibility of extending the protocol to handle
commands that are universally commutative, i.e., commute with every
other commands. For these commands, it is known before executing them
that they will not generate any conflicts, and therefore it is not
necessary to check them against concurrently executing commands.  This
allows us to optimize the protocol by decreasing the number of phase
$2b$ messages required to learn to a smaller $f+1$ quorum. Since, by
definition, these sequences are guaranteed to never produce conflicts,
the $N-f$ quorum is not required to prevent learners from learning
conflicting sequences. Instead, a quorum of $f+1$ is sufficient for
the learner to be sure that the proposed sequence was proposed by a
correct proposer.

% Weakly consistent replication - also diff order but ok to have different v

\subsection{Generalized Paxos and weak consistency}


The Byzantine Generalized Paxos protocol tackles two challenges in two different avenues of research, fault tolerance and relaxed consistency models. By specifying the generalized consensus problem, Generalized Paxos~\cite{Lamport2005} allows learners to learn concurrent proposals in different orders when the proposals commute. This idea can be related to models like causal consistency~\cite{Ahamad1995} and newer variable consistency models like RedBlue~\cite{Li2012} that attempt to decrease latency costs by reducing coordination requirements between replicas. 



% Extension to diff replica grps


\subsection{Handling faults in the fast case}

Generalized Paxos reduces coordination requirements by allowing proposers to propose directly to the acceptors, thus reducing the number of communication steps to two. However, to preserve the consistency property, for $f$ classic faults and $e$ fast-learning faults, the total system size $N$ must uphold two conditions:
\begin{itemize}
	\item $N > 2f$
	\item $N > 2e+f$
\end{itemize} 
Additionally, the fast and classic quorums are, respectively, $N-e$ and $N-f$ which means there is a dependence relationship between these three concepts: number of tolerated faults, system size and quorum size. If we wish to keep $e$ small, then the fast ballot quorum will be a larger amount of the total system size which itself will be smaller. If we wish to allow fast ballots to progress in the presence of a higher number of faults, $e=f$ which means $N >3f$ and $N-e=N-f$. However, we intend to provide Byzantine fault tolerance which also requires a total system size of $N>3f$ and a quorum size of $2f+1$. Due to this observation, we manage to amortize the cost of both features by using a larger system size to achieve both at the cost of one. \par

%\textbf{Optimizations} One possible optimization of the Byzantine Generalized Paxos protocol leverages universally commutative commands or sequences of commands, which we define as sequences which commute with any other. Universally commutative sequences allows us to reduce latency by decreasing the number of phase $2b$ messages required to learn to a smaller $f+1$ quorum. Since, by definition, these sequences are guaranteed to never produce conflicts, the $N-f$ quorum isn't required to prevent learners from learning conflicting sequences. Instead, a $f+1$ quorum is sufficient for the learner to be sure that the proposed sequence was proposed by a correct proposer. 


\section{Conclusion}

In this paper, we presented a simplified description of the Generalized Paxos protocol and specification, which is meant to pave the way for new avenues of research in this area. In addition, we present a Byzantine fault tolerant version of the protocol, and we prove the correctness of this protocol. In the future, we would like to implement this protocol and fully evaluate it, in addition to gaining a better understanding of its practical applicability.
