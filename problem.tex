\chapter{Consensus Problem}

As previously mentioned, one of our most important contributions is the revised consensus problem. In Lamport's original specification, the consensus problem is generalized from agreeing on a single value to agreeing on an increasing set of values~\cite{Lamport2005}. The consensus problem is defined in terms of \textit{c-structs} and includes four properties, where $learned[l]$ is the sequence of commands for a learner $l$, $propCmd$ is the set of all proposed commands and $Str(propCmd)$ is the set of all \textit{c-structs} that can be constructed from the elements in $propCmd$:
\begin{itemize}
	\item \textbf{Non-Triviality} $learned[l] \in Str(propCmd)$ always holds, for every learner $l$.
	\item \textbf{Stability} It is always the case that $learned[l] = v$ implies $v \sqsubseteq learned[l]$ at all later times, for any learner $l$ and \textit{c-struct} $v$.
	\item \textbf{Consistency} $learned[l_1]$ and $learned[l_2]$ are always compatible, for all learners $l_1$ and $l_2$.
	\item \textbf{Liveness} If $c \in propCmd$ then eventually $learned[l]$ contains $c$.
\end{itemize}

\par
The reason why the generalized consensus problem and its solution, the Generalized Paxos protocol, are so hard to understand, is due to the amount of mathematical operations defined and how they're used to compose a consensus-solving algorithm. In the following sections, we will step through these mathematical formalisms and provide the intuition as to how they're used in the Generalized Paxos protocol. Once that is clear, we will introduce our simplified consensus problem and compare the two.

\section{Operations on \textit{c-structs}} \label{Operations}
The consistency property stated above defines what a consensus-solving protocol must guarantee to be considered \textit{safe}. In generalized consensus, consistency relies on a notion of \textit{compatibility} between \textit{c-structs}. As Lamport states, two \textit{c-structs}, $v$ and $w$ are compatible if and only if they have a common upper bound, that is, if there exists a \textit{c-struct} $z$ to which they are both prefixes, $v \sqsubseteq z \land w \sqsubseteq z$. Intuitively, this means that if two learned \textit{c-structs} can be extended to the same \textit{c-struct}, then they it is safe to learn them. However, if two \textit{c-structs} contain conflicting commands in different relative orders, then there is no \textit{c-struct} that extends both and the \textit{c-structs} are cannot be safely learned.\par
Upper and lower bounds are mostly used in the Generalized Paxos protocol in the form of least upper bounds ($\sqcup$) and greater lower bounds ($\sqcap$). A least upper bound $\sqcup$ of a set of \textit{c-structs} $T$  is an upper bound $v$ of $T$ such that $v \sqsubseteq w$ for any upper bound $w$ in $T$. One could define the $\sqcup T$, as the upper bound of $T$ that is a lower bound of every upper bound in $T$. Note that if $v = w$ then $v\sqsubseteq w$ and $w\sqsubseteq v$. Analogously, a greater lower bound $\sqcap$ of a set of \textit{c-structs} $T$ is a lower bound $v$ of $T$ such that $w \sqsubseteq v$ for any lower bound $w$ in $T$. \par
Let $T$ be the set of \textit{c-structs} $\{A ; A \bullet B ; A \bullet B \bullet C\}$ and let the set of all \textit{c-structs} be any combination of the commands $A$, $B$, $C$, $D$ and $E$, as long as they appear in the alphabetical order. If an upper bound of $T$ is a \textit{c-struct} $v$ such that $w \sqsubseteq v, \forall w \in T$, then there are three possible upper bounds, namely $A \bullet B \bullet C$, $A \bullet B \bullet C \bullet D$ and $A \bullet B \bullet C \bullet D \bullet E$. Of these three upper bounds, we can compute $\sqcup T$ by noting that only $A \bullet B \bullet C$ is a lower bound of all three upper bounds. As previously mentioned, the consistency property requires learned \textit{c-structs} to be compatible (i.e., to have a common upper bound). Any set of compatible \textit{c-structs} is guaranteed to have a lower upper bound. \par 
Let's redefine the set $T$ to contain the following \textit{c-structs} $\{A ; A \bullet C ; A \bullet B \bullet C\}$. This may seem like an innocuous modification but it's enough to make the computation of $\sqcup T$ impossible. Taking into account the same upper bound, there is no upper bound that to which every \textit{c-struct} in $T$ is a prefix. No upper bound of $A \bullet B \bullet C$ will also be an upper bound of $A \bullet C$. In other words, $A \bullet B \bullet C$ and $A \bullet C$ are not compatible.\par 
These concepts can be applied to any set of \textit{c-structs}. However, the generalized consensus problem gets its name from the fact that it allows for the definition of any set of rules on what can is a valid \textit{c-struct}. We already gave one example of this since when we defined a valid \textit{c-struct} to be one where the commands are alphabetically ordered. {\color{red} This generality carries some complications since the way (...) RETHINK}

\section{\textit{C-structs} in Generalized Paxos}
The operations defined in section \ref{Operations} are significant 

