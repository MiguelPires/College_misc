%\subsection{Description}
This section presents our Visigoth fault tolerant Generalized Paxos protocol. 
\subsection{Overview}

The Visigoth model differs from its Byzantine counterpart by allowing $s$ processes to be slow but correct \cite{}. A process $i$ is defined to be slow with respect to $j$ if messages from $i$ to $j$ (or vice-versa) take more than $T$ time units to be transmitted. This assumption allows us to gather more efficient quorums by leveraging the knowledge that only $s$ processes can take more than $T$ time units to send a message. In VFT's Quorum Gathering Primitive (QGP), a process gathers a quorum by waiting for messages from $N-s$ distinct processes. After $T$ time units have passed, of the $x$ processes that are unresponsive, only $s$ of them may be slow, which means that $x-s$ processes must be faulty. This allows us to leverage the assumption of $s$ slow but correct processes to decrease the quorum size to $N-u$ while still guaranteeing the intersection properties necessary for safety. We make use of this mechanism to adapt our Byzantine Generalized Paxos protocol to the Visigoth fault model. This includes changing the quorum gathering in phase 1b, where acceptors relay their previous votes to the leader, and phase 2b, where acceptors send their votes to the learners. \par
In Byzantine Generalized Paxos, a Byzantine leader can at most prevent progress until a new leader is elected. Even if a leader causes a split vote by sending different values to some acceptors in its phase 2a messages, at most one of those values can obtain the $N-f$ votes require to be learned. However, since in the Visigoth model we want to take advantage of the additional assumptions to reduce the quorum to $N-u$, its possible for the leader to employ a split vote to send two different values to two sets of acceptors and ignore others such that the ignored acceptors are more than $s$ and the timeout is reached, causing the required quorum size to be reduced. Since $o$ of the remaining acceptors can be Byzantine and vote for both values and the leader can employ split vote between the remaining $u+s+1$ (including the acceptors that were previously ignored), there are enough votes for both values to be committed, violating the safety property. One configuration where this would happen would be with $u=o=2, s=1, N=u+o+\min(u,s)+1=6$. In this system the initial quorum is $N-s=5$ and the reduced quorum is $N-u=4$. If the leader sends $v_1$ to two acceptors, $v_2$ to other two and ignores the last two, then the timeout is reached and the required quorum size is reduced from 5 to 4. Each value has two votes from both of the Byzantine acceptors and one vote from one of the two acceptors that received the split vote. Since the previously ignored acceptors are correct, the leader can employ another split vote to divide them between $v_1$ and $v_2$ to achieve four votes for both values. \par
To prevent this situation, when sending phase 2b messages to learners, acceptors also resend the leader's phase 2a message to other acceptors. This prevents the leader from maliciously ignoring acceptors to force the quorum to decrease. To ensure that this replayed message originated from the leader and not from a Byzantine acceptor, it includes the leader's signature. This modification implies that, when gathering a quorum, the timeout must be increased from 2T to 3T since there are up to three message delays until a message is received by a learner.
