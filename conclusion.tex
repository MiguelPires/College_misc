\chapter{Conclusion} \label{conclusion}
This dissertation focuses on the problem of achieving consensus in a distributed system and its solutions. In particular, this work builds upon one of the protocols of the widely known Paxos family of consensus-solving protocols, Generalized Paxos~\cite{Lamport2005}. This protocol makes use of a generic consensus problem that uses special structures, called \textit{c-structs}, to allow for the definition of any consensus problem. By solving this generalized consensus problem, Generalized Paxos is able to present a solution that covers a broad spectrum of problems. One of these problems is the command history problem which makes the observation that commands don't necessarily conflict with each other and may result in the same state being produced regardless of the order in which they are executed. This problem allows coordination requirements to be reduced since not every command is required to be committed in a total order. However, the universality of the generalized consensus specification comes at the cost of a very complex description of both generalized consensus and Generalized Paxos. With this in mind, one of the initial goals of this work was to produce a specification that still took advantage of the command history problem but didn't carry with it all the generality that made the original problem and protocol so complex. \par
The Paxos protocol family also includes variants that reduce the number of message steps required to learn a value. The aforementioned Generalized Paxos protocol is one of these protocols that enable the system to learn commands in just two message steps in the common case instead of the usual four. This fastness relative to the complexity of the protocol's message pattern comes at the cost of increasing the system's quorum sizes and total number of processes in function of the number of faults that we wish to tolerate while still allowing two-step executions. This work also intended to retain this property whenever doing so didn't imply an unreasonable cost in another aspect of the protocol. \par
Another facet of the distributed systems literature focuses on the Byzantine fault model, in which protocols are designed to withstand not only crash faults but also arbitrary and possibly malicious behavior by some of the system's participants. Several works make their contributions in this field by extending consensus-solving algorithms to the Byzantine model. However, few works have aligned this goal with the previously stated ones, namely allowing fast executions and taking advantage of the commutativity assumption to reduce coordination requirements. This is a sparse area in the distributed protocols literature to which we wanted to contribute.\par
Finally, newer non-crash fault models make the observation that the Byzantine model is very pessimistic in that it allows for up to $f$ processes, in a system of at least $3f+1$, to display malicious behavior in a coordinated way. In order to reduce the stringent system requirements imposed by the strong faulty behavior permitted to the processes, models like \acrshort{vft}~\cite{Porto2015} and \acrshort{xft}~\cite{Liu2015} have introduced specifications that allow the system to tolerate some arbitrary behavior with a lesser cost. VFT also allows the network administrator to parameterize the allowed behavior both in terms of synchronism and fault tolerance. These models are specially suited for datacenter environments where not only is the network highly secured and monitored, making coordinated malicious behavior unlikely, but also the high amount of required equipment makes protocols with less requirements more attractive. Protocols can take advantage of the more realistic assumptions provided by these models to become more viable options in scenarios like the one that was just mentioned. Therefore, we also wanted to build a version of the protocol that took advantage of these newer models to be better suited for datacenter environments and perhaps scenarios in which systems are geo-replicated. \par

\section{Achievements}
In line with the previously mentioned goals, this thesis proposes three protocols, each targeting a different fault model. The contributions are added incrementally such that each protocol contains contributions introduced in previous ones, when applicable. \par
The first protocol targets the \acrshort{cft} model and its major contribution is the simplification of the generalized consensus problem to a problem that still takes advantage of the commutativity observation without generalized consensus' increased complexity. By specializing generalized consensus into the problem of agreeing on commands histories, the specification of the protocol also becomes easier to understand and less reliant on complicated mathematical formalisms. This is an important contribution since it increases the understandability of both the problem and its solution, making it easier to translate both to an actual implementation. This protocol is described extensively both textually and through pseudocode. We also describe an optimization that allows certain commands to be committed with a reduced quorum. This extension further extends the protocol's applicability to scenarios such as datacenter environments.\par
Our second major contribution is the adaptation of Generalized Paxos to the Byzantine fault model. In order to build this protocol, we propose a specification of consensus that builds upon our previous simplified consensus problem by taking into account the possibility of Byzantine behavior. With this problem in mind, we describe \acrlong{bgp}, providing both a textual description and also pseudocode that can help the mapping of the protocol into a practical implementation. Two extensions are proposed to this protocol. The first describes how the protocol can take advantage of universally commutative commands to both commit a sequence using a reduced quorum and also bypass the additional verification phase imposed by the possibility of Byzantine faults. The second extension is a checkpointing feature that allows the protocol to deal with the accumulation of state at both the learners and the acceptors. This extension increases the protocol's applicability in practical scenarios since it allows the system to free up memory space by executing the checkpointing subprotocol when resources demand it. To further convince the reader of this protocol's validity, we also present proofs that argue that the protocol correctly ensures each of the properties described in the consensus specification.\par
The third protocol, \acrlong{vgp}, and the final contribution of this thesis is the adaptation of \acrshort{bgp} to the Visigoth fault model. We believe \acrshort{vft} to be a good fit for Generalized Paxos since VFT's synchrony assumptions and Generalized Paxos commutativity observation have the potential to greatly reduce coordination requirements within a controlled system such as a datacenter. The overall message pattern of \acrshort{vgp} is similar to that of its Byzantine counterpart. However, VFT's synchrony assumptions and their effects on quorum sizes have non-obvious consequences on the resolution of the command history problem. For this contribution, the description focuses mostly on the problems introduced by VFT's assumptions and how the protocol is modified to deal with them. The protocol is accompanied by a pseudocode description and correctness proofs, both structured similarly to their analogous versions in \acrshort{bgp}. \par
A final minor contribution that precedes the previously mentioned protocols and consensus problems can be found in Chapter \ref{problem}. This chapter attempts to describe some important but complicated aspects of the original generalized consensus and Generalized Paxos description. This is intended to help familiarize the reader with the complexities of the works to which our contributions try to add to and may help shed some light on some of Generalized Paxos' more opaque components.

\section{Future work}

The contributions proposed in this dissertation can be augmented through several extensions which are mostly aligned with the goal of making the described protocols more attractive in real-world systems. We propose that this can be done either by showcasing its execution in tandem with proven implementations or by reducing our protocols' costs and assumptions.\par
One of the directions in which this work could be extended would be that of integrating either \acrshort{bgp} or our \acrshort{cft} version of Generalized Paxos with a proven implementation such as Zookeeper~\cite{Hunt2010}. This would demonstrate \acrshort{bgp}'s validity in a realistic environment and also point out any shortcomings that might constrain its ability to handle real workloads in terms of either throughput or latency. \par
One particularly interesting application of our \acrshort{cft} consensus protocol would be in the contest of distributed transactions. The reason why this is a noteworthy domain that might benefit from an integration with our protocol is because it would allow clients to safely replicate transactions throughout a distributed system while taking advantage of the amount of commutativity allowed by each transaction to improve performance. Transactions with commands that could cause conflicts with others would possibly require a classic ballot to be committed while transactions composed of commands with relaxed consistency requirements (e.g., queries to the system's state that don't require a precise response) would allow for greater concurrency, lowering latency and the total number of messages exchanged. Furthermore, in systems where non-commutative concurrency is less frequent, most transactions could be committed using fast ballots in just two message steps. \par
Another interesting direction in which our work could be extended would be switching our network model's point-to-point authentication from digital signatures to vectors of \acrfull{macs}. This transformation would lower the cost imposed by an implementation of \acrshort{bgp} and is certain to be feasible since a generic implementation of this scheme has been been proposed~\cite{Aiyer:2008}.