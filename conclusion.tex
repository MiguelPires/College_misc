\chapter{Conclusion} \label{conclusion}
This dissertation focuses on the problem of achieving consensus in a distributed system and its solutions. In particular, this work builds upon one of the protocols of the widely known Paxos family of consensus-solving protocols, Generalized Paxos~\cite{Lamport2005}. This protocol makes use of a generic consensus problem that uses special structures, called \textit{c-structs}, to allow for the definition of any consensus problem. By solving this generalized consensus problem, Generalized Paxos is able to present a solution that covers a broad spectrum of problems. One of which was the command history problem that makes the observation that commands don't necessarily conflict with each other and may result in the same state being produced regardless of the order in which they are executed. This problem allows coordination requirements to be reduced since not every commands is required to be committed in a total order. However, the generality of this consensus problem comes at the cost of a very complex specification of both generalized consensus and Generalized Paxos. With this in mind, of the initial goals of this work was to produce a specification that still took advantage of the command history problem but didn't carry with it all the generality that made the original problem and protocol so complex. \par
The Paxos protocol family also includes variants that reduce the number of message steps required to learn a value. The aforementioned Generalized Paxos protocol is one of these protocols that enable the system to learn commands in just two message steps instead of the usual four. This fastness relative to the protocol's broadcast rounds comes at the cost of increasing the system's quorum sizes and total number of processes in function of the number of faults that we wish to tolerate while still allowing two-step executions. This work also intended to retain this property whenever doing so didn't imply an unreasonable cost in another aspect of the protocol. \par
Another facet of the distributed systems literature focuses on the Byzantine fault model, in which protocols are designed to withstand not only crash faults but also arbitrary and possibly malicious behavior by some of the system's participants. Several works make their contributions in this field by extending consensus solving algorithms to the Byzantine model. However, few works have aligned this goal with the previously stated ones, namely allowing fast executions and taking advantage of the commutativity assumption to reduce coordination requirements. This is a sparse are in the distributed protocols literature to which we wanted to contribute.\par
Finally, newer non-crash fault models make the observation that the Byzantine model is very pessimistic in that it allows for up to $f$ processes to display malicious behavior in a coordinated way. In order to reduce the stringent system requirements imposed by the strong faulty behavior permitted to the processes, models like VFT and XFT have introduced specifications that allow the network administrator to parameterize the allowed behavior. These models are especially suited for datacenter environments where not only is the network highly secured and monitored, making coordinated malicious behavior unlikely, but the high amount of required equipment makes protocols with less requirements more attractive. Protocols can take advantage of the realistic assumptions provided by these models to become more viable options in scenarios like the one mentioned previously. Therefore, we also wanted to build a version of the protocol that took advantage of these newer models to be better suited for datacenter environments and perhaps scenarios in which systems are geo-replicated. \par

\section{Achievements}
In line with the previously mentioned goals, this thesis proposes three protocols, each targeting a different fault model. The contributions are added incrementally such that each protocol contains contributions introduced in previous ones, when applicable. \par
The first protocol targets the \acrshort{cft} model and its major contribution is the simplification of the generalized consensus problem to a problem that still takes advantage of the commutativity observation. By specializing generalized consensus into the problem of agreeing of commands histories, the specification of the protocol also becomes easier to understand and less reliant on complicated mathematical formalisms. This is an important contribution since it increases the understandability of both the problem and its solution, making it easier to translate to an actual implementation. This protocol is described extensively both textually and through pseudocode. We also describe an optimization that allows certain commands to be committed with a reduced quorum. This extension further extends the protocol's applicability to scenarios such as datacenter environments.\par
Our second major contribution is the adaptation of Generalized Paxos to the Byzantine fault model. In order to build this protocol, we propose a specification of consensus that builds upon our previous simplified consensus problem by taking into account the possibility of Byzantine behavior. With this problem in mind, we describe \acrlong{bgp}, providing both a textual description and also pseudocode that can help the mapping of the protocol into a practical implementation. Two extensions are proposed to this protocol. The first describes how the protocol can take advantage of universally commutative commands to both commit a sequence using a reduced quorum and also bypass the additional verification phase imposed by the possibility of Byzantine faults. The second extension is a checkpointing feature that allows the protocol to deal with the accumulation of state at both the learners and the acceptors. This extension increases the protocol's applicability in practical scenarios since it allows the system to free up memory space by executing the checkpointing subprotocol when resources demand it. To further convince the reader of this protocol's validity, we also present proofs that argue that the protocol correctly ensures each of the properties described in the consensus specification.\par
The third protocol, \acrlong{vgp}, and the final contribution of this thesis is the adaptation of \acrshort{bgp} to the Visigoth fault model. We believe \acrshort{vft} to be a good fit for Generalized Paxos since VFT's synchrony assumptions and Generalized Paxos commutativity observation have the potential to greatly reduce coordination requirements within a controlled system such as a datacenter. The overall message pattern of \acrshort{vgp} is similar to that of its Byzantine counterpart. However, VFT's synchrony assumptions and their effects on quorum sizes have non-obvious consequences on the resolution of the command history problem. For this contribution, the description focuses mostly on the problems introduced by VFT's assumptions and how the protocol is modified to deal with them. The protocol is accompanied by a pseudocode description and correctness proofs, both structured similarly to their analogous versions in \acrshort{bgp}. \par
A final minor contribution that precedes the previously mentioned protocols and consensus problems can be found in Chapter \ref{problem}. This chapter attempts to describe some important but complicated aspects of the original generalized consensus and Generalized Paxos description. This is intended to help familiarize the reader with the complexities of the works upon which our contributions try to build on and may help shed some light on some of Generalized Paxos' more opaque components.

\section{Future work}
