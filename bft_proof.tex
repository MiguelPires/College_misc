\subsection{Correctness Proofs}

This section argues for the correctness of the Byzantine Generalized Paxos protocol in terms of the specified consensus properties.\par


\begin{table}[h!]
	\renewcommand{\arraystretch}{1.5}
	\centering
	\begin{tabularx}{\linewidth}{ |c|X|}
		%\hline
		%\multicolumn{2}{|c|}{Notation}\\
		\hline
		Invariant/Symbol & Definition \\
		\hline
		$\thicksim$ & Equivalence relation between sequences \\
		\hline
		$\mathcal{L}$ & Set of learner processes \\
		\hline
		$X \overset{e}{\implies} Y$ & $X$ implies that $Y$ is eventually true \\
		\hline
		$learned_{l_i}$ & Learner $l_i$'s $learned$ sequence of commands \\
		\hline
		$learned(l_i,b,s)$ & $learned_{l_i}$ contains sequence $s$ at the end of ballot $b$  \\
		\hline
		$maj\_accepted(b,s)$ & At least $N-f$ acceptors broadcasted their acceptance votes to the learners\\
		\hline
		$verified(b,c)$ & At least $N-f$ learners have verified that command $c$ commutes with any other command learned in ballot $b$\\
		\hline
	\end{tabularx} 
	\caption{Proof notation} 
	\label{table:1}
\end{table}

\subsubsection{Consistency}
\begin{theorem}
At any time and for any two learners $l_i$ and $l_j$, $learned_{l_i}$ and $learned_{l_j}$ can subsequently be extended to equivalent sequences \label{C-T1}
\end{theorem} 
\textbf{Proof:} \par
1. At the end of a ballot, any two learned sequences are equivalent \par
\indent\indent\textbf{Proof:} Lemma \ref{C-L4} \par
2. At any moment of a ballot, $learned_{l_i}$ and $learned_{l_j}$ are extendable to equivalent sequences \par
\indent\indent\textbf{Proof:} By 1, if any two learned sequences are equivalent at the end of a ballot then, by definition, they are extendable to equivalent sequences\par
3. Q.E.D \par
\indent\indent\textbf{Proof:} By 1 and 2\par

\begin{lemma}
At the end of a ballot, $learned_{l_i} \thicksim learned_{l_j},\ \forall\ i,j \in \mathcal{L}$ \label{C-L4} \par
\end{lemma}
\textbf{Proof:} \par
1.  $learned(l_j,b,s) \overset{e}{\implies} learned(l_i,b,s),\ \forall\ lj, l_i \in \mathcal{L}$\par
\indent\indent\textbf{Proof:} Lemma \ref{C-L3} \par
2. Eventually, $learned_{l_i} \thicksim learned_{l_j}$  \par
\indent\indent\textbf{Proof:} If two learned sequences contain the same commands (1) and they are all commutative (Lemma \ref{C-L5}), they sequences are equivalent.\par
3. Q.E.D \par
\indent\indent\textbf{Proof:} By 1 and 2\par

\begin{lemma}
	$learned(l_j,b,s) \overset{e}{\implies} learned(l_i,b,s),\ \forall\ lj, l_i \in \mathcal{L}$ \label{C-L3} \par
\end{lemma}
\textbf{Proof:} \par
1. $learned(l_j,b,s) \implies maj\_accepted(b,s)$\par
\indent\indent\textbf{Proof:} Lemma \ref{C-L1} \par
2. $maj\_accepted(b,s) \overset{e}{\implies} learned(l_i,b,s)$ \par
\indent\indent\textbf{Proof:} Lemma \ref{C-L2} \par
%3. $learned(l_j,b,s) \overset{e}{\implies} learned(l_i,b,s),\ \forall\ l_j,l_i \in \mathcal{L}$ \par
4. Q.E.D \par
\indent\indent\textbf{Proof:} By 1 and 2\par


\begin{lemma}
$learned(l_i,b,s) \land learned(l_j,b,s') \land \mathcal{S} = \mathcal{S'} \implies s \thicksim s'$ \todo{What I mean with this is that $s$ and $s'$ must have the same elements, just not necessarily in the same order} \label{C-L5} \par
\end{lemma}
\textbf{Proof:} By Algorithm 4 lines \{10-16, 35-42\}, two commands or sequences of commands are only learned in the same ballot if they commute.


\begin{lemma}
$learned(l_i,b,s) \implies maj\_accepted(b,s),\ \forall\ l_i \in \mathcal{L} $ \label{C-L1} \par
\end{lemma} 

\textbf{Proof:} By Algorithm \ref{BFT-Acc} lines \{14-16\} and Algorithm \ref{BFT-Learn} lines \{8-14\}, if a learner learned a sequence $s$ in a ballot $b$ then a majority of acceptors must have executed phase $2b$ for $s$ in the same ballot.

\begin{lemma}
$maj\_accepted(b,s) \overset{e}{\implies} learned(l_i,b,s),\ \forall\ l_i \in \mathcal{L}$ \label{C-L2} \par
\end{lemma}
\textbf{Proof:} By Algorithm \ref{BFT-Acc} lines \{10-16, 26-42\} and Algorithm \ref{BFT-Learn} lines \{8-14\}, when a quorum of $N-f$ acceptors accepts a sequence $s$ in ballot $b$, eventually s will be learned by any correct learner during ballot $b$.


\subsubsection{Non-Triviality}
\begin{theorem}
$learned_l$ can only contain proposed commands \label{N-T1} \par
\end{theorem} 

\textbf{Proof:} By Lemma \ref{N-L1}, a learner only learns commands or sequences of commands for which a Byzantine quorum of acceptors have voted. By Lemma \ref{N-L2}, a Byzantine quorum of votes can only be reached if the value is proposed by some proposer.

\begin{lemma}
$learned(l_j,b,s) \implies maj\_accepted(b,s)$ \label{N-L1} \par
\end{lemma}

\textbf{Proof:} By Algorithm \ref{BFT-Learn} lines \{1-6\} and Algorithm \ref{BFT-Acc} lines \{16-21;23-28\}, a learner can only learn a command or sequence of commands if it has been accepted by a Byzantine quorum of acceptors.

\begin{lemma}
$maj\_accepted(b,s) \implies proposed(b,s)$ \label{N-L2} \par
\end{lemma}

\textbf{Proof:} By Algorithm \ref{BFT-Acc} lines \{7-8;16-21\}, in a classic ballot, or by Algorithm \ref{BFT-Acc} lines \{1-2;23-28\}, in a fast ballot, a majority of accepting votes can only be reached if it was sent by a proposer (in the case of a classic ballot, a \textit{distinguished} proposer).

\subsubsection{Stability}
\begin{theorem}
If $learned_l = v$ then, at all later times, $v \sqsubseteq learned_l$, for any $l$ and $v$ \par \label{S-T1}
\end{theorem} 
\textbf{Proof:} In a fast ballot, proposers can only propose increments to be appended to the learned command history and, therefore, learned sequences in fast ballots are always extensions of previous sequences. In classic ballots, for a learner to learn a sequence that isn't an extension of a previous learned sequence, at least $2f+1$ acceptors would have to vote for it and for that to happen the leader would have to propose it. The subsequent reasoning depends on whether the leader is faulty or not.\par
If the leader is Byzantine it can propose an invalid sequence. However, since the acceptors check if the last accepted sequence is a prefix of the proposal, it wouldn't reach $2f+1$ votes. In the situation where the leader tries to ignore commands accepted in the last ballot, at most $f$ faulty acceptors would vote for it due to maliciousness and another $f$ would vote for it because they didn't participate in the latest ballot, which wouldn't be enough to learn the sequence. \par
If the leader is not Byzantine, the acceptors could still report altered sequences when the leader attempts to read the state in \textit{phase 1/view-change}. For instance, $f$ faulty acceptors could maliciously report a sequence without the latest command and another $f$ could have not participated in the last ballot which only leaves a single acceptor that would report the latest correct sequence. This single correct value would be indistinguishable from an incorrect one. To prevent this, the leader signs proposals with his private key before sending them to acceptors. When a leader (possibly not the one that signed the last proposal) reds the state from the acceptors, he checks the signature to ensure that the sequence wasn't maliciously manufactured. At most, an acceptor could report an out-of-date but correct value. This means that $2f$ acceptors could report out-of-date sequence which still leaves one correct and updated sequence. Since the signatures assure us that this value isn't manufactured and the quorum intersection properties assure us that the acceptor participated in the latest ballot, the value must be correct and the leader can append his proposal to the value to which every other value is a prefix.