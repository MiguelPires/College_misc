\section{Discussion}
% and Concluding remarks}
\label{sec:disc}

\subsection{Extending the protocol to universally commutative commands}

% Generalized Paxos vs Paxos spec.
The Generalized Paxos specification dictates that the only commands
whose relative order can be reversed at different learners are those
that commute with each other. This restriction has the advantage of
allowing for running a replicated state machine on top of the
Generalized Paxos protocol, while still ensuring that the state
machine behaves in a way that is indistinguishable from running it on
top of the original Paxos protocol.

% Extension to ops that commute with all
However, the downside of this use of commutative operations in the
context of Generalized Paxos is that this commutativity check is done
at runtime, and, if two non-commutative operations are issued
concurrently in the same ballot, then we must fall back to the slower
classic Paxos protocol.

This raises the possibility of extending the protocol to handle
commands that are universally commutative, i.e., commute with every
other commands. For these commands, it is known before executing them
that they will not generate any conflicts, and therefore it is not
necessary to check them against concurrently executing commands.  This
allows us to optimize the protocol by decreasing the number of phase
$2b$ messages required to learn to a smaller $f+1$ quorum. Since, by
definition, these sequences are guaranteed to never produce conflicts,
the $N-f$ quorum is not required to prevent learners from learning
conflicting sequences. Instead, a quorum of $f+1$ is sufficient for
the learner to be sure that the proposed sequence was proposed by a
correct proposer. (Furthermore, for read-only commands that do not
change the state of the state machine, it is possible to run these
against one single learner in the crash model, since there is no
requirement to withstand $f$ faults for these commands.)

% Weakly consistent replication - also diff order but ok to have different v

\subsection{Generalized Paxos and weak consistency}


%The Byzantine Generalized Paxos protocol tackles two challenges in two different avenues of research, fault tolerance and relaxed consistency models. By specifying the generalized consensus problem,
The key distinguishing feature of the specification of Generalized
Paxos~\cite{Lamport2005} is allowing learners to learn concurrent
proposals in a different order, when the proposals commute. This idea
is closely related to the work on weaker consistency models like eventual or
causal consistency~\cite{Ahamad1995}, or consistency models that mix
strong and weak consistency levels like RedBlue~\cite{Li2012}, which attempt
to decrease the cost of executing operations by reducing coordination
requirements between replicas. 

The link between the two models becomes clearer after we introduce the
notion of universally commutative commands from the previous subsection.
In the case of weakly consistent replication (or multi-level replication),
weakly consistent requests can be executed as if they were universally
commutative, even if in practice that may not be the case. E.g., checking 
the balance of a bank account and making a deposit do not commute since
the output of the former depends on their relative order. However,
some systems prefer to run both as weakly consistent operations, even
though it may cause executions that are not explained by a sequential
execution of all commands, since the semantics are still acceptable given
that the final state that is reached is the same and no invariants 
of the application are violated~\cite{Li2012}.



% Extension to diff replica grps


\subsection{Handling faults in the fast case}

A result that was stated in the original Generalized Paxos
paper~\cite{Lamport2005} is that to tolerate $f$ crash faults and
allow for fast ballots whenever there are up to $e$ crash faults, the
total system size $N$ must uphold two conditions:
\begin{itemize}
	\item $N > 2f$
	\item $N > 2e+f$
\end{itemize} 
Additionally, the fast and classic quorums are of size $N-e$ and $N-f$, respectively, which means there is a dependence relationship between these three concepts: number of tolerated faults, system size, and quorum size: if we not mind keeping $e$ small, then the fast ballot quorum will be a larger fraction of the total system size, which itself will be smaller. Conversely, if we wish to allow fast ballots to progress in the presence of the maximum number of faults, we can set $e=f$ which means that $N >3f$ and $N-e=N-f$. 

An interesting observation stemming from our results is that, since Byzantine fault tolerance also requires a total system size of $N>3f$ and a quorum size of $2f+1$, we are able to amortize the cost of both features, i.e., we are able to tolerate the maximum number of faults for fast execution while not increasing the overal replication factor.

%\textbf{Optimizations} One possible optimization of the Byzantine Generalized Paxos protocol leverages universally commutative commands or sequences of commands, which we define as sequences which commute with any other. Universally commutative sequences allows us to reduce latency by decreasing the number of phase $2b$ messages required to learn to a smaller $f+1$ quorum. Since, by definition, these sequences are guaranteed to never produce conflicts, the $N-f$ quorum isn't required to prevent learners from learning conflicting sequences. Instead, a $f+1$ quorum is sufficient for the learner to be sure that the proposed sequence was proposed by a correct proposer. 


\section{Conclusion}

In this paper, we presented a simplified description of the Generalized Paxos protocol and specification, which is meant to pave the way for new avenues of research in this area. In addition, we present a Byzantine fault tolerant version of the protocol, and we prove the correctness of this protocol. In the future, we would like to implement this protocol and fully evaluate it, in addition to gaining a better understanding of its practical applicability.
