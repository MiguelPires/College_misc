% Importance of Paxos, initially theoretical, now at the heart of
% data center infrastructure, namely smr and a special instance of smr
% called coordination. Such a transition took decades, arguably due
% to a dense presentation and non-trivial transformation of certain
% aspects of protocol
The evolution of the Paxos~\cite{Lam98} protocol is an unique
chapter in the history of Computer Science. It was first described in
1989 through a technical report~\cite{paxos:tr}, and was only
published a decade later~\cite{Lam98}. Another long wait took
place until the protocol started to be studied in depth and used by
researchers in various fields, namely the distributed
algorithms~\cite{DPLL97} and the distributed systems~\cite{petal}
research communities. And finally, another decade later, the protocol
made its way to the core of the implementation of the services that
are used by millions of people over the Internet, in particular since
Paxos-based state machine replication is the key component of Google's
Chubby lock service~\cite{chubby}, or the open source ZooKeeper
project~\cite{zookeeper}, used by Yahoo!\ among others. Arguably, the
complexity of the presentation may have stood in the way of a faster
adoption of the protocol, and several attempts have been made at
writing more concise explanations of
it~\cite{L01,Renesse2011}.


% Recently, interesting developments came along in space of both Paxos
% . On Paxos land you have interesting
% new variants, most notably generalized Paxos which extends fast paxos
% to allow for fewer msg steps / need commut / change in spec
More recently, several variants of Paxos have been proposed and
studied. Two important lines of research can be highlighted in this
regard. First, a series of papers hardened the protocol against
malicious adversaries by solving consensus in a Byzantine fault
model~\cite{Martin2006,Lamport2011}. The importance of this line of
research is now being confirmed as these protocols are now in widespread
use in the context of cryptocurrencies and distributed ledger
schemes such as blockchain~\cite{bitcoin}.
Second, many proposals target improving
the Paxos protocol by eliminating communication costs~\cite{L06},
including an important evolution of the protocol called Generalized
Paxos~\cite{Lamport2005}, which has the noteworthy aspect of
having lower communication costs by leveraging a more general specification than traditional consensus that can lead to a weaker requirement in terms of ordering of commands across replicas. In particular, instead of forcing all
processes to agree on the same value, it allows processes to pick an
increasing sequence of commands that differs from process to process
in that commutative commands may appear in a different order.
The practical importance of such weaker specifications is underlined
by a significant research activity on the corresponding weaker consistency
models for replicated systems~\cite{LLS90,dynamo}.
%(This evolution has a parallel to the recent trend in studying replicated
%systems that offer consistency models that are weaker than


% [SKIP parag] smr has also interesting trend - weak consistency, motivated
% by availabiltiy and perf, and made even more important as replicas
% are more spread and less well connected

% We argue that, similarly to the clarification was helpful to bring
% Paxos to the light of day, we need the same for generalized Paxos
% a clarification and connection to practice
In this paper, we draw a parallel between the evolution of the Paxos
protocol and the current status of Generalized Paxos. In particular,
we argue that, much in the same way that the clarification of the Paxos
protocol contributed to its practical adoption, it is also important
to simplify the description of Generalized Paxos. Furthermore, we believe
that evolving this protocol to the Byzantine model is an important
task, since it will contribute to the understanding
and also open the possibility of adopting Generalized Paxos in
scenarios such as a Blockchain deployment.

As such, the paper makes several contributions, which are listed next.
%
\begin{itemize}
\item
We present a simplified version of the specification of Generalized
Consensus, which is focused on the most commonly used case of the
solutions to this problem, which is to agree on a sequence of
commands;

\item
we extend the Generalized Paxos protocol to the Byzantine fault model; 

\item
we present a description of the Byzantine Generalized Paxos protocol
that is more accessible than the original description, namely including the
respective pseudocode, in order to to make it 
easier to implement;

\item
we prove the correctness of the Byzantine Generalize Paxos protocol;

\item
and we discuss several extensions to the protocol in the context of relaxed consistency models and fault tolerance.
\end{itemize}
%
The remainder of the paper is organized as follows: 
Section~\ref{sec:related} is a detailed overview of Paxos and the respective family of related protocols.
Section~\ref{sec:model} introduces the model and simplified specification of Generalized Paxos.
Section~\ref{sec:protocol} presents the Generalized Paxos protocol that is resilient against Byzantine failures along with the corresponding proof sketches. 
Section~\ref{sec:disc} discusses several optimizations and practical considerations, and Section~\ref{sec:conc} concludes the paper. In Appendix~\ref{bft_code} we present the pseudocode for the protocol and appendix~\ref{proof} presents the respective correctness proofs.
%
