\section{Background and Related Work}
 
\subsection{Paxos and its Variants} \label{Paxos} 

The Paxos protocol family solves consensus by circumventing the well-known FLP impossibility result \cite{FLP85}. It does this by making the observation that, while consensus is unsolvable in asynchronous systems, most of the time systems can be considered synchronous since delays are sporadic and temporary. Therefore, as long as consistency is guaranteed regardless of synchrony, Paxos can forgo of progress during the temporary periods of asynchrony. Using this insight, Paxos ensures consistency even when the system is asynchronous but can only guarantee progress while it is synchronous and no more than $f$ faults occur for a system of $2f+1$ replicas \cite{L01}. The classic form of Paxos employs a set of proposers, acceptors and learners and runs in a sequence of ballots. To ensure progress during synchronous periods, proposals are serialized by a distinguished proposer, the leader.\par
Paxos is most commonly deployed as Multi(Decree)-Paxos, which provides an optimization of the basic message pattern by omitting the first phase of messages from all but the first ballot for each leader \cite{Renesse2011}. This means that a leader only needs to send a \textit{phase 1a} message once and subsequent proposals may be sent directly in \textit{phase 2a} messages without requesting a promise from the acceptors. This reduces the message pattern in the common case from five message delays to just three (from proposal to learning). Since there are no implications on the quorum size or guarantees provided by Paxos, the reduced latency comes at no additional cost. \par

Fast Paxos observes that it's possible to improve on previous results by allowing proposers to propose values directly to acceptors \cite{L06}. To this end, the protocol distinguishes between fast and classic ballots, where fast ballots bypass the leader by sending proposals directly to acceptors and classic ballots work as in Basic Paxos. The fast ballots' reduced latency comes at the additional cost of using a quorum size of $n-e$ instead of a classic majority quorum, where $e$ is the number of faults that can be tolerated while using fast ballots. In addition to the usual requirement that $n> 2f$, to ensure that fast and classic quorums intersect, a new requirement must be met: $n > 2e+f$. This means that if we wish to tolerate the same number of faults for classic and fast ballots (i.e., $e=f$), then the total number of replicas is $3f+1$ instead of the usual $2f+1$ and the quorum size for fast and classic ballots is the same. The optimized commit scenario occurs during fast ballots, in which only two messages broadcasts are necessary: \textit{phase 2a} messages between a proposer and the acceptors, and \textit{phase 2b} messages between acceptors and learners. This creates the possibility of two proposers concurrently proposing values to the acceptors and generating a conflict which must be resolved using coordinated or uncoordinated recovery. \par

Generalized Paxos addresses Fast Paxos' shortcomings regarding collisions. More precisely, it allows acceptors to accept different sequences of commands as long as non-commutative operations are totally ordered \cite{Lamport2005}.  Non-commutativity between operations is generically represented as an interference relation. Generalized Paxos abstracts the traditional consensus problem of agreeing on a single value to the problem of agreeing on an increasing set of values. \textit{C-structs} provide this increasing sequence abstraction and allow us to define different consensus problems. If we define the sequence of learned commands of a learner $l_i$ as a \textit{c-struct} $learned[l_i]$, then the consistency requirement for consensus can be defined as:\par
\textbf{Consistency} $learned[l_1]$ and $learned[l_2]$ are always compatible, for all learners $l_1$ and $l_2$. \par
For two \textit{c-structs} to be compatible, they must have a \textit{common upper bound}. This means that, for any two learned \textit{c-structs} such as $learned[l_1]$ and $learned[l_2]$, there must exist some \textit{c-struct} to which they are both prefixes. This prohibits non-commutative commands from being concurrently accepted because no subsequent \textit{c-struct} would extend them both since it wouldn't have a total order of non-commutative operations. Defining \textit{c-structs} as command histories enables acceptors to agree on different sequences of commands and still preserve consistency as long as dependence relationships are not violated. This means that commutative commands can be ordered differently regarding each other but interfering commands must preserve the same order across each sequence at any learner. This guarantees that solving the consensus problem for histories is enough to implement a state-machine replicated system. \par

Mencius is also a variant of Paxos that tries to address the bottleneck of having a single leader through which every proposal must go through. In Mencius, the leader of each round rotates between every process. The leader of round $i$ is the process $p_k$, such that $k = n\ mod\ i$.  Leaders with nothing to propose can skip their turn by proposing a \textit{no-op}. If a leader is slow or faulty, the other replicas can execute \textit{phase 1} to revoke the leader's right to propose a value but they can only propose a \textit{no-op} instead \cite{Mao2008}. Considering that non-leader replicas can only propose \textit{no-ops}, a \textit{no-op} command from the leader can be accepted in a single message delay since there is no chance of another value being accepted. If some non-leader server revokes the leader's right to propose and suggests a \textit{no-op}, then the leader can still suggest a value $v \neq$ \textit{no-op} that will eventually be accepted as long as $l$ isn't permanently suspected. The usage of an unreliable failure detector is enough to guarantee that eventually all faulty processes, and only those, will be suspected. Mencius also takes advantage of commutativity by allowing out-of-order commits, where values $x$ and $y$ can be learned in different orders by different learners if there isn't a dependence relationship between them. Experimental evaluation confirms that Mencius is able of scaling to a higher throughput than Paxos. \par

Egalitarian Paxos (EPaxos) extends Mencius' goal of achieving a better throughput than Paxos by removing the bottleneck caused by having a leader. Additionally, EPaxos also achieves optimal commit latency in certain conditions and graceful performance degradation if replicas fail or become slow \cite{Moraru2013}. To avoid choosing a leader, the proposal of commands for a command slot is done in a decentralized manner, taking advantage of the commutativity observations made by Generalized Paxos \cite{Lamport2005}. If two replicas unknowingly propose commands concurrently, one will commit its proposal in one round trip after getting replies from a quorum of replicas. However, some replica will see that another command was concurrently proposed and may interfere with the already committed command. If the commands are non-commutative then the replica must reply with a dependency between the commands, committing its command in two rounds trips. This commit latency is achieved by using a \textit{fast-path quorum} of $f+\lfloor\frac{f+1}{2}\rfloor$ replicas. Similarly to Mencius, EPaxos achieves a substantially higher throughput than Multi-Paxos, especially when $2\%$ of commands interfere with each other. However, since, unlike Mencius, replicas don't have to wait for previous replicas to propose their commands, commit latency is significantly lower.

\subsection{Non-crash Fault Models} \label{Non-Crash}
Non-crash fault models emerged to cope with the effect of malicious attacks and software errors. These models (e.g., the arbitrary fault model) assume a stronger adversary than previous crash fault models. The Byzantine Generals Problem is defined as a set of Byzantine generals that are camped in the outskirts of an enemy city and have to coordinate an attack. Each general can either decide to attack or retreat and there may be $f$ traitors among the generals that try to prevent the loyal generals from agreeing on the same action. The problem is solved if every loyal general agrees on what action to take \cite{LSP82}. Like the traitorous generals, the Byzantine adversary is one that may force a faulty replica to display arbitrary behaviour and even coordinate multiple faulty replicas in an attack. \par
PBFT is a protocol that solves consensus for state machine replication while tolerating up to $f$ Byzantine faults \cite{CL99}. The system moves through configurations called \textit{views} in which one replica is the primary and the remaining replicas are the backups. The safety property of the algorithm requires that operations be totally ordered. The protocol starts when a client sends a request for an operation to the primary, which in turn assigns a sequence number to the request and multicasts a \textit{pre-prepare} message to the backups. If a backup replica accepts the pre-prepare message, it multicasts a \textit{prepare} message and adds both messages to its log. The prepare message is similar to the pre-prepare message except that it doesn't contain the client's request message. Both of these phases are needed to ensure that the requested operation is totally ordered at every correct replica, therefore satisfying the protocol's safety property. After receiving $2f$ prepare messages, a replica multicasts a \textit{commit} message and commits the message to its log when it has received $2f$ commit messages from other replicas. The liveness property requires that clients must eventually receive replies to their requests, provided that there are at most $\lfloor\frac{n-1}{3}\rfloor$ faults and the transmission time doesn't increase continuously. This property represents a weak liveness condition but one that is enough to circumvent Fischer, Lynch and Paterson's impossibility result \cite{FLP85}. Backups can trigger new views if they suspect the leader to be Byzantine. \par