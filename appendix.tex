\section{Crash-fault Tolerant Protocol}
\label{app:cft}
\section{Crash Fault Model} \label{Crash Fault Model}

This section describes the crash fault tolerant version of the Generalized Paxos protocol for our simplified problem. The only modifications applied to the protocol were made to make it simpler while still ensuring its correctness. The protocol should still be recognizable as Generalized Paxos since its message pattern and control flow remain the same. However, we chose to describe it in detail, both in the interest of clarity and also so that in the description of later iterations of the protocol we can focus on the aspects that differ. 

\subsection{Agreement protocol} 

The consensus protocol allows learner processes to agree on equivalent sequences of commands (according to our previous definition of equivalence).
An important conceptual distinction between the original Paxos protocol and Generalized Paxos is that, in the original Paxos, each instance of consensus is called a ballot and agrees upon a single value, whereas in Generalized Paxos, instead of being separate instances of consensus, ballots correspond to an extension to the sequence of learned commands of a single ongoing consensus instance.
In both protocols, ballots can either be \textit{classic} or \textit{fast}. \par

In classic ballots, a leader proposes a single sequence of commands, such that it can be appended to the commands learned by the learners. 
A classic ballot in Generalized Paxos follows a protocol that is very similar to the one used by classic Paxos~\cite{Lam98}. This protocol comprises a first phase where each acceptor conveys to the leader the sequences that the acceptor has already voted for (so that the leader can resend commands that may not have gathered enough votes).
This is followed by a second phase where the leader picks an extension to the sequence of previously proposed commands and broadcasts it to the acceptors. The acceptors send their votes to the learners, who then, after gathering enough support for a given extension to the current sequence, append the new commands to their own sequences of learned commands and discard the already learned ones.\par

In fast ballots, multiple proposers can concurrently propose either single commands or sequences of commands by sending them directly to the acceptors. (We use the term \textit{proposal} to denote either the command or sequence of commands that was proposed.)
In this case, concurrency implies that acceptors may receive proposals in a different order. If the resulting sequences are equivalent, then the fast ballots are successfully learned in two message delays. If not, the protocol must fall back to using a classic ballot.

Next, we present the protocol for each type of ballot in detail.

\subsection{Classic ballots} 

Classic ballots work in a way that is very close to the original Paxos protocol~\cite{Lam98}. Therefore, throughout our description, we will highlight the points where Generalized Paxos departs from that original protocol, in particular where it's due to behaviors that are particular to the our simplified specification of Generalized Paxos.

In this part of the protocol, the leader continuously collect proposals by assembling commands that received from the proposers in a sequence. This sequence is built by appending arriving proposals to a sequence containing every proposal received since the previous ballot. (This differs from classic Paxos, where it suffices to keep a single proposed value that the leader attempts to reach agreement on.)

When the next ballot is triggered, the leader starts the first phase by sending phase $1a$ messages to all acceptors containing just the ballot number. Similarly to classic Paxos, acceptors reply with a phase $1b$ message to the leader, which reports all sequences of commands they voted for. This message also carries a promise not to participate in lower-numbered ballots, in order to prevent safety violations~\cite{Lam98}.

After gathering a quorum of $N-f$ phase $1b$ messages, the leader initiates phase $2a$ by sending a message with a proposal to the acceptors. This proposal is assembled by appending the sequences gathered from the proposers to a sequence that contains every command in the sequences that were previously accepted by the acceptors in the quorum (instead of sending a single value with the highest ballot number in the classic specification).

The acceptors reply to phase $2a$ messages by sending phase $2b$ messages to the learners, containing the ballot and the proposal from the leader. After receiving $N-f$ votes for a sequence, a learner learns it by extracting the commands that are not contained in his $learned$ sequence and appending them in order. Note that for a sequence to be learned, a learner doesn't have to receive $N-f$ votes for the exact same sequence but for equivalence sequences (in accordance to our previous definition of equivalence).

\subsection{Fast ballots} 

In contrast to classic ballots, fast ballots are able to leverage the weaker specification of generalized consensus (compared to classic consensus) in terms of command ordering at different replicas, to allow for the faster execution of commands in some cases.

The basic idea of fast ballots is that proposers contact the acceptors directly, bypassing the leader, and then the acceptors send their vote for the current sequence directly to the learners, where this sequence now incorporates the proposed value. If a learner can gather $N-f$ votes for a sequence (or an equivalent one), then it is learned. If, however, a conflict exists between sequences then they will not be considered equivalent and at most one of them will gather enough votes to be learned. Conflicts are dealt with by maintaining the proposals at the acceptors so they can be sent to the leader and learned in the next classic ballot. This differs from Fast Paxos where recovery is performed through an additional round-trip. \par
Next, we explain each of these steps in more detail.


\noindent {\bf Step 1: Proposer to acceptors.}
To initiate a fast ballot, the leader informs both proposers and acceptors that the proposals may be sent directly to the acceptors. Unlike classic ballots, where the sequence proposed by the leader consists of the commands received from the proposers appended to previously proposed commands, in a fast ballot, proposals can be sent to the acceptors in the form of either a single command or a sequence to be appended to the command history. These proposals are sent directly from the proposers to the acceptors.

\noindent {\bf Step 2: Acceptors to learners.}
Acceptors append the proposals they receive to the proposals they have previously accepted in the current ballot and broadcast the result to the learners. Similarly to what happens in classic ballots, the fast ballot equivalent of the phase $2b$ message, which is sent from acceptors to learners, contains the current ballot number and the command sequence. However, since commands (or sequences of commands) are concurrently proposed, acceptors can receive and vote for non-commutative proposals in different orders. To ensure safety, correct learners must learn non-commutative commands in a total order. To this end, a learner must gather $N-f$ votes for equivalent sequences. That is, sequences do not necessarily have to be equal in order to be learned since commutative commands may be reordered. Recall that a sequence is equivalent to another if it can be transformed into the second one by reordering its elements without changing the order of any pair of non-commutative commands. (Note that, in the pseudocode, equivalent sequences are being treated as belonging to the same index of the \emph{messages} variable, to simplify the presentation.) By requiring $N-f$ votes for a sequence of commands, we ensure that, given two sequences where non-commutative commands are differently ordered, only one sequence will receive enough votes. {\color{red} Depends on the system size? Is it 2f+1 or what?} Since each acceptor will only vote for a single sequence, there are only enough correct processes to commit one of them. Note that the fact that proposals are sent as extensions of previous sequences is critical to the safety of the protocol. In particular, since the votes from acceptors can be reordered by the network before being delivered at the learners, if these values were single commands it would be impossible to guarantee that non-commutative commands would be learned in a total order. \par

\noindent \textbf{Arbitrating an order after a conflict.} When, in a fast ballot, non-commutative commands are  concurrently proposed, these commands may be incorporated into the sequences of various acceptors in different orders, and therefore the sequences sent by the acceptors in phase $2b$ messages will not be equivalent and will not be learned. In this case, the leader subsequently runs a classic ballot and gathers these unlearned sequences in phase $1b$. Then, the leader will assemble a single serialization for every previously proposed command, which it will then send to the acceptors. Therefore, if non-commutative commands are concurrently proposed in a fast ballot, they will be included in the subsequent classic ballot and the learners will learn them in a total order, thus preserving consistency. \par 
Note that the leader may assemble gathered sequences in an order that is non-commutative to some sequence that has previously gathered enough votes to be learned. Since the leader can only wait $N-f$ phase $1b$ messages from acceptors, there is no way guarantee which sequence may have been learned. However, this is not a problem because, by making learners process phase $2b$ messages in order, if some sequence gathered enough votes to be learned then its commands will be learned before the new non-commutative sequence is and the new-sequence's repeated commands will be discarded.
\begin{table}[h!]
	\renewcommand{\arraystretch}{1.5}
	\centering
	\begin{tabular}{ |c|c|}
		\hline
		\multicolumn{2}{|c|}{Notation}\\
		\hline
		Symbol & Definition \\
		\hline
		$\sqcup$ & least upper bound \\
		\hline
		$\sqcap$ & greatest lower bound \\
		\hline
		$\bullet$ & append operator \\
		\hline
	\end{tabular} 
	\caption{Notation for the pseudocode} 
	\label{table:1}
\end{table}

\begin{algorithm}
	\caption{Generalized Paxos - Proposer p}
	\begin{algorithmic}[1]
		
		\Function{propose}{\textit{C}}
		\If{fast\_ballot}
		\State \Call{send}{$fast, C$} to Acceptors;
		\Else
		\State \Call{send}{\textit{propose, C}} to Leader;
		\EndIf
		\EndFunction
		
		
	\end{algorithmic}
\end{algorithm}

\begin{algorithm}
	\caption{Generalized Paxos - Leader l}
	\textbf{Local variables:} $ballot_l = 0,\ maxTried_l = \bot,\ C_l = \bot, messages = \bot$
	\begin{algorithmic}[1]
		\State \textbf{upon} \textit{receive(propose, C)} from proposer $p_i$ \textbf{do} 
		\State \hspace{\algorithmicindent} $C_l = C$;
		\State \hspace{\algorithmicindent} \Call{phase\_1a}{$ $};
		\State
		\State \textbf{upon} \textit{receive(statement)} from acceptor $a_i$ \textbf{do}
		\State \hspace{\algorithmicindent}  $messages[ballot_l][a_i] = statement$;
		\State \hspace{\algorithmicindent} \textbf{if} $\#(messages) \geq \lfloor \frac{N}{2} \rfloor +1$ \textbf{then} 
		\State \hspace{\algorithmicindent}\hspace{\algorithmicindent} \Call{phase\_2a}{$ballot_l, Q$};
		\State
		%\item[] % unnumbered empty line
		\Function{phase\_1a}{}
		\State \Call{send}{$p1a, ballot_l$} to Acceptors;
		\EndFunction
		
		\State
		\Function{phase\_2a}{$bal, Q$}
		\State $maxTried_l$ = \Call{proved\_safe}{$Q, bal$};
		\State $maxTried_l = maxTried_l \bullet C_l$;
		\State \Call{send}{$p2a,ballot_l, maxTried_l$} to Acceptors;
		\EndFunction
		
		\State
		\Function{proved\_safe}{$ballot$}
		\State $maxTried = \bot$;
		\State \textbf{for} $val$ in $message[ballot]$ \textbf{do}
		\State \hspace{\algorithmicindent} \textbf{if} $maxTried = \bot$ or $\Call{is\_prefix}{val, maxTried}$ \textbf{then}
		\State\hspace{\algorithmicindent}\hspace{\algorithmicindent} $maxTried = val$;
		\State \textbf{end for}
		\State \textbf{return} $maxTried$;
		\EndFunction
		
			\State
		\Function{is\_prefix}{$new\_sequence, old\_sequence$}
		\If {$\lVert old\_sequence \rVert > \lVert new\_sequence \rVert$}
		\State \textbf{return} \textit{false};
		\EndIf
		
		\State
		\item[] outer:	
		\State \textbf{for} $c_{old}$ in $old\_sequence$ \textbf{do}
		\State \hspace{\algorithmicindent} \textbf{for} $c_{new}$ in $new\_sequence$ \textbf{do}
		\State\hspace{\algorithmicindent}\hspace{\algorithmicindent} \textbf{if} $c_{old} = c_{new}$ \textbf{then}
		\State \hspace{\algorithmicindent}\hspace{\algorithmicindent}\hspace{\algorithmicindent} \textbf{continue} \textit{outer};
		
		\State \hspace{\algorithmicindent}\hspace{\algorithmicindent} \textbf{else if} !\Call{are\_commutative}{$c_{old}, c_{new}$} \textbf{then}
		\State\hspace{\algorithmicindent}\hspace{\algorithmicindent}\hspace{\algorithmicindent} \textbf{return} \textit{false};
		\State \hspace{\algorithmicindent}\textbf{end for}
		\State \hspace{\algorithmicindent}\textbf{return} \textit{false};
		\State \textbf{end for}
		\State
		\State \textbf{return} \textit{true};
		\EndFunction
		
	\end{algorithmic}
\end{algorithm}

\begin{algorithm}
	\caption{Generalized Paxos - Acceptor a}
	\textbf{Local variables: } $bal_a = 0,\ val_a = \bot$ 
	\begin{algorithmic}[1]
		
		\State \textbf{upon} \textit{receive(fast, val)} from proposer \textit{p} \textbf{do}
		\State \hspace{\algorithmicindent} \Call{phase\_2b\_fast}{$val$};
		
		\State
		\State \textbf{upon} \textit{receive(p1a, ballot)} from leader \textit{l} \textbf{do}
		\State \hspace{\algorithmicindent} \Call{phase\_1b}{$ballot, l$};
		
		\State
		\State \textbf{upon} \textit{receive(p2a, ballot, value)} from leader \textit{l} \textbf{do}
		\State \hspace{\algorithmicindent} \Call{phase\_2b\_classic}{$ballot, value$};
		
		\State
		\Function{phase\_1b}{$m, l$}
		\If {$bal_a < m$}
		\State \Call{send}{$p1b, m, bal_a, val_a$} to leader l;
		\State $bal_a = m$;
		\EndIf
		\EndFunction
		
		\State
		\Function{phase\_2b\_classic}{$m, v$}
		\State $k = max(i\ |\ (i < m) \wedge (\exists a \in Q :\ val_a[i]\ \neq null))$;
		\If {$m \geq k$}
		\State $val_a = v$;
		\State \Call{send}{$p2b, bal_a, val_a$} to learners;
		\EndIf
		\EndFunction
		
		\State
		\Function{phase\_2b\_fast}{$v$}
		\State $k = max(i\ |\ (i < m) \wedge (\exists a \in Q :\ val_a[i]\ \neq null))$;
		\If {$bal_a == k$}
		\State $val_a = val_a \bullet v$;
		\State \Call{send}{$p2b, bal_a, val_a$} to learners;
		\EndIf
		\EndFunction
		
	\end{algorithmic}
\end{algorithm}

\begin{algorithm}
	\caption{Generalized Paxos - Learner l}
	\textbf{Local variables: } $learned = \bot, messages = \bot$ 
	\begin{algorithmic}[1]
		\State \textbf{upon} \textit{receive($p2b, bal, val$)} from acceptor $a_i$ \textbf{do}
		\State \hspace{\algorithmicindent} $messages[bal][a_i] = val$;
		\State \hspace{\algorithmicindent} \textbf{if} {($fast\_ballot$ and $\#(messages[bal]) \geq \lceil \frac{3N}{4} \rceil$) or
			\State \hspace{\algorithmicindent} \hspace{\algorithmicindent}	($classic\_ballot$ and $\#(messages[bal]) \geq \lfloor \frac{N}{2}\rfloor+1$)} \textbf{then}
		\State \hspace{\algorithmicindent} \hspace{\algorithmicindent} \hspace{\algorithmicindent} $learned = learned \sqcup val$;
	\end{algorithmic}
\end{algorithm}


\clearpage
\section{Byzantine Fast Quorums - OUTDATED}
To adapt Generalized Paxos to the Byzantine setting, the fast quorums must be recalculated to ensure agreement. Generalized Consensus' Approximate Theorem 3 states that any two fast quorums, $Q_{f1}$ and $Q_{f2}$, and a classic quorum $Q_c$, must have a non-empty intersection. To adapt the protocol to the Byzantine scenario, this intersection can't only be non-empty, it also needs to be larger than $f$ replicas. We obtain the minimum quorum size by forcing the intersection between quorums to be larger than $f$ in the worst case scenario. In the worst case, given two fast quorums of size $Q_f$ and a classic quorum of size $Q_c$, $Q_c$ would intersect with all the replicas of the fast quorums that don't intersect with each other and with some replicas of the fast quorums that do intersect. We name $x$ as the intersection between the three quorums. To ensure agreement, we need to ensure that the intersection between the classic quorum and the two fast quorums, $x$, will have to be larger than $f$. Therefore, the following statements must always hold:

\begin{gather*}
\begin{cases}
Q_c \geq N - Q_f + N-Q_f + x \\
x > f \\
Q_c = \lceil \frac{N+f}{2}\rceil
\end{cases} \\ 
\end{gather*}

The first equation states that the classic quorum is composed of the replicas of the fast quorums that don't intersect with each other plus some replicas that do intersect. The second equation forces the intersection to always be larger than $f$ and the third equation is the minimum quorum size for classic quorums. To solve the system, we can substitute the $Q_c$ in the first equation by $\lceil \frac{N+f}{2}\rceil$ and we obtain:
\begin{gather*} \\
Q_c \geq N - Q_f + N-Q_f + x \label{eq_1} \tag{1} \\ 
\lceil\frac{N+f}{2}\rceil \geq 2N - 2Q_f + x \label{eq_2} \tag{2} \\
N+f \geq 4N - 4Q_f + 2x \label{eq_3} \tag{3} \\
Q_f \geq \frac{3N+2x-f}{4} \label{eq_4} \tag{4} \\ 
\text{Since $x > f$} \implies Q_f \geq \frac{3N+f+1}{4} \label{eq_5} \tag{5}  \\
\end{gather*}

Note that when transitioning from \eqref{eq_2} to \eqref{eq_3}, the ceiling operator was ignored. However, this is safe because, for any $x$, $y$ and $z$, if $\frac{x}{y} > z$ then $\lceil \frac{x}{y} \rceil > z$ holds.\par

