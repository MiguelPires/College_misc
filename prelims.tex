\section{Preliminaries}

To adapt Generalized Paxos to the Byzantine setting, the fast quorums must be recalculated to ensure agreement. Generalized Consensus' Approximate Theorem 3 states that any two fast quorums, $Q_{f1}$ and $Q_{f2}$, and a classic quorum $Q_c$, must have a non-empty intersection. To adapt the protocol to the Byzantine scenario, this intersection can't only be non-empty, it also needs to be larger than $f+1$ replicas. Therefore, the following reasoning can be applied to derive the size of the fast quorums. Given two fast quorums of size $Q_f$ and classic quorum $Q_c$, in the worst case scenario $Q_c$ would intersect with the replicas of the fast quorums that don't intersect with each other and some replicas of the fast quorums that do intersect. To ensure agreement, the intersection between the classic quorum and the two fast quorums will have to be larger than $f$. We name the intersection as $x$. Therefore, the following statements must always hold:

$\\ \begin{cases}
		Q = N - Q_f + N-Q_f + x \\
		x > f \\
		Q = \lceil \frac{N+f}{2}\rceil
\end{cases} \\$

By solving the system, we get:

$\lceil\frac{N+f}{2}\rceil = 2N - 2Q_f + x \\
N+f = 4N - 4Q_f + 2x \\
Q_f = \frac{3N+2x-f}{4} \\ 
\text{Since $x > f$} \implies Q_f \geq \frac{3N+f+1}{4} \\$


\par
Note that the ceiling is discarded for simplicity but the implications of that aren't fully accounted for yet.
